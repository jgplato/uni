\documentclass[a4paper,german,12pt]{article}
\usepackage[T1]{fontenc}
\usepackage[utf8]{inputenc}
\usepackage{babel}
\usepackage{tikz}
\usepackage{geometry}
\usepackage{amsmath}
\geometry{a4paper, top=25mm, left=20mm, right=40mm, bottom=20mm, headsep=10mm, footskip=12mm}
\linespread{1.5}
\setlength\parindent{0pt}
\begin{document}

Mathematik für Physiker I, Wintersemester 2012/2013, 1. Übungsblatt

Markus Fenske und Florian Neumeyer, Tutor: Stephan Schwartz

\section*{Aufgabe 1.1}
\subsection*{Teil a)}

\section*{Aufgabe 1.2}

Wir betrachten komplexen Zahlen hier geometrisch und in Polarform. Sofern $z_0$
im Ursprung liegt muss $z_2$, um mit $z_0$ und $z_1$ ein gleichseitiges Dreieck
zu bilden die um 60 Grad um den Ursprung gedrehte Abbildung von $z_1$ sein.

Sollte $z_0$ nicht im Ursprung liegen, wird das Referenzsystem durch eine
Verschiebungstransformation so geändert, dass $z_0$ im Ursprung liegt, die
Drehung durchgeführt und die Transformation rückgängig gemacht.

Konkret wird also zuerst um $-z_0$ verschoben, dann durch Multiplikation mit
der noch zu findenden komplexen Zahl $w$ die Zahl $z_1$ um 60 Grad gedreht und
anschließend wieder um $z_0$ verschoben. Das Ergebnis ist $z_2$. Die Gleichung
sieht also so aus:

$$z_2 = (z_1 - z_0) \cdot w + z_0$$

Da $w$ eine Drehung ohne Streckung ausführen soll, muss $|w| = 1$. Somit ist $w
= e^{i\phi}$ mit $\phi = \frac{\pi}{3}$ (Bogenmaß von 60 Grad). Somit $w =
\cos(\frac{\pi}{3}) + i\sin(\frac{\pi}{3}) = \frac{1}{2} +
\frac{\sqrt{3}{2}}i$

Die Formel lautet also:

$$z_2 = \left(\frac{1}{2} + \frac{\sqrt{3}}{2}i\right) \left(z_1 - z_0\right)  + z_0$$

\section*{Aufgabe 1.3}
\subsection*{Teil a)}

Wir suchen $$\left|\frac{z-1}{z+1}\right| = 1, z \in \mathbf{C}$$

\begin{align}
\left|\frac{z-1}{z+1}\right| &= 1 \\
\frac{|z-1|}{|z+1|} &= 1
\end{align}

Wir substituieren $z = x+yi$ wobei $x,y \in \mathbf{R}$, dann ist der Betrag
$|z| = \sqrt{x^2+y^2}$

\begin{align}
\frac{\sqrt{(x-1)^2+y^2}}{\sqrt{(x+1)^2+y^2}} &= 1 \\
\frac{(x-1)^2+y^2}{(x+1)^2+y^2} &= 1^2 \\
(x-1)^2+y^2 &= (x+1)^2+y^2 \\
(x-1)^2 &= (x+1)^2 \\
x^2-2x+1 &= x^2+2x+1 \\
-2x &= +2x
\end{align}

Dafür existriert nur eine Lösung: $x = 0$. Somit hat $\frac{z-1}{z+1}$ nur dann den
Betrag 1, wenn $\operatorname{Re}(z) = 0$. Geometrisch gedacht verlaufen
alle Lösungen also direkt auf der imaginären Aches.

\subsection*{Teil b)}

Nicht bearbeitet.

\subsection*{Teil c)}

Nicht bearbeitet.


\end{document}

