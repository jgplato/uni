\documentclass[a4paper,german,12pt,smallheadings]{scrartcl}
\usepackage[T1]{fontenc}
\usepackage[utf8]{inputenc}
\usepackage{babel}
\usepackage{tikz}
\usepackage{geometry}
\usepackage{amsmath}
\usepackage{amssymb}
\usepackage{float}
\usepackage{wrapfig}
\usepackage[thinspace,thinqspace,squaren,textstyle]{SIunits}
\restylefloat{table}
\geometry{a4paper, top=15mm, left=20mm, right=40mm, bottom=20mm, headsep=10mm, footskip=12mm}
\linespread{1.5}
\setlength\parindent{0pt}
\begin{document}
\begin{center}
\bfseries % Fettdruck einschalten
\sffamily % Serifenlose Schrift
\vspace{-40pt}
Experiementalphysik I, Wintersemester 2012/2013, 10. Übungsblatt

Normen Peulecke und Markus Fenske, Tutor: Alex Krüger
%Markus Fenske, Tutor: Alex Krüger
\vspace{-10pt}
\end{center}

\section*{36. Kinetische Gastheorie}

Zuerst bestimmen wir, wie viele Atome jeweils gestoßen werden, wenn ein
Helium-Atom durch ein Volumen aus feststehenden Helium-Atomen fliegt. Ein Stoß
passiert, sofern der Abstand zwischen zwei Atomen geringer als das doppelte der
jeweiligen Durchmesser ist.

Die Anzahl der gestoßenenen Atome auf der Länge $l$ ist also gleich der Anzahl der Atome, die
sich in einem Zylinder mit dem Radius $2r$ und der Länge $l$ befinden. Das Volumen ist dann:

\begin{align*}
  V &= A \cdot l \\
    &= 2\pi(2r)^2 l \\
    &= 8\pi r^2 l
\end{align*}

Die Anzahl der Atome in diesem Zylinder mit dem Volumen $V$ lässt sich aus der
thermischen Zustandsgleichung herleiten.

\begin{align*}
  PV &= NkT \\
  N &= \frac{PV}{kT} \\
  N &= \frac{P \cdot 8 \pi r^2 l}{kT}
\end{align*}

Die mittlere freie Weglänge ist dann ($W = N$)

\begin{align*}
  \lambda &= \frac{l}{W} \\
  \lambda &= \frac{l}{\frac{P \cdot 8 \pi r^2 l}{kT}} \\
  \lambda &= \frac{kT}{P \cdot 8 \pi r^2}
\end{align*}

Um die mittlere Zeit zwischen zwei Stößen zu bestimmen, berechnen wir die Geschwindigkeit des Teilchens aus seiner kinetischen Energie.

\begin{align*}
  E &= \frac{3}{2} k T \\
  E &= \frac{1}{2} m v^2 \\
  v &= \sqrt{\frac{3 k T}{m}}
\end{align*}

Die Masse eines einzelnen Teilches ergibt sich aus der molaren Masse. Bei 4
Gramm pro Mol ($N_A$ Teilchen) ergibt sich pro Teilchen eine Masse von $m =
M^*/N_A$. Somit:

\begin{align*}
  v &= \sqrt{\frac{3 k T N_A}{M^*}} \\
  v &= \sqrt{\frac{3 RT}{M^*}}
\end{align*}

Die Zeit $\tau$ zwischen zwei Stößen ist dann gegeben durch 

\begin{equation*}
\tau = \frac{\lambda}{v}
\end{equation*}

Durch Einsetzen der gegebenen Werte ergibt sich für die mittlere freie Weglänge
$\lambda$ im Fall a $\lambda = 1{,}64 \cdot 10^-5 \meter$ und im Fall b
$\lambda = 1{,}64 \cdot 10^8 \meter$. Die Geschwindigkeit ist in beiden Fällen
aufgrund der gleichen Temperatur $v = 43{,}10 \meter\per\second$. Damit ergibt
sich für die Zeit zwischen zwei Stößen im Fall a $\tau = 3{,}80 \cdot 10^{-7}
\second$ und im Fall b $\tau =  3{,}80 \cdot 10^{6} \second$.


\section*{37. Wärmeleitung}

Für den Wärmestrom gilt $\dot Q = \frac{A \lambda \delta T}/l$. Für 0{,}1
\meter Länge und 0{,}01 \meter Durchmesser ergibt sich:

\begin{align*}
  \dot Q &= \frac{\pi 0{,}01 \cdot 380 \cdot 100}{0{,}1} \watt
         &\approx 11937 \watt
\end{align*}

Mit 0{,}01 \meter Länge und 0{,}1 \meter Durchmesser ergibt sich:

\begin{align*}
  \dot Q &= \frac{\pi 0{,}1 \cdot 380 \cdot 100}{0{,}01} \watt
         &\approx 1194 \kilo\watt
\end{align*}

\section*{38. Innere Energie}

Die durchschnittliche kinetische Energie eines Teilchens ist gegeben durch
$\epsilon = \frac{3}{2} kT$. Die Anzahl der Teilchen durch $N = \frac{PV}{kT}$.
Damit ergibt sich für die Gesamtenergie:

\begin{align*}
  E &= N \cdot \epsilon \\
    &= \frac{PV}{kT} \frac{3}{2} kT \\
    &= \frac{3}{2} PV
\end{align*}

Der Druck ist über die Zeit konstant, dann durch das Schlüsselloch findet ein
Druckausgleich statt. Das Volumen des Zimmers ändert sich ebenfalls nicht. Die
innere Energie der Luft, die im Zimmer verbleibt, ist konstant. Ich bin mir
jedoch nicht sicher, ob das die Frage war.

Sollte sich die Frage auf die Energie beziehen, die die Luft hat, die vor der
Erwärmung im Zimmer war, dann lässt sich diese Energie über die Volumenänderung
berechnen.

\begin{align*}
  \frac{PV_2}{NkT_2} &= \frac{PV_1}{NkT_1} \\
  \frac{V_2}{T_2} &= \frac{V_1}{T_1} \\
  V_2 &= V_1 \frac{T_2}{T_1}
\end{align*}

Somit erhöht sich die Energie um

\begin{align*}
  \delta E &= \frac{3}{2} (PV_2 - PV_1) \\
  \delta E &= \frac{3}{2} \left(PV_1 \frac{T_2}{T_1} - PV_1\right) \\
  \delta E &= \frac{3}{2} PV_1 \left(\frac{T_2}{T_1} - 1\right)
\end{align*}

Mit $P = 101300 \pascal$, $V_1 = 50 \cubic\meter$, $T_1 = 278 \kelvin$, $T_2 =
295 \kelvin$ ergibt sich eine Erhöhung um $\approx 9292 \joule$.

\section*{39. Zustandsänderung}
\subsection*{a. Arbeit}

Die Arbeit lässt sich berechnen durch $\delta W_{rev} = -\int_1^2 PdV$. Da im
PV-Diagramm eine gerade Linie sein soll, gilt $P \propto V$. Somit also $P(V) =
\alpha V$. Aus den Werten in $\pascal$ und $\cubic\meter$ lässt sich $\alpha = \frac{100000}{0{,}025} = 4\cdot 10^6$ ablesen. Somit ist
das Integral:

\begin{align*}
  \delta W_{rev} &= -\int_1^2 PdV \\
                 &= -4 \cdot 10^6 \int_{0{,}025}^{0{,}075} V dV \\
                 &= -4 \cdot 10^6 \left[\frac{1}{2}V^2\right]_{0{,}025}^{0{,}075} \\
                 &= -2 \cdot 10^6 \left(0{,}025^2-0{,}075^2\right) \\
                 &= 10 \kilo\joule
\end{align*}

\subsection*{b. Temperatur}

Gemäß $PV = nRT$ muss gelten

\begin{align*}
  \delta T &= \frac{P_2V_2 - P_1V_1}{nR}
\end{align*}

Mit den gegebenen Werten ($P_1 = 100000 \pascal$, $V_1 = 0{,}025 \cubic\meter$,
$P_2 = 300000 \pascal$, $V_2 = 0{,}075 \cubic\meter$, $n = 1 \operatorname{mol}$ ergibt sich
eine Temperaturänderung von $\delta T \approx 2405 \kelvin$.

\end{document}
