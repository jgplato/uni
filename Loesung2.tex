\documentclass[a4paper,german,12pt,smallheadings]{scrartcl}
\usepackage[T1]{fontenc}
\usepackage[utf8]{inputenc}
\usepackage{babel}
\usepackage{tikz}
\usepackage{geometry}
\usepackage{amsmath}
\usepackage{amssymb}
\usepackage{float}
%\usepackage{wrapfig}
\usepackage[thinspace,thinqspace,squaren,textstyle]{SIunits}
\restylefloat{table}
\geometry{a4paper, top=15mm, left=20mm, right=40mm, bottom=20mm, headsep=10mm, footskip=12mm}
\linespread{1.5}
\setlength\parindent{0pt}
\begin{document}
\begin{center}
\bfseries % Fettdruck einschalten
\sffamily % Serifenlose Schrift
\vspace{-40pt}
Elektrodynamik und Optik, Sommersemester 2013, 3. Blatt \\
Markus Fenske, Tutor: Dr. Marko Wietstruk
\vspace{-10pt}
\end{center}
\section*{Aufgabe 1}
\subsection*{Teil a}

Homogene Ladungsverteilungen auf einer Kugel verhalten sich von außerhalb der
Kugel betrachtet wie eine Punktladung im Zentrum. Dementsprechend:

\begin{align*}
  &E = \frac{q}{4 \pi \epsilon_0r^2} \\
  \Leftrightarrow \quad&q = \frac{E 4 \pi \epsilon_0}{r^2}
\end{align*}

Einsetzen der Werte:

\begin{align*}
  q &= \frac{3 \cdot 10^6 \cdot 4 \cdot \pi \cdot 8{,}86 \cdot 10^{-12}}{0{,}05^2} \coulomb \\
    & \approx 0{,}13 \;\coulomb
\end{align*}

\subsection*{Teil b}

Homogene Ladungsverteilungen auf einer Kugel verhalten sich von außerhalb der
Kugel betrachtet wie eine Punktladung im Zentrum. FÜr Potential zwischen einem
Punkt in unendlichem Abstand und einem Punkt im Abstand zu einer Punktladung gilt:

\begin{align*}
  \Phi &= \frac{E}{r} = \frac{3 \cdot 10^6}{0{,}05} \volt = 60 \;\mega\volt
\end{align*}

\subsection*{Teil c}

In einem kubischen Atomgitter befindet sich ein Atom pro Gittereinheit. Die Atomdichte ist
diesem Fall also $\frac{1}{0{,}25^3 \;\nano\meter^3} = \frac{1}{1{,}56 \cdot 10^{-29} \;\meter^3}$.

Für das Volumen der Hohlkugel gilt:

\begin{align*}
  V &= \frac{4}{3} \pi (r_a^3 - r_i^3), \quad r_a = 50 \;\milli\meter, r_i = 47\;\milli\meter \\
    &= 8{,}87 \cdot 10^{-5}\; \meter^3 \\
\end{align*}

Die Anzahl der Atome ist dann

\begin{align*}
  n &= V \cdot \frac{n}{V} = 8{,}87 \cdot 10^{-5} \cdot \frac{1}{1{,}56 \cdot 10^{-29} \;\meter^3} \\
    &= 5{,}69 \cdot 10^{24}
\end{align*}

Die Anzahl der Elektronen in der Ladung ist

\begin{align*}
  n_e &= \frac{0{,}133 \;\coulomb}{1{,}6 \cdot 10^{-19} \coulomb} \\
      &= 8{,}3 \cdot 10^{17}
\end{align*}

Das bedeutet, dass bei der gegebenen Ladung etwa $10^{7}$-mal mehr Atome im
Hohlkopf enthalten sind, als Elektronen auf der Oberfläche.


In der obersten Schicht ist ein Atom auf einer Oberfläche von jeweils $(0{,}25
\;\nano\meter)^2 = 6{,}25 \cdot 10^{-20} \meter^2$. Die Oberfläche des
Hohlkopfes ist

\begin{align*}
  A = 4 \pi r^2 = 4 \pi \cdot 0{,}05^2 \meter^2 \approx 3{,}1 \cdot 10^{-2} \meter^2
\end{align*}

Die Anzahl der Atome auf der Oberfläche ist dann

\begin{align*}
  n &= A \cdot \frac{n}{A} \\
    &= 3{,}1 \cdot 10^{-2} \meter^2 \cdot \frac{1}{6{,}25 \cdot 10^{-20} \;\meter^2} \\
    &\approx 5{,}0 \cdot 10^{17}
\end{align*}

Die Anzahl der Elektronen auf der Oberfläche liegt trotzdem noch um den Faktor
$1{,}6$ über diesem Wert.

\section*{Aufgabe 4}

\begin{align*}
  C_{\operatorname{gesamt}} = \frac{1}{\frac{1}{C_1} + \frac{1}{C_2 + C_3}} = 0{,}1\overline{3} \;\micro\farad
\end{align*}


\end{document}
