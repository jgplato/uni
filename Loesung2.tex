\documentclass[a4paper,german,12pt,smallheadings]{scrartcl}
\usepackage[T1]{fontenc}
\usepackage[utf8]{inputenc}
\usepackage{babel}
\usepackage{tikz}
\usepackage{geometry}
\usepackage{amsmath}
\usepackage{amssymb}
\usepackage{float}
\usepackage{wrapfig}
\usepackage[thinspace,thinqspace,squaren,textstyle]{SIunits}
\restylefloat{table}
\geometry{a4paper, top=15mm, left=20mm, right=40mm, bottom=20mm, headsep=10mm, footskip=12mm}
\linespread{1.5}
\setlength\parindent{0pt}
\begin{document}
\begin{center}
\bfseries % Fettdruck einschalten
\sffamily % Serifenlose Schrift
\vspace{-40pt}
Analysis I, Sommersemester 2013, 2. Übungsblatt
Luis Herrmann und Markus Fenske, Tutor: Adam Schienle
\vspace{-10pt}
\end{center}

\section*{Aufgabe 2.1}
\subsection*{Teil a}
Wenn $f, g$ beschränkt sind, gilt $|f(x)| \le K_1 \all x$ und $|g(x)| \le K_2
\all x$. Addieren der beiden Ungleichungen:

\begin{align*}
  |f(x)| + |g(x)| \le K_1 + K_2 \all x \\
\end{align*}
Wegen der Dreieckesungleichung ($|a + b| \le |a| + |b|$) gilt dann:
\begin{align*}
  |f(x) + g(x)| \le K_1 + K_2 \all x \\
\end{align*}

Somit ist $f + g$ beschränkt.
\subsection*{Teil b}
\textbf{FIXME} Fehlt % FIXME

\subsection*{Teil c}
Sei $f: x \mapsto x^2$ und $g: x \mapsto x$:
\begin{align*}
  \sup
\end{align*}

\end{document}
