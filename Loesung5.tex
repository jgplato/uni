\documentclass[a4paper,german,12pt,smallheadings]{scrartcl}
\usepackage[T1]{fontenc}
\usepackage[utf8]{inputenc}
\usepackage{babel}
\usepackage{tikz}
\usepackage{geometry}
\usepackage{amsmath}
\usepackage{amssymb}
\usepackage{float}
%\usepackage{wrapfig}
\usepackage[thinspace,thinqspace,squaren,textstyle]{SIunits}
\restylefloat{table}
\geometry{a4paper, top=15mm, left=20mm, right=40mm, bottom=20mm, headsep=10mm, footskip=12mm}
\linespread{1.5}
\setlength\parindent{0pt}
\begin{document}
\begin{center}
\bfseries % Fettdruck einschalten
\sffamily % Serifenlose Schrift
\vspace{-40pt}
Elektrodynamik und Optik, Sommersemester 2013, 5. Blatt \\
Markus Fenske, Tutor: Dr. Marko Wietstruk
\vspace{-10pt}
\end{center}
\section*{Aufgabe 1}
\subsection*{Teil a}

Zur Anwendung der Kirchhoffschen Regel teilen ich den Stromkreis in einen
linken Kreis ($I_1$) und einen rechten Kreis ($I_2$). Der linke Kreis fließe
gegen den Uhrzeigersinn, der rechte Kreis fließe im Uhrzeigersinn.  Gestartet
wird jeweils vom Knotenpunkt $b$.

Somit ergeben sich folgende Gleichungen:

\begin{align*}
  \underbrace{-3I_1+5}_{\text{links}}\underbrace{-3I_2+5}_{\text{rechts}}+7-2I_1 &= 0 \\
  \underbrace{-3I_1+5}_{\text{links}}\underbrace{-3I_2+5}_{\text{rechts}}+1I_2 &= 0
\end{align*}

Die Lösung dieses Gleichungssystems lautet:

\begin{align*}
  I_1 = \frac{38}{11}\;\ampere\approx3{,}5\;\ampere,\quad I_2 = -\frac{1}{11}\;\ampere\approx-0{,}1\;\ampere
\end{align*}

Durch den Widerstand $R_1$ fließt der Strom $I_2$ von oben nach unten, mit dem
Betrag wie oben angegeben. Durch den Widerstand $R_2$ fließt der Strom $I_1$
von links nach rechts, Betrag wie oben angegeben. Durch den Widerstand $R_3$
fließen die Ströme $I_1$ und $I_2$ in die selbe Richtung, nämlich von unten
nach oben. Es wird also addiert. Der Strom durch $R_3$ ist dann
$\frac{37}{11}\;\ampere$.

\subsection*{Teil b}

Die Spannungsdifferenz zwischen Punkt $a$ und Punkt $b$ ist ($I$ sei der Strom
durch $R_3$ (siehe oben)):

\begin{align*}
  R_3 \cdot I - 5\;\volt = \frac{111}{11}\;\volt - \frac{55}{11} = \frac{56}{11}\;\volt \approx 5{,}1\;\volt
\end{align*}

\subsection*{Teil c}
Die Spannungsquelle $U_l$ muss den Strom $I_1$ liefern, die Leistung ist somit:

\begin{align*}
  P_l = U_l \cdot I_1 = \frac{266}{11} \;\watt \approx 24{,}2\;\watt
\end{align*}

Die Spannungsquelle $U_r$ liefert den Strom $I_1 + I_2$:

\begin{align*}
  P_r = U_r \cdot (I_1 + I_2) = \frac{185}{11} \;\watt \approx 16{,}8\;\watt
\end{align*}

\end{document}
