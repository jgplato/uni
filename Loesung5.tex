\documentclass[a4paper,german,12pt,smallheadings]{scrartcl}
\usepackage[T1]{fontenc}
\usepackage[utf8]{inputenc}
\usepackage{babel}
\usepackage{tikz}
\usepackage{geometry}
\usepackage{amsmath}
\usepackage{amssymb}
\usepackage{float}
\usepackage[thinspace,thinqspace,squaren,textstyle]{SIunits}
\restylefloat{table}
\geometry{a4paper, top=15mm, left=20mm, right=40mm, bottom=20mm, headsep=10mm, footskip=12mm}
\linespread{1.5}
\setlength\parindent{0pt}
\begin{document}
\begin{center}
\bfseries % Fettdruck einschalten
\sffamily % Serifenlose Schrift
\vspace{-40pt}
Analytische Mechanik, Sommersemester 2013, 5. Blatt \\
Luis Herrmann und Markus Fenske, Tutor: Clemens Meyer zu Rheda
\vspace{-10pt}
\end{center}
\section*{Aufgabe 1: Vektoren}
\subsection*{Teil a: Beweis der Lagrange-Identität}

Es gelte Einstein'sche Summenkonvention.

\begin{align*}
  &(\vec{a} \times \vec{b}) \cdot (\vec{c} \times \vec{d}) =\\
  &\delta_{ab}(\vec{a} \times \vec{b})_a (\vec{c} \times \vec{d})_b =\\
  &\delta_{ab}(\epsilon_{aij} a_i b_j) (\epsilon_{bkl}c_k d_l) =\\
  &\delta_{ab}\epsilon_{aij}\epsilon_{blk} a_i b_jc_kd_l =\\
  &\epsilon_{aij}\epsilon_{alk} a_i b_jc_kd_l =\\
  &(\delta_{ik}\delta_{jl}-\delta_{il}\delta_{jl}) a_i b_j c_k d_l =\\
  &(\delta_{ik}a_ic_k \delta_{jl}b_jd_l)-(\delta_{il}a_id_l\delta_{jk}b_jc_k) =\\
  &(\vec{a} \cdot \vec{c})(\vec{b} \cdot \vec{d})-(\vec{a}\cdot\vec{d})(\vec{b} \cdot \vec{c})
\end{align*}


\end{document}
