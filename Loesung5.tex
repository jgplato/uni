\documentclass[a4paper,german,12pt,smallheadings]{scrartcl}
\usepackage[T1]{fontenc}
\usepackage[utf8]{inputenc}
\usepackage{babel}
\usepackage{tikz}
\usepackage{geometry}
\usepackage{amsmath}
\usepackage{amssymb}
\usepackage{float}
\usepackage[thinspace,thinqspace,squaren,textstyle]{SIunits}
\restylefloat{table}
\geometry{a4paper, top=15mm, left=20mm, right=40mm, bottom=20mm, headsep=10mm, footskip=12mm}
\linespread{1.5}
\setlength\parindent{0pt}
\begin{document}
\begin{center}
\bfseries % Fettdruck einschalten
\sffamily % Serifenlose Schrift
\vspace{-40pt}
Analytische Mechanik, Sommersemester 2013, 5. Blatt \\
Luis Herrmann und Markus Fenske, Tutor: Clemens Meyer zu Rheda
\vspace{-10pt}
\end{center}
\section*{Aufgabe 1: Vektoren}
\subsection*{Teil a: Beweis der Lagrange-Identität}

Bekanntlich ist $\vec{a} \cdot (\vec{b} \times \vec{c}) = \det(\vec{a},
\vec{b}, \vec{c}) = (\vec{a} \times \vec{b}) \cdot \vec{c}$. Das nutzen wir.

\begin{align*}
  &(\vec{a} \times \vec{b}) \cdot (\vec{c} \times \vec{d}) =\\
  &\vec{a} \cdot (\vec{b} \times (\vec{c} \times \vec{d})) =
\end{align*}

Mithilfe der ``Bac-Cab-Regel'' erhalten wir dann:

\begin{align*}
  &\vec{a} \cdot (\vec{b} \times (\vec{c} \times \vec{d})) =\\
  &\vec{a} \cdot (\vec{c}(\vec{b} \cdot \vec{d}) - \vec{d} (\vec{b} \cdot \vec{c})) =\\
  &(\vec{a} \cdot \vec{c})(\vec{b} \cdot \vec{d}) - (\vec{a} \cdot \vec{d}) (\vec{b} \cdot \vec{c})
\end{align*}

\subsection*{Teil b: Nabla-Produktregel}

Wir beweisen rückwärts. Dabei müssen wir berücksichtigen, dass $\vec{\nabla}$
sowohl ein Vektor, als auch ein Differentialoperator ist. Das bedeutet, dass es
auf Terme wirkt, die (innerhalb eines Produkts) rechts von ihm stehen. Damit
wir die Gleichung nicht auf Komponentenschreibweise aufdröseln müssen, ändern
wir diese Konvention und unterstreichen einfach die Terme, auf die $\nabla$
wirkt. Ausgangsgleichung ist:

\begin{align*}
  (\vec{a}\cdot\vec{\nabla})\vec{b} + (\vec{b} \cdot \vec{\nabla})\vec{a} + \vec{a} \times (\vec{\nabla} \times \vec{b}) + \vec{b} \times (\vec{\nabla} \times \vec{a}) =
\end{align*}

Mit der ``Bac-Cab-Regel'' erhalten wir für den dritten Term $\vec{a} \times
(\vec{\nabla} \times \vec{b}) = \vec{\nabla}(\vec{a}\cdot\vec{b}) -
\vec{b}(\vec{a}\cdot\vec{\nabla})$. Diesen setzen wir ein und uns sehen, dass
dabei die Terme wegheben.

\begin{align*}
  &(\vec{a}\cdot\vec{\nabla})\vec{b} + (\vec{b} \cdot \vec{\nabla})\vec{a} + \vec{\nabla}(\vec{a}\cdot\vec{b}) - \vec{b}(\vec{a}\cdot\vec{\nabla}) + \vec{b} \times (\vec{\nabla} \times \vec{a}) = \\
  &(\vec{b} \cdot \vec{\nabla})\vec{a} + \vec{\nabla}(\vec{a}\cdot\vec{b}) + \vec{b} \times (\vec{\nabla} \times \vec{a}) =
\end{align*}

Genauso verfahren wir mit $\vec{b} \times (\vec{\nabla} \times \vec{a}) = \vec{\nabla}(\vec{b} \cdot \vec{a}) - \vec{a}(\vec{b} \cdot \vec{\nabla})$

\begin{align*}
  &(\vec{b} \cdot \vec{\nabla})\vec{a} + \vec{\nabla}(\vec{a}\cdot\vec{b}) +  \vec{\nabla}(\vec{b} \cdot \vec{a}) - \vec{a}(\vec{b} \cdot \vec{\nabla}) =\\
  &\vec{\nabla}(\vec{a}\cdot\vec{b}) + \vec{\nabla}(\vec{b} \cdot \vec{a}) =\\
\end{align*}

\end{document}
