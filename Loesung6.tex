\documentclass[a4paper,german,12pt,smallheadings]{scrartcl}
\usepackage[T1]{fontenc}
\usepackage[utf8]{inputenc}
\usepackage{babel}
\usepackage{tikz}
\usepackage{geometry}
\usepackage{amsmath}
\usepackage{amssymb}
\usepackage{float}
\usepackage[thinspace,thinqspace,squaren,textstyle]{SIunits}
\restylefloat{table}
\geometry{a4paper, top=15mm, left=20mm, right=40mm, bottom=20mm, headsep=10mm, footskip=12mm}
\linespread{1.5}
\setlength\parindent{0pt}
\begin{document}
\begin{center}
\bfseries % Fettdruck einschalten
\sffamily % Serifenlose Schrift
\vspace{-40pt}
Analytische Mechanik, Sommersemester 2013, 6. Blatt \\
Luis Herrmann und Markus Fenske, Tutor: Clemens Meyer zu Rheda
\vspace{-10pt}
\end{center}
\section*{Aufgabe 1: Doppelpendel}

Sei $(x_i, y_i)$ die Position der Masse $m_i$ relativ zum Ursprung. Die
$y$-Achse zeige nach oben, die $x$-Achse zeige nach rechts.

\begin{align*}
  T &= \frac{m_1}{2} (\dot{x}_1^2 + \dot{y}_1^2) + \frac{m_2}{2} (\dot{x}_2^2 + \dot{y}_2^2) \\
  V &= mgy_1 + mgy_2
\end{align*}

Die Zwangsbedingungen ergeben sich aus der zeitlichen Konstanz von $l_i$, so
dass wir mit den beiden unabhängigen Koordinaten $\phi_i$ arbeiten können.

\begin{align*}
  x_1 =  l_1 \sin \phi_1 \\
  y_1 = -l_1 \cos \phi_1 \\
  x_2 =  l_1 \sin \phi_1 + l_2 \sin \phi_2 \\
  y_2 = -l_1 \cos \phi_1 - l_2 \cos \phi_2
\end{align*}

Mit den Zeitableitungen:

\begin{align*}
  \dot{x_1} = l_1 \dot{phi_1} \cos \phi_1 \\
  \dot{y_1} = l_1 \dot{phi_1} \sin \phi_1 \\
  \dot{x_2} = l_1 \dot{phi_1} \cos \phi_1 + l_2 \dot{phi_2} \cos \phi_2\\
  \dot{y_2} = l_1 \dot{phi_1} \sin \phi_1 + l_2 \dot{phi_2} \sin \phi_2
\end{align*}

Eingesetzt in $T$:

\begin{align*}
  T &= \frac{m_1}{2} (l_1^2 \dot{\phi}_1^2 \cos^2 \phi_1 + l_1^2 \dot{\phi}_1^2 \sin^2 \phi_1) +
       \frac{m_2}{2} (l_1^2 \dot{\phi}_1^2 \cos^2 \phi_1 + l_1^2 \dot{\phi}_1^2 \sin^2 \phi_1 +
                      l_2^2 \dot{\phi}_2^2 \cos^2 \phi_2 + l_2^2 \dot{\phi}_2^2 \sin^2 \phi_2) \\
    &= \frac{m_1}{2} (l_1^2 \dot{\phi}_1^2) + \frac{m_2}{2} (l_1^2 \dot{\phi}_1^2 + l_2^2 \dot{\phi}_2^2) \\
    &= \frac{m_1 + m_2}{2} (l_1^2 \dot{\phi}_1^2) + \frac{m_2}{2} (l_2^2 \dot{\phi}_2^2) \\
\end{align*}

Eingesetzt in $V$:
\begin{align*}
  V = -mgl_1 \cos \phi_1 - m_2g(l_1 \cos \phi_1 + l_2 \cos \phi_2)
\end{align*}

Insgesamt:
\begin{align*}
  L = T - V = \frac{m_1 + m_2}{2} (l_1^2 \dot{\phi}_1^2) + \frac{m_2}{2} (l_2^2 \dot{\phi}_2^2) + mgl_1 \cos \phi_1 + m_2g(l_1 \cos \phi_1 + l_2 \cos \phi_2)
\end{align*}

Durch Anwendung von $\partial_t \partial_{\dot{\phi}} - \partial_\phi$ erhalten wir für $\phi_1$:

\begin{align*}
  &(m_1 + m_2) l_1^2 \ddot{\phi_1} + m_1g l_1 \sin \phi_1 + m_2g l_1 \sin \phi_2 = 0 \\
  \Leftrightarrow\quad& (m_1 + m_2) l_1^2 \ddot{\phi_1} + (m_1 + m_2) gl_1 \sin \phi_1  = 0 \\
  \Leftrightarrow\quad& (m_1 + m_2) (l_1^2 \ddot{\phi_1} + gl_1 \sin \phi_1)  = 0 \\
\end{align*}

Und für $\phi_2$:

\begin{align*}
  m_2 l_2^2 \ddot{\phi}_2 + m_2 g l_2 \sin \phi_2 = 0
\end{align*}

\end{document}
