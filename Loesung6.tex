\documentclass[a4paper,german,12pt,smallheadings]{scrartcl}
\usepackage[T1]{fontenc}
\usepackage[utf8]{inputenc}
\usepackage{babel}
\usepackage{tikz}
\usepackage{geometry}
\usepackage{amsmath}
\usepackage{amssymb}
\usepackage{float}
%\usepackage{wrapfig}
\usepackage[thinspace,thinqspace,squaren,textstyle]{SIunits}
\restylefloat{table}
\geometry{a4paper, top=15mm, left=20mm, right=40mm, bottom=20mm, headsep=10mm, footskip=12mm}
\linespread{1.5}
\setlength\parindent{0pt}
\begin{document}
\begin{center}
\bfseries % Fettdruck einschalten
\sffamily % Serifenlose Schrift
\vspace{-40pt}
Elektrodynamik und Optik, Sommersemester 2013, 6/7. Blatt \\
Markus Fenske, Tutor: Dr. Marko Wietstruk
\vspace{-10pt}
\end{center}
\section*{Aufgabe 1}
Das Biot-Savart-Gesetz lautet:

\begin{equation}
  \vec{B}(\vec{r}) = \frac{\mu_0 I}{4 \pi} \int_{\text{Leiter}} \frac{(\vec{r} - \vec{r'}) \times \vec{ds}}{|\vec{r}-\vec{r'}|^3}
\end{equation}

Da hier das Magnetfeld im Ursprung berechnet werden soll, ist:


\begin{equation}
  \vec{B} = \frac{\mu_0 I}{4 \pi} \int_{\text{Leiter}} \frac{(-\vec{r'}) \times \vec{ds}}{|\vec{r'}|^3}
\end{equation}

$\vec{r'}$ ist der der Vektor zum jeweiligen Punkt auf dem Leiter, während
$\vec{ds}$ entlang des Leiters zeigt. Für das gerade Leiterstück ist $\vec{ds}$
also parallel zu $\vec{r}$, womit $\vec{r} \times \vec{ds} = \vec{0}$. Das
gerade Stück kann also vernachlässigt werden.

\begin{equation}
  \vec{B} = \frac{\mu_0 I}{4 \pi} \int_{\text{Bogen}} \frac{(-\vec{r'}) \times \vec{ds}}{|\vec{r'}|^3}
\end{equation}

Auf dem Kreisbogen steht $\vec{ds}$ senkrecht auf $\vec{r}$. Damit gilt dann
$-\vec{r'} \times \vec{ds} = -\widehat{z} (r' ds)$:

\begin{align}
  \vec{B} &= \frac{\mu_0 I}{4 \pi} \int_{\text{Bogen}} -\widehat{z} \frac{r' ds}{r'^3} \\
  \vec{B} &= \frac{\mu_0 I}{4 \pi} \int_{\text{Bogen}} -\widehat{z} \frac{1}{r'^2} ds \\
\end{align}

Sei nun $R$ der Radius des Kreises, dann ist $|\vec{r'}| = R$. Und somit:

\begin{align}
  \vec{B} = -\frac{\mu_0 I}{4 \pi R^2} \widehat{z}\int_{\text{Bogen}} ds \\
\end{align}

Der Kreis hat den Umfang $2 \pi R$, der Halbkreis dementsprechend die Länge
$\pi R$. Das führt zu:

\begin{align}
  \vec{B} = -\frac{\mu_0 I}{4 \pi R^2} \widehat{z} \pi R \\
\end{align}

Und zusammengefasst:

\begin{equation}
  \vec{B} = -\frac{\mu_0 I}{4 \pi R} \widehat{z}
\end{equation}

Da die $z$-Achse aus der Oberseite des Aufgabenblattes herausragt, stimmt dies
auch mit der Rechte-Faust-Regel überein.

\section*{Aufgabe 3}

\begin{equation}
  \vec{p}_m = I \vec{A}
\end{equation}

Das magnetische Dipolmoment ist $p_m = I \cdot \vec{A}$. Dabei ist $I$ der
Strom und $\vec{A}$ die Flächennormale der Kreisbahn. Da diese parallel zum
Drehmoment, genau wie zum Drehimpuls ist, reicht es, hier mit Beträgen zu
arbeiten.

\begin{equation}
  p_m = I A
\end{equation}

Für $I$ gilt $I = \frac{dQ}{dt}$. Das Elektron kommt auf seiner Kreisbahn mit
dem Radius $r$ genau einmal pro Umlaufzeit $t$ am Ausgangspunkt vorbei. Der
Strom ist also

\begin{equation}
  I = \frac{-e}{t}
\end{equation}

Die Umlaufzeit auf der Kreisbahn $s$ mit der Geschwindigkeit $v$ ist:

\begin{equation}
  t = \frac{s}{v} = \frac{2 \pi r}{v}
\end{equation}

Eingesetz in die Ausgangsgleichung:

\begin{equation}
  p_m = \frac{-e}{2 \pi r} v A
\end{equation}

Die Kreisfläche ist $A = \pi r^2$. Eingesetzt:

\begin{equation}
  p_m = \frac{-e}{2 \pi r} v \pi r^2 = \frac{-e}{2} v r
\end{equation}

Mit dem Impuls $p = mv$ erhält man:

\begin{equation}
  p_m = \frac{-e}{2m} p r
\end{equation}

Und mit dem Drehimpulsbetrag $L = pr$:

\begin{equation}
  p_m = \frac{-e}{2m} L
\end{equation}

Wenn ich nun $L = \hbar$ einsetze erhalte ich:

\begin{equation}
  p_m = \frac{-e}{2m_e} \hbar \approx -9{,}274 \cdot 10^{-24} \;\joule\per\tesla
\end{equation}

\section*{Aufgabe 7}
\subsection*{Teil a}
Für das Magnetfeld einer langen Spule gilt

\begin{equation}
  B = \mu_0 \mu_r \frac{n}{l} I
\end{equation}

In diesem Fall:

\begin{align*}
  B \approx 60 \;\milli\tesla
\end{align*}

\subsection*{Teil b}
Aus obiger Gleichung ist klar, dass $I$ mit dem Faktor $1200$ multipliziert
werden muss, wenn $\mu$ durch $1200$ dividiert wird. Dementsprechend muss dann
$I = 1200 \cdot 20\;\milli\ampere = 24\;\ampere$ sein.



\end{document}
