\documentclass[a4paper,german,12pt,smallheadings]{scrartcl}
\usepackage[T1]{fontenc}
\usepackage[utf8]{inputenc}
\usepackage{babel}
\usepackage{tikz}
\usepackage{geometry}
\usepackage{amsmath}
\usepackage{amssymb}
\usepackage{float}
%\usepackage{wrapfig}
\usepackage[thinspace,thinqspace,squaren,textstyle]{SIunits}
\restylefloat{table}
\geometry{a4paper, top=15mm, left=20mm, right=40mm, bottom=20mm, headsep=10mm, footskip=12mm}
\linespread{1.5}
\setlength\parindent{0pt}
\begin{document}
\begin{center}
\bfseries % Fettdruck einschalten
\sffamily % Serifenlose Schrift
\vspace{-40pt}
Elektrodynamik und Optik, Sommersemester 2013, 6/7. Blatt \\
Markus Fenske, Tutor: Dr. Marko Wietstruk
\vspace{-10pt}
\end{center}
\section*{Aufgabe 3}

\begin{equation}
  \vec{p}_m = I \vec{A}
\end{equation}

Das magnetische Dipolmoment ist $p_m = I \cdot \vec{A}$. Dabei ist $I$ der
Strom und $\vec{A}$ die Flächennormale der Kreisbahn. Da diese parallel zum
Drehmoment, genau wie zum Drehimpuls ist, reicht es, hier mit Beträgen zu
arbeiten.

\begin{equation}
  p_m = I A
\end{equation}

Für $I$ gilt $I = \frac{dQ}{dt}$. Das Elektron kommt auf seiner Kreisbahn mit
dem Radius $r$ genau einmal pro Umlaufzeit $t$ am Ausgangspunkt vorbei. Der
Strom ist also

\begin{equation}
  I = \frac{-e}{t}
\end{equation}

Die Umlaufzeit auf der Kreisbahn $s$ mit der Geschwindigkeit $v$ ist:

\begin{equation}
  t = \frac{s}{v} = \frac{2 \pi r}{v}
\end{equation}

Eingesetz in die Ausgangsgleichung:

\begin{equation}
  p_m = \frac{-e}{2 \pi r} v A
\end{equation}

Die Kreisfläche ist $A = \pi r^2$. Eingesetzt:

\begin{equation}
  p_m = \frac{-e}{2 \pi r} v \pi r^2 = \frac{-e}{2} v r
\end{equation}

Mit dem Impuls $p = mv$ erhält man:

\begin{equation}
  p_m = \frac{-e}{2m} p r
\end{equation}

Und mit dem Drehimpulsbetrag $L = pr$:

\begin{equation}
  p_m = \frac{-e}{2m} L
\end{equation}

Wenn ich nun $L = \hbar$ einsetze erhalte ich:

\begin{equation}
  p_m = \frac{-e}{2m_e} \hbar \approx -9{,}274 \cdot 10^{-24} \;\joule\per\tesla
\end{equation}

\end{document}
