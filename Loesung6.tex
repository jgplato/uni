\documentclass[a4paper,german,12pt,smallheadings]{scrartcl}
\usepackage[T1]{fontenc}
\usepackage[utf8]{inputenc}
\usepackage{babel}
\usepackage{tikz}
\usepackage{geometry}
\usepackage{amsmath}
\usepackage{amssymb}
\usepackage{float}
\usepackage[thinspace,thinqspace,squaren,textstyle]{SIunits}
\restylefloat{table}
\geometry{a4paper, top=15mm, left=20mm, right=40mm, bottom=20mm, headsep=10mm, footskip=12mm}
\linespread{1.5}
\setlength\parindent{0pt}
\begin{document}
\begin{center}
\bfseries % Fettdruck einschalten
\sffamily % Serifenlose Schrift
\vspace{-40pt}
Analytische Mechanik, Sommersemester 2013, 6. Blatt \\
Luis Herrmann und Markus Fenske, Tutor: Clemens Meyer zu Rheda
\vspace{-10pt}
\end{center}
\section*{Aufgabe 1: Doppelpendel}

Sei $(x_i, y_i)$ die Position der Masse $m_i$ relativ zum Ursprung. Die
$y$-Achse zeige nach oben, die $x$-Achse zeige nach rechts.

\begin{align*}
  T &= T_1 + T_2 = \frac{m_1}{2} (\dot{x}_1^2 + \dot{y}_1^2) + \frac{m_2}{2} (\dot{x}_2^2 + \dot{y}_2^2) \\
  V &= m_1gy_1 + m_2gy_2
\end{align*}

Die Zwangsbedingungen ergeben sich aus der zeitlichen Konstanz von $l_i$, so
dass wir mit den beiden unabhängigen Koordinaten $\phi_i$ arbeiten können.

\begin{align*}
  x_1 =  l_1 \sin \phi_1 \\
  y_1 = -l_1 \cos \phi_1 \\
  x_2 =  l_1 \sin \phi_1 + l_2 \sin \phi_2 \\
  y_2 = -l_1 \cos \phi_1 - l_2 \cos \phi_2
\end{align*}

Mit den Zeitableitungen:

\begin{align*}
  \dot{x_1} = l_1 \dot{phi_1} \cos \phi_1 \\
  \dot{y_1} = l_1 \dot{phi_1} \sin \phi_1 \\
  \dot{x_2} = l_1 \dot{phi_1} \cos \phi_1 + l_2 \dot{phi_2} \cos \phi_2\\
  \dot{y_2} = l_1 \dot{phi_1} \sin \phi_1 + l_2 \dot{phi_2} \sin \phi_2
\end{align*}

Eingesetzt in $T_1$ (Kinetische Energe von $m_1$):

\begin{align*}
  T_1 &= \frac{m_1}{2} (l_1^2 \dot{\phi}_1^2 \cos^2 \phi_1 + l_1^2 \dot{\phi}_1^2 \sin^2 \phi_1) \\
    &= \frac{m_1}{2} (l_1^2 \dot{\phi}_1^2)
\end{align*}

Eingesetzt in $T_2$ (Kinetische Energe von $m_2$):

\begin{align*}
  T_2 &= \frac{m_2}{2} (l_1^2 \dot{\phi}_1^2 \cos^2 \phi_1 + l_2^2 \dot{\phi}_2^2 \cos^2 \phi_2 + 2l_1l_2\dot{\phi}_1\dot{\phi_2}\cos(\phi_1) \cos(\phi_2) \\
      &\quad + l_1^2 \dot{\phi}_1^2 \sin^2 \phi_1 + l_2^2 \dot{\phi_2}^2 \sin^2 \phi_2 + 2l_1l_2\dot{\phi_1}\dot{\phi_2}\sin(\phi_1)\sin(\phi_2)) \\
      &= \frac{m_2}{2} (l_1^2 \dot{\phi}_1^2 + l_2^2 \dot{\phi}_2^2 + 2l_1l_2\dot{\phi}_1\dot{\phi_2} (\cos(\phi_1) \cos(\phi_2) + \sin(\phi_1)\sin(\phi_2))) \\
      &= \frac{m_2}{2} (l_1^2 \dot{\phi}_1^2 + l_2^2 \dot{\phi}_2^2 + 2l_1l_2\dot{\phi}_1\dot{\phi_2} \cos(\phi_1 - \phi_2)) \\
\end{align*}

Insgesamt ($T = T_1 + T_2$):

\begin{align*}
  T &= \frac{m_1}{2} (l_1^2 \dot{\phi}_1^2) + \frac{m_2}{2} (l_1^2 \dot{\phi}_1^2 + l_2^2 \dot{\phi}_2^2) + 2l_1l_2\dot{\phi}_1\dot{\phi_2} \cos(\phi_1 - \phi_2) \\
    &= \frac{m_1 + m_2}{2} l_1^2 \dot{\phi}_1^2 + \frac{m_2}{2} l_2^2 \dot{\phi}_2^2 + m_2l_1l_2\dot{\phi}_1\dot{\phi_2} \cos(\phi_1 - \phi_2)
\end{align*}

Für die potentielle Energie gilt:
\begin{align*}
  V &= -m_1gl_1 \cos \phi_1 - m_2g(l_1 \cos \phi_1 + l_2 \cos \phi_2) \\
    &= -(m_1 + m_2)gl_1 \cos \phi_1 - m_2gl_2 \cos \phi_2
\end{align*}

Insgesamt:
\begin{align*}
  L = &\frac{m_1 + m_2}{2} l_1^2 \dot{\phi}_1^2 + \frac{m_2}{2} l_2^2 \dot{\phi}_2^2 + m_2l_1l_2\dot{\phi}_1\dot{\phi_2} \cos(\phi_1 - \phi_2) \\
      &+(m_1 + m_2)gl_1 \cos \phi_1 + m_2gl_2 \cos \phi_2
\end{align*}

\textbf{TODO:} Lagrange-Gleichung anwenden.

\end{document}
