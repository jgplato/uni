\documentclass[a4paper,german,12pt,smallheadings]{scrartcl}
\usepackage[T1]{fontenc}
\usepackage[utf8]{inputenc}
\usepackage{babel}
\usepackage{tikz}
\usepackage{geometry}
\usepackage{amsmath}
\usepackage{amssymb}
\usepackage{float}
\usepackage{cancel}
%\usepackage{wrapfig}
\usepackage[thinspace,thinqspace,squaren,textstyle]{SIunits}
\restylefloat{table}
\geometry{a4paper, top=15mm, left=20mm, right=40mm, bottom=20mm, headsep=10mm, footskip=12mm}
\linespread{1.5}
\setlength\parindent{0pt}
\begin{document}
\begin{center}
\bfseries % Fettdruck einschalten
\sffamily % Serifenlose Schrift
\vspace{-40pt}
Elektrodynamik und Optik, Sommersemester 2013, 8. Blatt \\
Markus Fenske, Tutor: Dr. Marko Wietstruk
\vspace{-10pt}
\end{center}
\section*{Aufgabe 1: Dynamo}

Aus der Vorlesung ist bekannt, dass

\begin{align*}
  U_{\text{ind}} = B \cdot N \cdot A \cdot \omega \sin \omega t
\end{align*}

Durch Einsetzen der Werte ergibt sich für die zeitabhängige Spannung $U(t)$:

\begin{align*}
  U(t) &= 0{,}1 \;\tesla \cdot 100 \cdot 0{,}01 \;\meter^2 \cdot 5 \;\second^{-1} \cdot \sin(5 \;\second^{-1} \cdot t) \\
       &= 0{,}5 \;\volt \cdot \sin 5 \;\second^{-1} \cdot t
\end{align*}

Da der Maximalwert des Sinus $1$ ist, ist die maximale Spannung $U_{\text{max}} = 0{,}5 \;\volt$.

\section*{Aufgabe 2: Magnetfeld im Koaxialkabel}
\subsection*{Teil a}

Vorüberlegung: Denken wir uns ein Kreisfläche, die senkrecht zu den Zylinderachsen durch das
Kabel verläuft, mit dem Mittelpunkt in der Zylinderachse. Wenn der Radius
dieser Kreisfläche kleiner as $r_1$ ist, geht kein Strom durch diese
Kreisfläche. Dementsprechend kann kein Magnetfeld im inneren des ersten
Zylinders vorhanden sein. Liegt der Radius zwischen $r_1$ und $r_2$, geht der
Strom des ersten Zylinders durch die Fläche, also existiert ein Magnetfeld. Ist
der Radius hingegen größer als $r_2$, geht der Strom beider Zylinder durch die
Fläche. Da der Strom durch einen Zylinder hin- und durch den anderen zurück
fließt ist die Summe der Ströme null, also existiert außerhalb des Kabels kein
Magnetfeld.

Aus dem Ampereschen Gesetz ergibt sich für das Feld zwischen den Zylindern dann

\begin{align*}
  \oint \vec{B} \cdot \vec{ds} = \mu_0 I
\end{align*}

Da das Kabel radialsymmetrisch ist, gilt das auch für das Feld. Dementsprechend
ist $B$ nur abhängig von $r$, senkrecht zu $\vec{ds}$ kann also vor das
Integral gezogen werden. Übrig bleibt der Betrag von $\vec{ds}$.

\begin{align*}
  B \oint ds = \mu_0 I
\end{align*}

Der Umfang des Kreises ist $2 \pi r$. Also:

\begin{align*}
  &B 2 \pi r = \mu_0 I \\
  \Leftrightarrow\quad &B(r) = \frac{\mu_0 I}{2 \pi r} \quad \text{wenn}\quad\; r_1 < r < r_2
\end{align*}


\end{document}
