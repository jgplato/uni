\documentclass[a4paper,german,12pt,smallheadings]{scrartcl}
\usepackage[T1]{fontenc}
\usepackage[utf8]{inputenc}
\usepackage{babel}
\usepackage{tikz}
\usepackage{geometry}
\usepackage{amsmath}
\usepackage{amssymb}
\usepackage{float}
\usepackage{cancel}
%\usepackage{wrapfig}
\usepackage[thinspace,thinqspace,squaren,textstyle]{SIunits}
\restylefloat{table}
\geometry{a4paper, top=15mm, left=20mm, right=40mm, bottom=20mm, headsep=10mm, footskip=12mm}
\linespread{1.5}
\setlength\parindent{0pt}
\begin{document}
\begin{center}
\bfseries % Fettdruck einschalten
\sffamily % Serifenlose Schrift
\vspace{-40pt}
Elektrodynamik und Optik, Sommersemester 2013, 8. Blatt \\
Markus Fenske, Tutor: Dr. Marko Wietstruk
\vspace{-10pt}
\end{center}
\section*{Aufgabe 1: Dynamo}

Aus der Vorlesung ist bekannt, dass

\begin{align*}
  U_{\text{ind}} = B \cdot N \cdot A \cdot \omega \sin \omega t
\end{align*}

Durch Einsetzen der Werte ergibt sich für die zeitabhängige Spannung $U(t)$:

\begin{align*}
  U(t) &= 0{,}1 \;\tesla \cdot 100 \cdot 0{,}01 \;\meter^2 \cdot 5 \;\second^{-1} \cdot \sin(5 \;\second^{-1} \cdot t) \\
       &= 0{,}5 \;\volt \cdot \sin 5 \;\second^{-1} \cdot t
\end{align*}

Da der Maximalwert des Sinus $1$ ist, ist die maximale Spannung $U_{\text{max}} = 0{,}5 \;\volt$.

\end{document}
