\documentclass[a4paper,german,12pt,smallheadings]{scrartcl}
\usepackage[T1]{fontenc}
\usepackage[utf8]{inputenc}
\usepackage{babel}
\usepackage{tikz}
\usepackage{geometry}
\usepackage{amsmath}
\usepackage{amssymb}
\usepackage{float}
\usepackage{wrapfig}
\usepackage[thinspace,thinqspace,squaren,textstyle]{SIunits}
\restylefloat{table}
\geometry{a4paper, top=15mm, left=20mm, right=40mm, bottom=20mm, headsep=10mm, footskip=12mm}
\linespread{1.5}
\setlength\parindent{0pt}
\begin{document}
\begin{center}
\bfseries % Fettdruck einschalten
\sffamily % Serifenlose Schrift
\vspace{-40pt}
Mathematik für Physiker I, Wintersemester 2012/2013, 8. Übungsblatt

Florian Neumeyer und Markus Fenske, Tutor: Stephan Schwartz
\vspace{-10pt}
\end{center}


\section*{Aufgabe 8.1}
\subsection*{Teil a+b}

Wir bestimmen den Koordinatenvektor $K_B(x_1)$ indem wir die folgende Gleichung, die dieser erfüllen müsste, lösen.

\begin{align*}
  \begin{pmatrix}
     1 & 4 & 3 \\
     2 & 2 & 0 \\
     3 & 1 & 0 \\
     4 & 3 & 2 \\
  \end{pmatrix}
  \cdot
  \begin{pmatrix}
    x \\
    y \\
    z
  \end{pmatrix}
  &=
  \begin{pmatrix}
    2 \\
    4 \\
    4 \\
    5
  \end{pmatrix} \\
  \begin{pmatrix}
     x + 4y + 3z \\
     2x + 2y \\
     3x + y \\
     4x + 3y + 2z \\
  \end{pmatrix}
  &=
  \begin{pmatrix}
    2 \\
    4 \\
    4 \\
    5
  \end{pmatrix}
\end{align*}

Das \textsc{Gauß}sche Verfahren wird hier nicht benötigt. Durch scharfes
Ansehen der zweiten und dritten Spalte erkannt man sofort $x = y = 1$, Somit
ergibt sich zusammen mit der oberen Spalte aus $1 + 4 + 3z = 2$, dass $z = -1$.
Dies erfüllt auch die untere Zeile. Somit $x_1 \in U$ und

\begin{equation*}
  K_B(x_1) = \begin{pmatrix} 1 \\ 1 \\ -1 \end{pmatrix}
\end{equation*}

Daraus folgt auch, dass $b_1$, $b_2$ und $b_3$ linear unabhängig sind, denn
sonst gäbe es keine eindeutige Lösung, und somit eine Basis von $U$.

Für $K_B(x_2)$ ergibt sich analog zu oben

\begin{align*}
  \begin{pmatrix}
     1 & 4 & 3 \\
     2 & 2 & 0 \\
     3 & 1 & 0 \\
     4 & 3 & 2 \\
  \end{pmatrix}
  \cdot
  \begin{pmatrix}
    x \\
    y \\
    z
  \end{pmatrix}
  &=
  \begin{pmatrix}
    1 \\
    2 \\
    3 \\
    -1
  \end{pmatrix} \\
  \begin{pmatrix}
     x + 4y + 3z \\
     2x + 2y \\
     3x + y \\
     4x + 3y + 2z \\
  \end{pmatrix}
  &=
  \begin{pmatrix}
    1 \\
    2 \\
    3 \\
    -1
  \end{pmatrix}
\end{align*}

Aus der zweiten und dritten Zeile folgt direkt $x = 1$ und $y = 0$. Eingesetzt
in die erste Zeile $1 + 3z = 1$, somit $z = 0$, was dann der letzten Zeile
widerspricht ($4 + 0 + 0 \neq -1$). Somit ist $x_2 \neq U$, denn es lässt sich
nicht durch die Basisvektoren darstellen.

\subsection*{Teil c}

\begin{align*}
  \Phi_B(z) &= -1 \cdot \begin{pmatrix} 1 \\ 2 \\ 3 \\ 4 \end{pmatrix} + 2 \cdot \begin{pmatrix} 4 \\ 2 \\ 1 \\ 3 \end{pmatrix} - 2 \cdot \begin{pmatrix} 3 \\ 0 \\ 0 \\ 2\end{pmatrix} \\
            &= \begin{pmatrix} -1 +8 -6 \\ -2 +4 \\ -3+2 \\ -4+6-4 \end{pmatrix} = \begin{pmatrix} 1 \\ 2 \\ -1 \\ -2\end{pmatrix}
\end{align*}

\section*{Aufgabe 8.2}
Aufgrund der an den $\mathbf{R}^5$ gestellten 2 Bedingungen, muss das Resultat
$U$ nur noch $5-2 = 3$ Dimensionen haben. Es werden also zusätzlich zum
gegebenen Basisvektor noch zwei weitere benötigt. Die Basisvektoren $\vec{a}_2,
\vec{a}_3$ müssen $\in U$ sein, somit als die Bedingung $A\vec{a}_n = 0$
erfüllen, und zum gegebenen $\vec{a}$ und untereinander linear unabhängig sein.

Wir suchen also zuerst zwei $\vec{a}_n$ mit

\begin{equation*}
  \begin{pmatrix} 1 & 1 & 1 & 1 & 1 \\ 1 & -1 & 1 & -1 & 0 \end{pmatrix} \cdot \vec{a}_n = 0
\end{equation*}

Es bieten sich als Lösungen offensichtlich an: $\vec{m}_1 = (0, 0, 1, 1, -2)^T$, $\vec{m}_2 = (1, 1, 0, 0, -2)^T$
oder $\vec{m_3} = (0, 1, 1, 0, -2)^T$. Da sich die zweite Möglichkeit
$\vec{m}_2$ bereits durch $\vec{m}_2 = \vec{a} + \vec{m}_3$ darstellen lässen,
wäre dies keine gute Wahl. Wir testen also die lineare Unabhängigkeit von
$\vec{m}_1$, $\vec{m}_3$ und $\vec{a}$.

\begin{equation*}
  \lambda_1 \begin{pmatrix} 1 \\ 1 \\ -1 \\ -1 \\ 0\end{pmatrix} +
  \lambda_2 \begin{pmatrix} 0 \\ 0 \\ 1 \\ 1 \\ -2\end{pmatrix} +
  \lambda_3 \begin{pmatrix} 0 \\ 1 \\ 1 \\ 0 \\ -2 \end{pmatrix} = 0
\end{equation*}

Aus der ersten Zeile sieht man bereits, dass die Gleichung nur für $\lambda_1 =
0$ erfüllt ist. Wenn man dies in die zweite Zeile einsetzt, erhält man
$\lambda_3 = 0$. Durch Einsetzen der Werte in die dritte Zeile dann auch
$\lambda_2 = 0$. Somit gibt es keine außer den trivialen Lösungen. Die Vektoren
sind linear unabhänig und somit die Basis von $U$.

\section*{Aufgabe 8.3}

Gegeben ist, dass $A \cup C$ eine Basis von $U$ ist. Dann lässt sich jedes
Element $u \in U$ als Linearkombination der Elemente von $a_1, a_2, \dotsb, a_n
\in A$ und $c_1, c_2, \dotsb, c_m \in C$ schreiben.

\begin{equation}
  u = \lambda_1a_1 + \lambda_2a_2 + \dotsb + \lambda_na_n + \lambda_{n+1}c_1 + \lambda_{n+2}c_2 + \lambda_{n+m} c_m
  \label{kombi_ac}
\end{equation}

Gegegeben ist außerdem, dass $B \cup C$ eine Basis von $W$ ist. Dann lässt sich
analog jedes $w \in W$ als Linearkombination der Elemente von $b_1, b_2,
\dotsb, b_n \in B$ und $c_1, c_2, \dotsb, c_m \in C$ schreiben.

\begin{equation}
  w = \mu_1b_1 + \mu_2b_2 + \dotsb + \mu_nb_n + \mu_{n+1}c_1 + \mu_{n+2}c_2 + \mu_{n+m} c_m
  \label{kombi_bc}
\end{equation}

$U + W$ bezeichnet den Vektorraum, dessen Elemente Linearkombinationen der
Elemente von $U$ und $W$ sind. Nennen wir diesen Vektorraum $X = U + W$. Wenn
$x \in X$, dann lässt sich jedes $x$ schreiben durch $u \in U, w \in W$

\begin{equation*}
  x = u + w
\end{equation*}

Dies können wir unter Benutzung von \ref{kombi_ac} und \ref{kombi_bc} auch
anders schreiben.

\begin{align*}
  x = &\lambda_1a_1 + \lambda_2a_2 + \dotsb + \lambda_na_n + \lambda_{n+1}c_1 +
       \lambda_{n+2}c_2 + \lambda_{n+m} c_m + \\
      &\mu_1b_1 + \mu_2b_2 + \dotsb + \mu_nb_n + \mu_{n+1}c_1 
       + \mu_{n+2}c_2 + \mu_{n+m} c_m
\end{align*}

Wir definieren nun $\nu_n = \lambda_{m+n} + \mu_{m+n}$, womit wir die Konstanten ersetzen können.

\begin{align*}
  x = &\lambda_1a_1 + \lambda_2a_2 + \dotsb + \lambda_na_n + \\
      &\nu_1c_1     + \nu_2    c_2 + \dotsb + \nu_nc_n     + \\
      &\mu_1b_1     + \mu_2b_2     + \dotsb + \mu_nb_n
\end{align*}

Wenn nun $A \cup B \cup C$ eine Basis ist, die den Vektorraum $Y$ aufspannt,
dann lässt sich jedes $y \in Y$ schreiben durch eine Linearkombination der der
Elemente $a_1, a_2, \dotsb, a_n \in A$, $b_1, b_2, \dotsb, b_n \in B$ und $c_1,
c_2, \dotsb, c_m \in C$  Also

\begin{align*}
  y = &\lambda_1a_1 + \lambda_2a_2 + \dotsb + \lambda_na_n + \\
      &\nu_1c_1     + \nu_2    c_2 + \dotsb + \nu_nc_n     + \\
      &\mu_1b_1     + \mu_2b_2     + \dotsb + \mu_nb_n
\end{align*}

Dies ist genau vorherige Gleichung. Somit folgt, dass $A \cup B \cup C$ eine Basis von $U + W$ ist.

Draus folgt auch $\dim U + \dim W = \dim(U \cup W) + \dim(U+W)$. Die Anzahl der
Basisvektoren eines Vektorraum ist bekanntlich gleich seiner Dimension. Da $A
\cup C$ eine Basis von $U$ ist, lässt sich jedes Element von $U$ ausdrücken
durch eine Linearkombination von $|A| + |C|$ Basisvektoren.

\begin{equation}
  \dim U = |A| + |C|
  \label{dim_u}
\end{equation}

Analog lässt sich jedes Element in $W$ ausdrücken durch eine Linearkombination von $|B| + |C|$ Basisvektoren.

\begin{equation}
  \dim W = |B| + |C|
  \label{dim_w}
\end{equation}

Aus der Aufgabenstellung gegeben ist, dass $C$ eine Basis von $U \cap W$ ist, also

\begin{equation}
  \dim(U \cap W) = |C|
  \label{dim_ucapw}
\end{equation}

Bewiesen wurde auch, dass $A \cup B \cup C$ eine Basis von $U + W$ ist, somit

\begin{equation}
  \dim(U + W) = |A| + |B| + |C|
  \label{dim_ucupw}
\end{equation}

Daraus folgt für die gegebene Gleichung

\begin{align*}
\dim U + \dim W &= \dim(U \cup W) + \dim(U+W) \\
|A| + |C| + |B| + |C| &= |C| + |A| + |B| + |C|
\end{align*}

Was damit bewiesen wäre.

\section{Aufgabe 8.4}

Gegeben ist, dass alle $\vec{a}_1, \dotsb, \vec{a}_n$ linear unabhängig sind.
Das bedeutet also, dass für folgende Gleichung nur die Lösung, dass alle $\lambda =
0$ existiert.

\begin{equation*}
  \lambda_1\vec{a}_1 + \lambda_2{a}_2 + \dotsb + \lambda_n\vec{a}_n = 0
\end{equation*}

Außerdem ist ein $\vec{b}$ gegeben, der ebenso wie alle $\vec{a}$ reell ist.
Das heisst, dass $i \vec{b}$ auf jeden Fall imaginär ist und damit linear
unabhängig zu allen $\vec{a}$.

Angenommen die Vektoren $\vec{a}_1 + i\vec{b}, \dotsb, \vec{a}_n + i\vec{b}$
wären linear unabhängig, dann müsste für folgende Gleichung nur die Lösung
existieren, dass alle $\lambda = 0$

\begin{equation*}
\lambda_1(\vec{a}_1 + i\vec{b}) + \dotsb + \lambda_n(\vec{a}_n + i\vec{b}) = 0
\end{equation*}

Durch Umformung erhält man:

\begin{align*}
  \lambda_1(\vec{a}_1 + i\vec{b}) + \dotsb + \lambda_n(\vec{a}_n + i\vec{b}) &= 0 \\
  \lambda_1\vec{a}_1 + \dotsb + \lambda_n\vec{a}_n + (\lambda_1 + \dotsb + \lambda_n) ni\vec{b} &= 0 \\
  \lambda_1\vec{a}_1 + \dotsb + \lambda_n\vec{a}_n &= -n(\lambda_1 + \dotsb + \lambda_n)i\vec{b}  \\
\end{align*}

Da auf der linken Seite eine reeller Vektor steht und auf der rechten Seite ein
imaginärer Vektor (denn es ist $\vec{b} \in \mathbb{R}^n$ gegeben), folgt daraus, dass
alle $\lambda = 0$. Somit sind alle $\vec{a}_1 + i\vec{b}, \dotsb, \vec{a}_n + i\vec{b}$
linear unabhängig, sofern alle $\lambda \in \mathbb{R}$ 

In der Aufgabestellung ist jedoch lineare Unabhängigkeit im Sinne von $\mathbb{C}^n$
gefordert (sofern ich das richtig verstanden habe), also $\lambda \in \mathbb{C}$.

Durch Aufsplitten in Real- und Imaginärteil ($\lambda = \mu + i\nu$) erhält man:

\begin{align*}
  \mu_1\vec{a}_{\mathbb{R},1} + \dotsb + \mu_n\vec{a}_{\mathbb{R},n} &= n(\nu_1 + \dotsb + \nu_n)\vec{b}_{\mathbb{R}}  \\
  \nu_1\vec{a}_{\mathbb{C},1} + \dotsb + \nu_n\vec{a}_{\mathbb{C},n} &= -n(\mu_1 + \dotsb + \mu_n)\vec{b}_{\mathbb{C}}
\end{align*}

Die untere Gleichung ist wertlos, denn da die Vektoren alle reell sind, ist der
Komplexteil $= 0$, somit lassen sich keine Aussagen herleiten.

Es bleibt übrig:
\begin{align*}
  \mu_1\vec{a}_{1} + \dotsb + \mu_n\vec{a}_{n} &= n(\nu_1 + \dotsb + \nu_n)\vec{b}
\end{align*}

Diese Gleichung lässt sich erfüllen, sofern $\vec{b}$ eine Linearkombination
von $\vec{a}_1, \dotsb, \vec{a}_n$ ist.

Wo liegt mein Fehler?

\end{document}
