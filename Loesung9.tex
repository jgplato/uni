\documentclass[a4paper,german,12pt,smallheadings]{scrartcl}
\usepackage[T1]{fontenc}
\usepackage[utf8]{inputenc}
\usepackage{babel}
\usepackage{tikz}
\usepackage{geometry}
\usepackage{amsmath}
\usepackage{amssymb}
\usepackage{float}
\usepackage{wrapfig}
\usepackage[thinspace,thinqspace,squaren,textstyle]{SIunits}
\restylefloat{table}
\geometry{a4paper, top=15mm, left=20mm, right=40mm, bottom=20mm, headsep=10mm, footskip=12mm}
\linespread{1.5}
\setlength\parindent{0pt}
\begin{document}
\begin{center}
\bfseries % Fettdruck einschalten
\sffamily % Serifenlose Schrift
\vspace{-40pt}
Mathematik für Physiker I, Wintersemester 2012/2013, 9. Übungsblatt

Florian Neumeyer und Markus Fenske, Tutor: Stephan Schwartz
\vspace{-10pt}
\end{center}


\section*{Aufgabe 9.1}
\subsection*{Teil a}

Sei $v \in V$, dann gilt aufgrund der Idempotenz des Endomorphismus $P$
\begin{align*}
  P^2(v) &= P(v) \\
  P^2(v) - P(v) &= 0
\end{align*}

Da $P$ ein Vektorraum-Endomorphismus, damit ein Vektorraum-Holomorphismus und
damit eine lineare Abbildung ist, folgt aus muss aus der Homogenität linearer
Abbildungen ($P(x) + P(y) = P(x+y), x{,}y \in V$) folgen, dass

\begin{align*}
  P(P(v) - v) &= 0
\end{align*}

Der Kern von $P$ ist definiert als $\ker P := \{v \in V: P(v) = 0\}$, somit
$P(v) - v \in \ker P$ sein.

Das Bild von $P$ ist per definitionem $\operatorname{Im} P := \{v \in V: P(w) = v, w \in V\}$.
Sei also $u \in V$, dann lässt es sich darstellen durch

\begin{align*}
  u &= u \\
  u &= u - P(u) + P(u) \\
  u &= x + y, x \in \ker P, y \in \operatorname{Im} P
\end{align*}

Womit bewiesen wäre, dass $V = \ker P + \operatorname{Im} P$. Allerdings ist
dies nur der Beweis für die Summe, nicht für die direkte Summe. Für die direkte
Summe ist zu zeigen (\textbf{Begründung einfügen, warum das zu zeigen ist}),
dass $\ker P \cap \operatorname{Im} P = \{0\}$.

Gegegeben sei $x \in \ker P$ und gleichzeitig $x \operatorname{Im} P$. Zu
zeigen ist, dass dann $x = 0$ gilt.

Aus $x \in \operatorname{Im} P$ folgt, dass $x$ darstellbar ist durch $x =
P(z), z \in V$.

\begin{align*}
  x &= P(z)
\end{align*}

Da $P$ eine lineare Abbildung ist, müssen bei Werten auch gleiche Bilder herrauskommen. Somit

\begin{align*}
  P(x) &= P(P(z))
\end{align*}

Da $P$ idempotent ($P^2 = P$) ist:

\begin{align*}
  P(x) &= P(z)
\end{align*}

Aus $x \in \ker P$ folgt muss $P(x) = 0$:
\begin{align*}
  0 &= P(z)
\end{align*}

Am Anfang war allerdings bereits gegeben, dass $x = P(z)$:
\begin{align*}
  0 &= x
\end{align*}

Somit $\ker P \cap \operatorname{Im} P = \{0\}$, und zusammen mit $V = \ker P + \operatorname{Im} P$ ergibt das:

\begin{equation}
  V = \ker P \oplus \operatorname{Im} P
\end{equation}

Und jetzt darf ich so ein Kästchen machen. $\Box$

\subsection*{Teil b}

Es ist gegeben, dass $P$ nicht alles in den Nullpunkt schickendarf ($O$), noch
untätig sein darf ($I$). Eine Projektion auf eine Achse ist eine Möglichkeit.

\begin{equation} 
  P\left(\begin{pmatrix} \alpha \\ \beta \end{pmatrix}\right) =
    \begin{pmatrix}\alpha \\ 0\end{pmatrix}
\end{equation}



\end{document}
