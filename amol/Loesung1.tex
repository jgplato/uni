\documentclass[a4paper,german,12pt,smallheadings]{scrartcl}
\usepackage[T1]{fontenc}
\usepackage[utf8]{inputenc}
\usepackage{babel}
\usepackage{geometry}
\usepackage[fleqn]{amsmath}
\usepackage{amssymb}
\usepackage{float}
\usepackage{enumerate}
\usepackage{commath} % http://tex.stackexchange.com/questions/14821/whats-the-proper-way-to-typeset-a-differential-operator
\usepackage{cancel}

% Number only referenced equations
\usepackage[fleqn]{mathtools}
\mathtoolsset{showonlyrefs}

%\usepackage{wrapfig}
\usepackage[thinspace,thinqspace,squaren,textstyle]{SIunits}

% New command for color underlining
\usepackage{xcolor}

\newsavebox\MBox
\newcommand\colul[2][red]{{\sbox\MBox{$#2$}%
  \rlap{\usebox\MBox}\color{#1}\rule[-1.2\dp\MBox]{\wd\MBox}{0.5pt}}}

\restylefloat{table}
\geometry{a4paper, top=15mm, left=10mm, right=20mm, bottom=20mm, headsep=10mm, footskip=12mm}
\linespread{1.5}
\setlength\parindent{0pt}
\DeclareMathOperator{\Tr}{Tr}
\DeclareMathOperator{\Var}{Var}
\begin{document}
\allowdisplaybreaks % Seitenumbrüche in Formeln erlauben
\begin{center}
\bfseries % Fettdruck einschalten
\sffamily % Serifenlose Schrift
\vspace{-40pt}
Atom- und Molekülphysik, Sommersemester 2014, Aufgabenblatt 1

Markus Fenske, Luis Herrmann, Tutor: Michael Kleinert
\vspace{-10pt}
\end{center}
\section*{Aufgabe 1: Potentialtopf I}

\textbf{FIXME:} Kurze Herleitung Potentialtopf (Wellenfunktion mit Normierung, Energie)

\begin{enumerate}[a)]
  \item
    Die Übergangsenergie ist $E_2 - E_1 = \frac{3 \hbar^2 \pi^2}{2 m_e L^2}$.
    Für $L = 1 \; \nano\meter$ ergibt sich eine Energie von $752 \;
    \milli\electronvolt$, für $L = 1 \; \centi\meter$ entsprechend $7{,}52 \cdot
    10^{-15} \;\electronvolt$.
  \item
    Wir nehmen an, dass mit ``mittlere Hälfte'' das Intervall $[\frac{1}{4} L,
    \frac{3}{4} L]$ gemeint ist. Die Aufenthaltswahrscheinlichkeit ist dann
    \begin{align}
        &\int\limits_{\frac{1}{4} L}^{\frac{3}{4} L} \dif x\; \envert{\psi(x)}^2
      =  \frac{2}{L} \int\limits_{\frac{1}{4} L}^{\frac{3}{4} L} \dif x\; \sin^2 \frac{\pi x}{L}
      =  \frac{1}{L} \int\limits_{\frac{1}{4} L}^{\frac{3}{4} L} \dif x\; 1 - \cos \frac{2 \pi x}{L}
      =  \frac{1}{L} \del{ \frac{L}{2} - \sbr{\frac{L}{2 \pi} \sin \frac{2 \pi x}{L}}_{\frac{L}{4}}^{\frac{3L}{4}} } \\
      = \;&\frac{1}{L} \del{\frac{L}{2} + \frac{L}{\pi}}
      = \frac{\pi + 2}{2 \pi}
      \approx 81{,}83 \; \%
    \end{aligned}
\end{enumerate}
\end{document}
