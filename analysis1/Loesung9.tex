\documentclass[a4paper,german,12pt,smallheadings]{scrartcl}
\usepackage[T1]{fontenc}
\usepackage[utf8]{inputenc}
\usepackage{babel}
\usepackage{tikz}
\usepackage{geometry}
\usepackage{amsmath}
\usepackage{amssymb}
\usepackage{float}
%\usepackage{wrapfig}
\usepackage{pdflscape}
\pagenumbering{gobble}
\usepackage[thinspace,thinqspace,squaren,textstyle]{SIunits}
\restylefloat{table}
\geometry{a4paper, top=15mm, left=20mm, right=40mm, bottom=20mm, headsep=10mm, footskip=12mm}
\linespread{1.5}
\setlength\parindent{0pt}
\begin{document}
\begin{center}
\bfseries % Fettdruck einschalten
\sffamily % Serifenlose Schrift
\vspace{-40pt}
Analysis I, Sommersemester 2013, 9. Übungsblatt \\
Luis Herrmann und Markus Fenske, Tutor: Adam Schienle
\vspace{-10pt}
\end{center}

\section*{Aufgabe 9.1}

Wir benutzen die Stetigkeit der Exponential- und der Logarithmusfunktion (denn
dann dürfen wir den Limes in die Funktion ziehen), außerdem, dass $\exp \log a
= a$ und die Regel von L'Hospital. Und natürlich, dass $\lim_{x \to \infty}
f(x) = \lim_{t \searrow 0} f\left(\frac{1}{t}\right)$.

\begin{align*}
  \lim_{x \to \infty} \left(2-a^\frac{1}{x}\right)^x &= \exp \lim_{x \to \infty} x \cdot \log \left(2-a^\frac{1}{x}\right) \\
  &= \exp \lim_{t \searrow 0} \frac{\log (2-a^t)}{t} \\
  &= \exp \lim_{t \searrow 0} \frac{a^t \cdot \log a}{2-a^t} \\
  &= \exp \frac{a^0 \cdot \log a}{2-a^0} \\
  &= \exp \frac{1 \cdot \log a}{2-1} \\
  &= \exp (-\log a) \\
  &= \frac{1}{a}
\end{align*}


\end{document}
