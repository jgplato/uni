\documentclass[a4paper,german,12pt,smallheadings]{scrartcl}
\usepackage[T1]{fontenc}
\usepackage[utf8]{inputenc}
\usepackage{babel}
\usepackage{tikz}
\usepackage{geometry}
\usepackage{amsmath}
\usepackage{amssymb}
\usepackage{float}
\usepackage{enumerate}
%\usepackage{wrapfig}
\usepackage[thinspace,thinqspace,squaren,textstyle]{SIunits}
\restylefloat{table}
\geometry{a4paper, top=15mm, left=20mm, right=40mm, bottom=20mm, headsep=10mm, footskip=12mm}
\linespread{1.5}
\setlength\parindent{0pt}
\begin{document}
\begin{center}
\bfseries % Fettdruck einschalten
\sffamily % Serifenlose Schrift
\vspace{-40pt}
Analysis II, Wintersemester 2013/2013, 1. Übungsblatt

Markus Fenske, Tutor: undefined
\vspace{-10pt}
\end{center}

\section*{Aufgabe 1.1}

Wir folgen dem Hinweis aus der Aufgabenstellung und zeigen zuerst für $k=0$:

\begin{equation*}
  \int_0^\infty e^{-nx} \; dx = \left[ -\frac{1}{n} e^{-nx} \right]_0^\infty = \underbrace{\left(\lim_{x \to \infty} - \frac{1}{n} e^{-nx} \right)}_{=0} + \frac{e^0}{n} = \frac{1}{n}
\end{equation*}

Anschließend für $k > 0$:

\begin{equation*}
  \int_0^\infty x^k \cdot e^{-nx} \; dx = \text{?}
\end{equation*}

Wir benutzen partielle Integration (Kurzschreibweise)

\begin{equation*}
  \int_a^b f' \cdot g = \left[f \cdot g\right]_a^b - \int_a^b f \cdot g'
\end{equation*}

mit

\begin{align*}
  f'(x) &= e^{-nx}              & g(x) &= x^k \\
        &\Downarrow             &      &\Downarrow\\
  f(x)  &= -\frac{1}{n} e^{-nx} & g'(x) &= kx^{k-1}
\end{align*}

und erhalten:

\begin{align*}
  \int_0^\infty x^k \cdot e^{-nx} \; dx  &= \overbrace{\left[-\frac{1}{n} e^{-nx} \cdot x^k\right]_0^\infty}^{=0-0} - \int_0^\infty -\frac{1}{n}e^{-nx} \cdot kx^{k-1} \; dx \\
                                         &= \int_0^\infty \frac{1}{n}e^{-nx} \cdot kx^{k-1} \; dx \\
                                         &= \frac{k}{n} \int_0^\infty e^{-nx} \cdot x^{k-1} \; dx
\end{align*}

Daraus ergibt sich die Rekursionsformel:

\begin{equation*}
  \int_0^\infty x^k \cdot e^{-nx} \; dx = \frac{1}{n} \cdot \underbrace{\frac{k}{n} \cdot \frac{k-1}{n} \cdot \frac{k-2}{n} \cdot \dots \cdot \frac{k-k}{n}}_{k\text{ Faktoren}} = \frac{k!}{n^{k+1}}
\end{equation*}

Was zu beweisen war.

\section*{Aufgabe 1.2}

\begin{enumerate}[(1)]
\item Stimmt.
\item Es gilt
  \begin{equation*}
    \sum_{n=1}^\infty \frac{x}{e^{nx}} = x \cdot \left(\sum_{n=1}^\infty \left(e^{-x}\right)^n\right) = x \cdot \left(\left(\sum_{n=0}^\infty \left(e^{-x}\right)^n\right) - 1\right)
  \end{equation*}

  Wenn $x>0$ ist, gilt die Geometrische Reihe, weil dann $|e^{-x}| < 0$ ist.

  \begin{equation*}
    = x \cdot \left( \frac{1}{1-e^{-x}} - 1 \right) = x \frac{1 - (1-e^{-x})}{1-e^{-x}} = x \frac{e^{-x}}{e^{-x} (e^x - 1)} = \frac{x}{e^x-1}
  \end{equation*}

  \textbf{TODO:} Gleichmäßige Stetigkeit zeigen!

\item \textbf{TODO:} WTF? Bitte was?
\end{enumerate}

\section*{Aufgabe 1.3}

\begin{enumerate}[(1)]
\item Gegenbeispiel. Sei

\begin{equation*}
  f_n(x) = \begin{cases} 
    0           & \mbox{wenn} \; x = 0 \\
    \frac{1}{x} & \mbox{wenn} \; x \neq 0
  \end{cases}
\end{equation*}

Es ist klar, dass die Funktion konvergiert, denn sie enthält überhaupt kein
$n$. Trotzdem ist $f$ unbeschränkt, denn

\begin{equation*}
  \lim_{x \to 0} \frac{1}{x} = \infty
\end{equation*}

\item Gegenbeispiel. Siehe oben. Die Funktion konvergiert gleichmäßig,
  weil sie mit ihrer Grenzfunktion identisch ist. Trotzdem ist sie
  unbeschränkt.

\end{enumerate}

\section*{Aufgabe 1.4}

Die Funktionenfolge $f_n(x)$ konvergiert punktweise gegen $f(x) = 0$.

\begin{equation*}
  \lim_{n \to \infty} f_n(x) = \lim_{n \to \infty} \frac{x}{1+nx^2} = \lim_{n \to \infty} \frac{x}{n \left(\frac{1}{n} + x^2\right)} = \lim_{n \to \infty} \frac{1}{n} \cdot \lim_{n \to \infty} \frac{x}{\frac{1}{n} + x^2} = 0 \cdot x = 0
\end{equation*}

Un zu untersuchen, ob die Funktionenfolge gleichmäßig konvergiert, untersuche ich zuerst die Beschränktheit von $f_n(x)$. Dazu berechne ich die Ableitung per Quotientenregel

\begin{equation*}
  f'_n(x) = \frac{1 \cdot (1+nx^2) - x \cdot 2nx}{(1+nx^2)^2} = \frac{1-nx^2}{(1+nx^2)^2}
\end{equation*}

die Extremwerte finden sich bei $f'_n(x) \overset{!}{=} 0$. Da der Zähler für
alle $x$ im Definitionsberech positiv ist, muss also der Nenner verschwinden

\begin{equation*}
  1-nx^2 \overset{!}{=} 0
\end{equation*}

Dies ist gegeben bei

\begin{equation}
  x = \pm \sqrt{\frac{1}{n}}
\end{equation}

Maximum und Minimum finden sich also bei

\begin{equation*}
  f_n\left(\pm\sqrt{\frac{1}{n}}\right) = \frac{\pm\sqrt{\frac{1}{n}}}{1+n\frac{1}{n}} = \pm\frac{1}{2\sqrt{n}}
\end{equation*}

Nun zeige ich gleichmäßige Konvergenz. Dafür muss ich beweisen, dass sich zu
jedem $\epsilon$ ein $n$ finden lässt, derart, dass $\left| f(x) - f_n(x)
\right| < \epsilon$. Mit dem errechneten Extremwerten kann ich nun eine
Abschätzung machen.

\begin{equation*}
  \left| f(x) - f_n(x) \right| = \left|0-\frac{x}{1+nx^2}\right| < \frac{1}{2\sqrt{n}} < \epsilon
\end{equation*}

Da $\frac{1}{2\sqrt{n}}$ Nullfolge ist, ist gleichmäßige Konvergenz gegeben.

\end{document}
