\documentclass[a4paper,german,12pt,smallheadings]{scrartcl}
\usepackage[T1]{fontenc}
\usepackage[utf8]{inputenc}
\usepackage{babel}
\usepackage{tikz}
\usepackage{geometry}
\usepackage{amsmath}
\usepackage{amssymb}
\usepackage{float}
\usepackage{enumerate}
%\usepackage{wrapfig}
\usepackage[thinspace,thinqspace,squaren,textstyle]{SIunits}
\restylefloat{table}
\renewcommand{\thefootnote}{\fnsymbol{footnote}}
\geometry{a4paper, top=15mm, left=20mm, right=40mm, bottom=20mm, headsep=10mm, footskip=12mm}
\linespread{1.5}
\setlength\parindent{0pt}
\begin{document}
\begin{center}
\bfseries % Fettdruck einschalten
\sffamily % Serifenlose Schrift
\vspace{-40pt}
Analysis II, Wintersemester 2013/2013, 1. Übungsblatt

Markus Fenske\footnote[1]{In Zusammenarbeit mit: Luis Herrmann}, Tutor: Sebastian Bierke
\vspace{-10pt}
\end{center}

\section*{Aufgabe 1.1}

%\textbf{Voraberklärung:} Die lockeren Schreibweisen $\int_{a}^{\infty} \dots$ und
%$\left[\dots\right]_a^\infty$ sind hier und in allen folgenden Übungsblättern
%wie folgt definiert, in der Annahme, dass diese Grenzwerte existieren.
%\begin{equation*}
%  \int_{a}^{\infty} \dots \; := \lim\limits_{b \to \infty}\int_{a}^{b} \dots \qquad \qquad 
%  \text{und}
%  \qquad \qquad \left[\dots\right]_a^\infty := \lim\limits_{b \to \infty}\left[\dots\right]_a^b
%\end{equation*}
%
%\rule{\textwidth}{0.4pt}
%\vspace{4mm}


Wir folgen dem Hinweis aus der Aufgabenstellung und zeigen zuerst für $k=0$:

\begin{equation*}
  \int_0^\infty e^{-nx} \; dx = \left[ -\frac{1}{n} e^{-nx} \right]_0^\infty = \underbrace{\left(\lim_{x \to \infty} - \frac{1}{n} e^{-nx} \right)}_{=0} + \frac{e^0}{n} = \frac{1}{n}
\end{equation*}

Anschließend für $k > 0$:

\begin{equation*}
  \int_0^\infty x^k \cdot e^{-nx} \; dx = \text{?}
\end{equation*}

Wir benutzen partielle Integration (Kurzschreibweise)

\begin{equation*}
  \int_a^b f' \cdot g = \left[f \cdot g\right]_a^b - \int_a^b f \cdot g'
\end{equation*}

mit

\begin{align*}
  f'(x) &= e^{-nx} & g(x) &= x^k \\
        &\Downarrow & &\Downarrow\\
  f(x) &= -\frac{1}{n} e^{-nx} & g'(x) &= kx^{k-1}
\end{align*}

und erhalten:

\begin{align*}
  \int_0^\infty x^k \cdot e^{-nx} \; dx  &= \overbrace{\left[-\frac{1}{n} e^{-nx} \cdot x^k\right]_0^\infty}^{=0+0\footnotemark[2]} - \int_0^\infty -\frac{1}{n}e^{-nx} \cdot kx^{k-1} \; dx \\
                                         &= \int_0^\infty \frac{1}{n}e^{-nx} \cdot kx^{k-1} \; dx \\
                                         &= \frac{k}{n} \int_0^\infty e^{-nx} \cdot x^{k-1} \; dx
\end{align*}

\footnotetext[2]{
  Falls das nicht offensichtlich sein sollte, gilt das wegen
  \begin{equation*}
  \left[-\frac{1}{n} e^{-nx} \cdot x^k\right]_0^\infty = \lim\limits_{x \to \infty}\left(-\frac{1}{e^{nx}}\right) - \frac{0^k}{e^{n\cdot 0}}=0 + 0
  \end{equation*}
}

Nach $k$-facher Rekursion erhalten wir demnach:

\begin{equation*}
  \int_0^\infty x^k \cdot e^{-nx} \; dx = \underbrace{\frac{k}{n} \cdot \frac{k-1}{n} \cdot \frac{k-2}{n} \cdot \dots \cdot \frac{1}{n}}_{k\text{ Faktoren}} \cdot \frac{1}{n} = \frac{k!}{n^{k+1}}
\end{equation*}

Was zu beweisen war.


\section*{Aufgabe 1.2}

\begin{enumerate}[(1)]
\item Stimmt.
\item Es gilt
  \begin{equation*}
    \sum_{n=1}^\infty \frac{x}{e^{nx}} = x \cdot \left(\sum_{n=1}^\infty \left(e^{-x}\right)^n\right) = x \cdot \left(\left(\sum_{n=0}^\infty \left(e^{-x}\right)^n\right) - 1\right)
  \end{equation*}

  Dies enthält die Geometrische Reihe mit $q=e^{-x}$, sie konvergiert
  für $|q|<1$.  Da wir auf einem Interval $[a,b]$ mit $a > 0$ operieren und
  $e^{-x} < 1$ für $x>0$, ist $|q| < 1$, also konvergiert die Reihe.

  \begin{equation*}
      = x \cdot \left( \frac{1}{1-e^{-x}} - 1 \right) = x \frac{1 - (1-e^{-x})}{1-e^{-x}} = x \frac{e^{-x}}{e^{-x} (e^x - 1)} = \frac{x}{e^x-1}
  \end{equation*}

  Für gleichmäßige Konvergenz muss für alle $x \in [a,b], 0 < a < b < \infty$ gegeben sein:

  \begin{align*}
    & \quad \lim_{m \to \infty} \left| g_m(x) - g(x) \right| = 0 \\
    \Leftrightarrow & \quad \lim_{m \to \infty} \left|\left(\sum_{n=1}^{m}\frac{x}{e^{nx}}\right) - x \left(\frac{1}{1-e^{-x}} - 1\right)\right| = 0 \\
    \Leftrightarrow & \quad \lim_{m \to \infty} x \left|\left(\sum_{n=1}^{m} \left(e^{-x}\right)^n \right) - \left(\frac{1}{1-e^{-x}} - 1\right)\right| = 0 \\
    \Leftrightarrow & \quad \lim_{m \to \infty} x \left|\left(\frac{1-e^{-x(m+1)}}{1-e^{-x}} - 1 \right)- \left(\frac{1}{1-e^{-x}} - 1\right)\right| = 0 \\
    \Leftrightarrow & \quad \lim_{m \to \infty} x \left|\left(\frac{1-e^{-x(m+1)}}{1-e^{-x}} \right)- \left(\frac{1}{1-e^{-x}} \right)\right| = 0 \\
    \Leftrightarrow & \quad \lim_{m \to \infty} x \left|\left(\frac{-e^{-x(m+1)}}{1-e^{-x}} \right)\right| = 0 \\
  \end{align*}

  Da $x > 0$ bleibt der Nenner stets größer als $0$. Es muss also gelten

  \begin{align*}
    \lim_{n \to \infty} x e^{-xn} = 0 \\
  \end{align*}

  Was für $x > 0$ immer wahr ist. Somit konvergiert die Reihe gleichmäßig.

  \item 
  Besteht aus 2 Teilschritten.
  Der erste Teiilschritt 
  \begin{equation*}
    \sum_{n=1}^{\infty}\frac{1}{n^2}=\sum_{n=1}^{\infty}\int_{0}^{\infty}\frac{xdx}{e^{nx}}
  \end{equation*}
  ist einfaches Einsetzen von Teil (1).

  Der zweite Teilschritt

  \begin{equation*}
    \sum_{n=1}^{\infty}\int_{0}^{\infty}\frac{xdx}{e^{nx}} = \int_0^\infty \frac{x \; dx}{e^x-1}
  \end{equation*}

  ist die Anwendung des Satzes der majorisierten Konvergenz. Dieser besagt, dass

  \begin{equation*}
    \lim_{n \to \infty} \int_0^\infty f_n(x) \; dx = \int_0^\infty f(x) \; dx
  \end{equation*}

  erlaubt ist, wenn
  \begin{enumerate}[a)]
    \item Alle Partialsummenfunktionen $f_n$ stetig sind.
    \item Die Folge auf jedem kompakten Teilintervall von $[0,\infty[$ gleichmäßig gegen $f$ konvergiert.
    \item Eine Funktion $g$ existiert, für die gilt $|f_n(x)| \le g(x)$, $\int_0^\infty g(x)$ existiert
  \end{enumerate}

  Für $f_n(x) = \sum_{n=1}^\infty xe^{-nx}$ und $f(x) = x/(e^x - 1)$ gilt:
  \begin{enumerate}[a)]
    \item Alle $f_n$ sind als Verknüpfung stetiger Funktionen stetig.
    \item Haben wir in (2) bewiesen
    \item Es gilt $|f_n(x)| \le f(x)$ (haben wir in (2) auch bewiesen) und $\int_0^\infty f(x)$ existiert.
  \end{enumerate}

  Deswegen gilt
  \begin{equation*}
    \sum_{n=1}^{\infty}\int_{0}^{\infty}\frac{xdx}{e^{nx}}
    =\lim\limits_{m \to \infty}\left(\sum_{n=1}^{m}\int_{0}^{\infty}\frac{xdx}{e^{nx}}\right)
    =\lim\limits_{m \to \infty} \int_{0}^{\infty} \left(\sum_{n=1}^{m}\frac{x}{e^{nx}}\right) \; dx
    =\lim\limits_{m \to \infty} \int_{0}^{\infty} f_m(x) \; dx
  \end{equation*}

  Nach Satz der majorisierten Konvergenz (deren Anwendbarkeit wir gerade begründet haben) ist
  \begin{equation*}
    \lim\limits_{m \to \infty} \int_{0}^{\infty} f_m(x) \; dx = \int_0^\infty f(x) \; dx = \int_0^\infty \frac{x \; dx}{e^x-1}
  \end{equation*}

  Was zu zeigen war.
\end{enumerate}
\section*{Aufgabe 1.3}
\begin{enumerate}[(1)]
\item Gegenbeispiel. Sei

\begin{equation*}
  f_n(x) = \begin{cases} 
    0           & \mbox{wenn} \; x = 0 \\
    \frac{1}{x + \frac{1}{n}} & \mbox{wenn} \; x \neq 0
  \end{cases}
\end{equation*}


Man sieht die Grenzfunktion sofort als

\begin{equation*}
  f(x) = \begin{cases} 
    0           & \mbox{wenn} \; x = 0 \\
    \frac{1}{x} & \mbox{wenn} \; x \neq 0
  \end{cases}
\end{equation*}

Die Grenzfunktion $f$ ist ganz klar unbeschränkt ($\lim_{x \to 0} f(x) = \infty$), während alle $f_n(x)$ beschränkt sind ($\lim_{x \to 0} f_n = n$).

\item
  Beweis: Wenn alle $f_n$ für alle $x_0 \in [0,1]$ beschränkt sind, bedeutet dies, dass

  \begin{equation*}
    \lim_{x \to x_0} |f_n(x)| \neq \infty
  \end{equation*}

  Wenn $f_n$ gleichmäßig gegen $f$ konvergiert, heißt dies, dass

  \begin{align*}
    &\quad \lim_{x \to x_0} \lim_{n \to \infty} |f_n(x) - f(x)| = 0 \\
    \Leftrightarrow&\quad \lim_{x \to x_0} \lim_{n \to \infty} |f_n(x)| = \lim_{x \to x_0} |f(x)| \\
  \end{align*}

  Daraus folgt
  \begin{equation*}
    \lim_{x \to x_0} |f(x)| \neq \infty
  \end{equation*}

  Also ist $f$ im Definitionsbereich beschränkt.
\end{enumerate}
\section*{Aufgabe 1.4}

Die Funktionenfolge $f_n(x)$ konvergiert punktweise gegen $f(x) = 0$.

\begin{equation*}
  \lim_{n \to \infty} f_n(x) = \lim_{n \to \infty} \frac{x}{1+nx^2} = \lim_{n \to \infty} \frac{x}{n \left(\frac{1}{n} + x^2\right)} = \lim_{n \to \infty} \frac{1}{n} \cdot \lim_{n \to \infty} \frac{x}{\frac{1}{n} + x^2} = 0 \cdot x = 0
\end{equation*}

Un zu untersuchen, ob die Funktionenfolge gleichmäßig konvergiert, untersuchen wir zuerst die Beschränktheit von $f_n(x)$. Dazu berechnen wir die Ableitung per Quotientenregel

\begin{equation*}
  f'_n(x) = \frac{1 \cdot (1+nx^2) - x \cdot 2nx}{(1+nx^2)^2} = \frac{1-nx^2}{(1+nx^2)^2}
\end{equation*}

die Extremwerte finden sich bei $f'_n(x) \overset{!}{=} 0$. Da der Zähler für
alle $x$ im Definitionsberech positiv ist, muss also der Nenner verschwinden

\begin{equation*}
  1-nx^2 \overset{!}{=} 0
\end{equation*}

Dies ist gegeben bei

\begin{equation}
  x = \pm \sqrt{\frac{1}{n}}
\end{equation}

Maximum und Minimum finden sich also bei

\begin{equation*}
  f_n\left(\pm\sqrt{\frac{1}{n}}\right) = \frac{\pm\sqrt{\frac{1}{n}}}{1+n\frac{1}{n}} = \pm\frac{1}{2\sqrt{n}}
\end{equation*}

Nun zeigen wir gleichmäßige Konvergenz. Dafür müssen wir beweisen, dass sich zu
jedem $\epsilon$ ein $n$ finden lässt, derart, dass $\left| f(x) - f_n(x)
\right| < \epsilon$. Mit dem errechneten Extremwerten können wir nun eine
Abschätzung machen.

\begin{equation*}
  \left| f(x) - f_n(x) \right| = \left|0-\frac{x}{1+nx^2}\right| < \frac{1}{2\sqrt{n}} < \epsilon
\end{equation*}

Da $\frac{1}{2\sqrt{n}}$ Nullfolge ist, ist gleichmäßige Konvergenz gegeben.

Für die Ableitung der Grenzfunktion gilt $f'(x) = 0$. Die Ableitung von $f_n(x)$ ist:

\begin{equation*}
  f_n'(x) = \frac{x(2nx) + (1+nx^2)}{(1+nx^2)^2} = \frac{1+ 3nx^2}{1+2nx^2 + n^2x^4}
\end{equation*}

Fallunterscheidung.

\begin{enumerate}[a)]
  \item
    Wenn $x=0$ ist $f_n'(x) = 1$, unabhängig von $n$

  \item
    Wenn $x\neq0$ ist $\lim_{n \to \infty} f_n(x) = 0$. Man sieht das (ohne
    $n$ bzw. $n^2$ explizit auszuklammern und zu kürzen) an der höheren Potenz
    von $n$ um Nenner-Polynom.

    Aufgrund der höheren $x$-Potenz im Nenner-Polynom gilt dies auch für
    Grenzwertbetrachtungen von $x$, ohne diese explizit durchführen zu wollen.
\end{enumerate}

$f_n'$ konvergiert also nur für alle $x \neq 0$ gegen $f'$
\end{document}
