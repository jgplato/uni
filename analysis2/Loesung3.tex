\documentclass[a4paper,german,12pt,smallheadings]{scrartcl}
\usepackage[T1]{fontenc}
\usepackage[utf8]{inputenc}
\usepackage{babel}
\usepackage{tikz}
\usepackage{geometry}
\usepackage{amsmath}
\usepackage{amssymb}
\usepackage{float}
\usepackage{enumerate}
%\usepackage{wrapfig}
\usepackage[thinspace,thinqspace,squaren,textstyle]{SIunits}
\restylefloat{table}
\renewcommand{\thefootnote}{\fnsymbol{footnote}}
\geometry{a4paper, top=15mm, left=20mm, right=40mm, bottom=20mm, headsep=10mm, footskip=12mm}
\linespread{1.5}
\setlength\parindent{0pt}
\begin{document}
\begin{center}
\bfseries % Fettdruck einschalten
\sffamily % Serifenlose Schrift
\vspace{-40pt}
Analysis II, Wintersemester 2013/2013, 3. Übungsblatt

Markus Fenske, Luis Herrmann, Tutor: Sebastian Bierke
\vspace{-10pt}
\end{center}

\section*{Aufgabe 3.1}
\begin{enumerate}[a)]
  \item
    \textbf{falsch}

    Es reicht, sich einen kleinen Schritt (beliebig kleines
    $\epsilon$) senkrecht zur Ebene zu bewegen um die Ebene zu verlassen.
    Die Menge besteht also ganz offensichtlich nur aus Randpunkten
    besteht und ist daher nicht offen.
  \item
    \textbf{wahr}

    Die Menge ist abgeschlossen, weil sie nur aus Randpunkten besteht (siehe
    oben).
  \item
    \textbf{wahr}

    Die Menge ist offensichtlich zusammenhängend, weil konvex (siehe unten) und
    nicht leer.

  \item
    \textbf{falsch}

    Die Menge ist unbeschränkt (da sie nicht in eine Kugel endlicher Größe passt).
  \item
    \textbf{falsch}

    Die Menge kann nicht kompakt sein, weil sie unbeschränkt ist.
  \item
    \textbf{wahr}

    Die Menge ist konvex, weil bekanntermaßen die Verbindungsgerade zwischen
    zwei beliebigen Punkten in der Ebene komplett in der Ebene liegt.
\end{enumerate}

\section*{Aufgabe 3.2}

\begin{enumerate}[(1)]
  \item
    Wenn $A$ offen ist, kann ich sie durch Grenzübergang verlassen. Seien $a
    \notin A$ ein beliebiger Randpunkt von $A$ (der nicht in der Menge liegt,
    weil die Menge offen ist). Es existiert auf jeden Fall eine Folge $a_n \in
    A \forall n$, für die gilt:

    \begin{equation*}
      \lim_{n \to \infty} a_n = a
    \end{equation*}

    Analog gilt für eine Folge $b_n$ mit den gleichen Eigenschaften wie oben
    und $b \notin B$:
    \begin{equation*}
      \lim_{n \to \infty} b_n = b
    \end{equation*}

    Also ist:
    \begin{equation*}
      \lim_{n \to \infty} (a_n + b_n) = \lim_{n \to \infty} a_n + \lim_{n \to \infty} b_n = (a + b) \notin A + B
    \end{equation*}

    Also enthält $A+B$ keinen ihrer Randpunkte und ist daher offen.

  \item
    Wenn $A$ zusammenhängend ist, kann man durch zwei Punkte $a_1, a_2 \in A$
    mindestens eine Kurve $k_a(t), t \in [0,1], k_a(0) = a_1, k_a(1) = a_2$
    finden, so dass alle Punkte dieser Kurve in $A$ liegen: $k_a(t) \in A$.

    Für $B$ analog mit analogen Bezeichnungen.

    Sei $C = A+B$ die Summe. Seien $c_1$ und $c_2$ beliebige Elemente aus $C$.
    Dann kann man $c_1 = a_1 + b_1$ und $c_2 = a_2 + b_2$ schreiben. Diese
    beiden Elemente werden durch die Kurve $k_c(t) = k_a(t) + k_b(t)$
    verbunden. Alle Punkte liegen in $C$.

    Also ist $A+B$ zusammenhängend.
  \item
    Wenn $A$ kompakt ist, heißt das, dass jede Folge $a_m$ von Elemementen aus
    $A$ eine Teilfolge enthält, die gegen einen Grenzwert aus $A$ konvergiert.
    Diese Teilfolge sei $a_{m_k}$. Der Grenzwert sei $a$. Für $B$ analog. Es
    gilt also:

    \begin{equation*}
      \lim (a_{m_k} + b_{m_k}) = a + b \in A + B
    \end{equation*}

    Da $a_{m_k} + b_{m_k}$ eine Folge in $A+B$ ist (und Teilfolge von sich
    selber) und der Grenzwert aufgrund der Stetigkeit der Vektoraddition in der
    Menge liegt, ist $A+B$ auch kompakt.

  \item
    Zur Bonusaufgabe:

    Sei die Menge $A := \{(x,y): y = \frac{1}{x}, x \ge 1\}$. Wir
    haben in Aufgabe $2.9$ bewiesen, dass die Graphen stetiger Funktionen $f:
    \mathbb{R} \to \mathbb{R}$ im $\mathbb{R}^2$ abgeschlossene Mengen sind. Da der
    Graph im Punkt $(1,1)$ abgeschnitten ist und dieser Punkt ein Randpunkt ist,
    handelt es sich bei $A$ um eine abgeschlossene Menge.

    Sei die Menge $B$ die $x$-Achse. Diese kann auch als Graph der Funktion $f(x) =
    0$ interpretiert werden und ist daher ebenfalls abgeschlossen.

    Sei $t \in \mathbb{R}, t > 1$. Dann ist $a = (t, \frac{1}{t}) \in A$. Außerdem
    $b = (-t, 0) \in B$. Dann muss $a+b = (0, \frac{1}{t}) \in A+B$ sein.

    Dann gilt
    \begin{equation*}
      \lim_{t \to \infty} (a+b) = (0,0) \notin A+B
    \end{equation*}

    Ich kann die Menge also durch Grenzübergang verlassen, also ist $A+B$ nicht
    abgeschlossen. Daraus folgt: $A + B$ muss nicht abgeschlossen sein, wenn $A$
    und $B$ beide abgeschlossen sind.
\end{enumerate}

\section*{Aufgabe 3.3}

\begin{enumerate}[(1)]
  \item
    Sei $D = [0,23]$, sei $E = ]23,42]$ (das sind Teilmengen des
    $\mathbb{R}^1$).  Sei $f(x) = 23$, Sei $g(x) = 42$. Dann ist es
    offensichtlich, dass $h(x)$ nicht stetig ist.

  \item
    Wenn sich $D$ und $E$ nicht schneiden, ist offensichtlich, dass $h$ stetig
    ist, denn die Funktion ist auf keinem Randpunkt definiert. Es gibt immer
    eine Definitionslücke zwischen $f$ und $g$.

    Wenn sich $D$ und $E$ schneiden, gibt es eine Schnittmenge $\empty \neq S = D \cap E$.
    Für $x \in S$ gilt laut Aufgabenstellung $f(x) = g(x)$.

    Die Stetigkeit außerhalb der Schnittmenge steht außer Frage, das $f$ und
    $g$ stetig sind.

    Die Stetigkeit innerhalb der Schnittmenge steht ebenfalls außer Frage (aus
    obigen Gründen).

    Interessant sind also nur die Randpunkte der Schnittmenge. Sei $d \in D$
    Randpunkt von $S$ ($S$ ist offen, deswegen $d \neq S$). An diesem Punkt hat
    $f$ den Wert $f(d)$. Sei $s_n$ eine Folge in $S$ mit dem Grenzwert $d$.
    Dann gilt aufgrund der Stetigkeit von $f$.

    \begin{equation*}
      \lim_{n \to \infty} f(s_n) = f(\lim s_n) = f(d)
    \end{equation*}

    Und weil $f$ und $g$ im Bereich von $S$ übereinsteimmen:

    \begin{equation*}
      \lim_{n \to \infty} f(s_n) = g(d)
    \end{equation*}

    Also geht es an dieser Stelle nahtlos weiter, die Funktion $h$ ist stetig.
\end{enumerate}

\section*{Aufgabe 3.4}
Wenn $D$ stetig wäre, müsste gelten ($\lim$ ist hier immer $\lim_{n \to \infty}$)

\begin{equation*}
  \lim D(f_n)(x) = D(\lim f_n)(x) \;\forall x \in [0, 2 \pi]
\end{equation*}

Wir nutzen die gegebene Funktionsfolge $f_n(x) = \frac{\sin nx}{n}$ und $x=0$
um das zu widerlegen.

\begin{equation*}
  \lim D(f_n)(0) = \lim \cos(n \cdot 0) = 1
\end{equation*}

Hingegen
\begin{equation*}
  D(\lim f_n)(0) = D(0) = 0
\end{equation*}

Also ist $D$ nicht stetig.

\end{document}
