\documentclass[a4paper,german,12pt,smallheadings]{scrartcl}
\usepackage[T1]{fontenc}
\usepackage[utf8]{inputenc}
\usepackage{babel}
\usepackage{tikz}
\usepackage{geometry}
\usepackage{amsmath}
\usepackage{amssymb}
\usepackage{float}
\usepackage{enumerate}
%\usepackage{wrapfig}
\usepackage[thinspace,thinqspace,squaren,textstyle]{SIunits}
\restylefloat{table}
\renewcommand{\thefootnote}{\fnsymbol{footnote}}
\geometry{a4paper, top=15mm, left=20mm, right=40mm, bottom=20mm, headsep=10mm, footskip=12mm}
\linespread{1.5}
\setlength\parindent{0pt}
\begin{document}
\begin{center}
\bfseries % Fettdruck einschalten
\sffamily % Serifenlose Schrift
\vspace{-40pt}
Analysis II, Wintersemester 2013/2013, 4. Übungsblatt

Markus Fenske, Luis Herrmann, Tutor: Sebastian Bierke
\vspace{-10pt}
\end{center}

\section*{Aufgabe 4.1}
\begin{enumerate}[(1)]
  \item
    Wir arbeiten hier im euklidischen Raum, also ist die Norm definiert über den
    Satz des Pythagoras in $n$ Dimensionen. Wir wenden die Kettenregel an, nutzen
    die Definition des Skalarproduktes, das Ausklammern in Summen, usw.

    \begin{align*}
      \frac{d}{dt} ||\vec{f}(t)|| &= \frac{d}{dt} \sqrt{\sum_{i=1}^n f_i(t)^2}
       = \frac{1}{2 \sqrt{\sum_{i=1}^n f_i(t)^2}} \sum_{i=1}^n 2 f_i(t) f_I'(t)
       = \frac{\vec{f}'(t) \bullet \vec{f}(t)}{||\vec{f}(t)||}
    \end{align*}
  \item
    Seien alle Komponenten des $k$-dimensionalen Vektorfeldes gleich (das spart Arbeit). Die
    Komponenten sollen $f_1(t), f_2(t), \dots, f_n(t), \dots, f_k(t)$ heißen. Da sich der
    Grenzwert in den Vektor ziehen lässt, reduziert sich das Problem darauf,
    Funktionen derart zu finden, dass $f_n(t_0) = 0$, aber $f_n'(t_0)$
    definiert bzw.  $f_n'(t_0)$ undefiniert.

    Beispiel für Ersteres: $\vec{f}(t) = \vec{0}$. Es gilt für jede Komponente:

    \begin{equation*}
      f_n'(t) = \lim_{h \to 0} \frac{f(t + h) - f(t)}{h} = \lim_{h \to 0} \frac{0}{h} = 0
    \end{equation*}

    Also ist $\vec{f}'(t) = \vec{0}$, insbesondere auch an der Stelle $t = t_0$.

    Beispiel für Zweiteres: $f_n(t) = \sqrt{t_0-t}$. Es gilt $f_n(t_0) =
    \sqrt{0} = 0$ und damit $\vec{f}(t_0) = \vec{0}$. Die Wurzelfunktion ist
    bekanntlich im Nullpunkt nicht differenzierbar:

    \begin{equation*}
      \lim_{h \to 0} f_n'(t_0) = \lim_{h \to 0} \frac{f_n(t_0 + h) - f_n(t_0)}{h} = 
      \lim_{h \to 0} \frac{\sqrt{h} - 0}{h} = \lim_{h \to 0} \frac{1}{\sqrt{h}}
    \end{equation*}

    Dieser Grenzwert divergiert, also existiert $\vec{f}'(t_0)$ nicht.
\end{enumerate}

\section*{Aufgabe 4.2}

Diese Kurve ist offenbar symmetrisch (Begründung: siehe Fußnote auf dem
Aufgabenblatt) und beim Quadrantenübergang nicht differenzierbar. Aufgrund der
Symmetrie ist die Länge in allen 4 Quadranten gleich. Es gilt also

\begin{align*}
  L &= 4 \int\limits_{\overline{\gamma}}  ds \qquad \text{(wobei }\overline{\gamma}\text{ die Kurve im 1. Quadranten ist)} \\
    &= 4 \int_0^{\frac{\pi}{2}} \sqrt{\left(\frac{dx}{dt}\right)^2 + \left(\frac{dy}{dt}\right)^2} dt
\end{align*}

Mit der gegebenen Parametrisierung $x=a \cos^3 t$, $y = a \sin^3 t$:

\begin{align*}
  &= 4 \int_0^{\frac{\pi}{2}}   \sqrt{\left(-3a \sin(t) \cos^2(t) \right)^2 + \left(3a \sin^2(t) \cos(t)\right)^2} \; dt \\
  &= 12a \int_0^{\frac{\pi}{2}} \sqrt{\sin^2(t) \cos^4(t) + \sin^4(t) \cos^2(t)} \; dt \\
  &= 12a \int_0^{\frac{\pi}{2}} \sqrt{(\sin^2(t) + \cos^2(t)) \sin^2(t)\cos^2(t)} \; dt \\
  &= 12a \int_0^{\frac{\pi}{2}} \sqrt{\sin^2(t)\cos^2(t)} \; dt \\
  &= 6a \int_0^{\frac{\pi}{2}}  \sqrt{\sin^2(2t)} \; dt \\
  &= 6a \int_0^{\frac{\pi}{2}}  \sin(2t) \; dt \qquad \text{(weil $\sin(2t) > 0$ für $0 \le t \le \frac{\pi}{2}$)} \\
  &= 3a \int_0^{\pi}            \sin(t) \; dt \\
  &= 6a
\end{align*}


\begin{equation*}
\end{equation*}


\end{document}
