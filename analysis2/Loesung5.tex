\documentclass[a4paper,german,12pt,smallheadings]{scrartcl}
\usepackage[T1]{fontenc}
\usepackage[utf8]{inputenc}
\usepackage{babel}
\usepackage{tikz}
\usepackage{geometry}
\usepackage{amsmath}
\usepackage{amssymb}
\usepackage{float}
\usepackage{enumerate}
\usepackage{cancel}
%\usepackage{wrapfig}
\usepackage[thinspace,thinqspace,squaren,textstyle]{SIunits}
\restylefloat{table}
\renewcommand{\thefootnote}{\fnsymbol{footnote}}
\geometry{a4paper, top=15mm, left=20mm, right=40mm, bottom=20mm, headsep=10mm, footskip=12mm}
\linespread{1.5}
\setlength\parindent{0pt}
\begin{document}
\begin{center}
\bfseries % Fettdruck einschalten
\sffamily % Serifenlose Schrift
\vspace{-40pt}
Analysis II, Wintersemester 2013/2013, 5. Übungsblatt

Markus Fenske, Luis Herrmann, Tutor: Sebastian Bierke
\vspace{-10pt}
\end{center}

\section*{Aufgabe 5.1}
Der Halbkreisbogen (den wir hier $\overline{\gamma}(\phi)$ nennen wollen) ist
parametrisiert durch
\begin{equation*}
  x = R \sin \phi,\quad y = R \cos \phi, \quad \phi \in [0, \pi]
\end{equation*}

Das führt auf die Ableitungen
\begin{equation*}
  \frac{d}{d \phi} x= \cos \phi,\quad \frac{d}{d \phi} y = -\sin \phi
\end{equation*}

Die Länge ist folglich
\begin{align*}
  L &= \int_{\overline{\gamma}} ds \\ 
    &= \int_0^\pi \sqrt{\left(\frac{dx}{d\phi}\right)^2 + \left(\frac{dy}{d \phi}\right)^2} \; d\phi \\
    &= \int_0^\pi \sqrt{R^2 \cos(\phi)^2 + R^2 \sin(\phi)^2} \; d\phi \\
    &= \int_0^\pi \sqrt{R^2 \underbrace{(\cos(\phi)^2 + \sin(\phi)^2)}_{=1}} \; d\phi \\
    &= \int_0^\pi |R| \; d\phi \\
    &= R \int_0^\pi \; d\phi \qquad \text{(weil $R$ immmer $>0$)} \\
    &= \pi R
\end{align*}

Der Schwerpunkt auf der $x$-Achse ist dann
\begin{align*}
  x_s &= \frac{1}{L} \int_{\overline{\gamma}} x \; ds \\
      &= \frac{1}{\pi R} \int_0^\pi x \; ds \\
      &= \frac{1}{\pi \cancel{R}} \int_0^\pi \cancel{R} \cos(\phi) \sqrt{R^2 \cos(\phi)^2 + R^2 \sin(\phi)^2} \; d\phi \\
      &= \frac{1}{\pi} \int_0^\pi \cos(\phi) \sqrt{R^2 \underbrace{(\cos(\phi)^2 + \sin(\phi)^2)}_{=1}} \; d\phi \\
      &= \frac{1}{\pi} \int_0^\pi \cos(\phi) |R| \; d\phi \\
      &= \frac{R}{\pi} \int_0^\pi \cos \phi  \; d\phi \qquad \text{(weil $R$ immer $>0$)}\\
      &= \frac{R}{\pi} \left[ \sin \phi\right]_0^\pi \\
      &= \frac{2R}{\pi}
\end{align*}

Die Oberfläche einer Sphäre berechnet sich dann gemäß der \textsc{Guldin}schen
Regel über die Mantelfläche der Kurve $\overline{\gamma}$. Diese wollen wir
hier $M$ nennen.

\begin{equation*}
  M = 2 \pi L x_s = 2 \pi \cdot \cancel{\pi} R \cdot \frac{2R}{\cancel{\pi}} = 4 \pi R^2
\end{equation*}



\end{document}
