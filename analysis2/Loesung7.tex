\documentclass[a4paper,german,12pt,smallheadings]{scrartcl}
\usepackage[T1]{fontenc}
\usepackage[utf8]{inputenc}
\usepackage{babel}
\usepackage{tikz}
\usepackage{geometry}
\usepackage{amsmath}
\usepackage{amssymb}
\usepackage{float}
\usepackage{enumerate}
\usepackage{cancel}
\usepackage{pgfplots}
\usepackage{commath}
\pgfplotsset{compat=1.7}
\usepgfplotslibrary{polar}
%\usepackage{wrapfig}
\usepackage[thinspace,thinqspace,squaren,textstyle]{SIunits}
\restylefloat{table}
\renewcommand{\thefootnote}{\fnsymbol{footnote}}
\geometry{a4paper, top=15mm, left=20mm, right=40mm, bottom=20mm, headsep=10mm, footskip=12mm}
\linespread{1.5}
\setlength\parindent{0pt}
\begin{document}
\begin{center}
\bfseries % Fettdruck einschalten
\sffamily % Serifenlose Schrift
\vspace{-40pt}
Analysis II, Wintersemester 2013/2013, 7. Übungsblatt

Markus Fenske, Luis Herrmann, Tutor: Sebastian Bierke
\vspace{-10pt}
\end{center}
\allowdisplaybreaks % Seitenumbrüche in Formeln erlauben
\section*{Aufgabe 7.1}
\begin{enumerate}[(1)]
  \item
    \begin{align*}
                            &x^2 + y^2 + z^2 = 3xyz \\
      \Leftrightarrow\quad  &x^2 + y^2 + z^2 - 3xyz = 0 \\
      \Rightarrow\quad      &F(x,y,z) = x^2 + y^2 + z^2 - 3xyz \\
    \end{align*}

    Wir prüfen $\pd{F}{z}\del{1,1,1} \neq 0$:
    \begin{align*}
                          &\pd{F}{z} = 2z - 3xy \\
      \Rightarrow\quad    &\pd{F}{z}(1,1,1) = 2-3 = -1
    \end{align*}

    Also gilt nach dem Satz über implizite Funktionen:
    \begin{align*}
      \pd{f}{x}(1,1) = \frac{\pd{F}{x}(1,1,f(1))}{\pd{F}{z}(1,1,f(1))}
    \end{align*}

    Nach dem Satz über implizite Funktionen gilt:
    \begin{align*}
      \pd{f}{x}(x,y) = \frac{\pd{F}{x}\del{x,y,f(x,y)}}{\pd{F}{z}\del{x,y,f(x,y)}}
    \end{align*}

    Dabei muss $f(x,y)$ bei $(x,y,z) = (1,1,1)$ erfüllen $f(1,1) = 1$. Wir rechnen also nach:

    \begin{align*}
      \pd{f}{x}\del{x,y,f(x,y)} = \pd{}{x} \del{x^2 + y^2 + f(x,y)^2 - 3xyf(x,y)} = 2x
    \end{align*}

    Durch Ableiten ergibt sich:
    \begin{align*}
                           &\pd{F}{x}\del{x,y,f(x,y)} = 0 \\
      \Leftrightarrow\quad &\pd{}{x}(x^2 + y^2 + f(x,y)^2 - 3xyf(x,y)) = 0 \\
      \Leftrightarrow\quad &2x +2f(x,y) \pd{}{x} f(x,y) - 3y f(x,y) - 3xy \pd{}{x} f(x,y) = 0 \\
      \Leftrightarrow\quad &2f(x,y)\pd{}{x}f(x,y) - 3xy \pd{}{x}f(x,y) = 3yf(x,y) - 2x \\
      \Leftrightarrow\quad &\pd{}{x}f(x,y) \del{2f(x,y) - 3xy} = 3yf(x,y) - 2x \\
      \Leftrightarrow\quad &\pd{}{x}f(x,y) = \frac{3yf(x,y) - 2x}{2f(x,y) - 3xy}
    \end{align*}

    Nun setzen wir ein: $x=1, y=1, f(x,y) = 1$:

    \begin{align*}
      \pd{}{x} f(1,1) = \frac{3-2}{2-3} = -1
    \end{align*}

    Wir verfahren analog für $g$. Sei $y = g(x,z)$. Prüfe:
    \begin{align*}
      \pd{F}{y}(x,y,z) = \pd{}{x}(x^2 + y^2+z^2-3xyz) = 2y-3xz
    \end{align*}

    Im Punkt $(1,1,1)$:
    \begin{align*}
      \pd{F}{y}(1,1,1) = 2-3 = -1 \neq 0
    \end{align*}

    Damit berechnen wir:
    \begin{align*}
      \pd{g}{x}(x,z) &= \frac{\pd{F}{x}(x, g(x,z), z)}{\pd{F}{y}(x,y,z)} \\
                     &= \frac{\pd{}{x}(x^2+y^2+z^2 - 3xyz}{\pd{}{y} (x^2+y^2+z^2 - 3xyz} \\
                     &= \frac{-(2x-3yz)}{2y - 3xz}
    \end{align*}

    Für $(x,y,z) = (1,1,1)$:
    \begin{align*}
      \pd{g}{x}(1,1) = \frac{-(2-3)}{2-3} = -1
    \end{align*}
  \item
    Wir betrachten $h(x,y,z = xy^2z^3 \Rightarrow h(x,y,f(x,y)) = xy^2f(x,y)^3$.

    \begin{align*}
      \pd{}{x} h(x,y,f(x,y)) &= \pd{}{x}(xy^2f(x,y)^3) \\
                             &= y^2f(x,y)^3 + xy^2 \pd{}{x}f(x,y)^3 \\
                             &= y^2f(x,y)^3 + 3xy^2f(x,y)^2 \pd{}{x} f(x,y)
    \end{align*}

    Im Punkt $(1,1)$:
    \begin{align*}
      \pd{}{x} h(x,y,f(x,y))(1,1) = \pd{}{x} h(1,1,f(1,1)) = 1^2\cdot 1^3 + 3 \cdot 1 \cdot 1 \cdot 1 \cdot (-1) = 1-3 = 2
    \end{align*}

    Analog verfahren wir im zweiten Fall:
    \begin{align*}
      \pd{}{x}h(x,g(x,z),z) &= \pd{}{x}(xg(x,z)^2z^3) \\
                            &= g(x,z)^2z^3 + x2g(x,z) \pd{}{x} g(x,z) z^3
    \end{align*}

    Demnach ist
    \begin{align*}
      \pd{}{x} h(1,g(1,1),1) = 1^21^2 + 1 \cdot 2 \cdot 1 \cdot (-1) \cdot 1^3 = 1-2 = -1
    \end{align*}
\end{enumerate}

\section*{Aufgabe 7.2}
\begin{enumerate}[(1)]
  \item Dazu pr"ufen wir zun"achst:
    \begin{align*}
      (1,2): &1\cdot e^{0+0} + 2 \cdot 0 \cdot 0 = 1 &&\Rightarrow \text{ wahre Aussage} \\
             &2\cdot e^{0+0} - \frac{0}{1+0} = 2     &&\Rightarrow \text{ wahre Aussage}
    \end{align*}

    Ferner gilt:
    \begin{align*}
                           &ye^{u-v} - \frac{u}{1+v} = 2x \\
      \Leftrightarrow\quad &ye^{2u} - \frac{ue^{u+v}}{1+v} = 2xe^{u+v} \\
      \Leftrightarrow\quad &\frac{ye^{2u}}{2} - \frac{ue^{u+v}}{2\del{1+v}} = xe^{u+v} \\
    \end{align*}

    ...WTF? Luis, was ist da los?
\end{enumerate}


\section*{Aufgabe 7.3}
  $a<b, a,b \in \mathbb{R}^+$. Suche $x,y$ sodass $a < x < y < b$ und
  $\frac{xy}{(a+x)(x+y)(y+b)}$ maximal wird. Das hei"st das Skalarfeld $f(x,y)$

  \begin{equation}
    f(x,y) = \frac{xy}{(a+x)(x+y)(y+b)}
  \end{equation}
  ist unter gegebenen Nebenbedingungen zu maximieren. % Luis schreibt "minimieren", ist nicht richtig, oder?

  Notwendige Bedingung: $\vec{\operatorname{grad}} f(x,y) = \vec{0}$.
  \begin{equation}
    \begin{pmatrix}
      \pd{f}{x}(x,y) \\
      \pd{f}{y}(x,y)
    \end{pmatrix}
    =
  \begin{pmatrix} 0 \\ 0 \end{pmatrix}
  \end{equation}

  Wir l"osen mit Quotientenregel
  \begin{enumerate}[(1)]
    \item
      \begin{align*}
        \pd{f}{x}(x,y) &= \frac{y(x+a)(y+x)(y+b) - xy(y+b)\del{(y+x)+(x+a)}}{(x+a)^2(y+x)^2(y+b)^2} \\
                       &= \frac{ay^2 -yx^2}{(a+x)^2(b+y)(x+y)^2}
      \end{align*}
      \item
        Aus Symmetriegr"unden (vertauschen von $a,b$ und $x,y$) gilt
        \begin{align*}
        \pd{f}{x}(x,y) &= \frac{bx^2 -xy^2}{(b+y)^2(a+x)(x+y)^2}
        \end{align*}
  \end{enumerate}

  Um nun den Extremwert zu finden, setzen wir den Gradienten Null. Dazu müssen die Zähler null werden. Wir erhalten das Gleichungssystem:
  \begin{align*}
    ay^2 - yx^2 &= 0\\
    bx^2 - xy^2 &= 0
  \end{align*}

  Da $y > 0$ erhalten wir aus der ersten Gleichung
  \begin{equation*}
    y = \frac{x^2}{a}
  \end{equation*}

  Einsetzen in die zweite Gleichung liefert
  \begin{align*}
                         & bx = \frac{x^4}{a^2} \\
    \Leftrightarrow\quad & x^3 = a^2b \\
    \Leftrightarrow\quad & x   = \sqrt[3]{a^2b}
  \end{align*}

  Aus Symmetriegründen (siehe oben) folgt damit das Minimum
  \begin{align*}
    x = \sqrt[3]{a^2 b}, y = \sqrt[3]{ab^2}
  \end{align*}

  Prüfen, ob dies $a<x<y<b$ erfüllt:
  \begin{align*}
    a<x : a = \sqrt[3]{a^3} < \sqrt[3]{a^2b} = x \\
    x<y : y = \sqrt[3]{a^2b} < \sqrt[3]{ab^2} = y \\
    y<b : y = \sqrt[3]{a^2b} < \sqrt[3]{b^3} = b
  \end{align*}

  Dieser Punkt ist also stark maximumsverdächtig (weil Extremwert). Zu prüfen,
  ob es sich wirklich um ein Maximum oder ein Minimum handelt war nicht mehr
  Teil der Aufgabenstellung.
\end{document}
