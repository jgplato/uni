\documentclass[a4paper,german,12pt,smallheadings]{scrartcl}
\usepackage[T1]{fontenc}
\usepackage[utf8]{inputenc}
\usepackage{babel}
\usepackage{tikz}
\usepackage{geometry}
\usepackage{amsmath}
\usepackage{amssymb}
\usepackage{float}
\usepackage{enumerate}
\usepackage{cancel}
\usepackage{pgfplots}
\usepackage{commath}
\pgfplotsset{compat=1.7}
\usepgfplotslibrary{polar}
%\usepackage{wrapfig}
\usepackage[thinspace,thinqspace,squaren,textstyle]{SIunits}
\restylefloat{table}
\renewcommand{\thefootnote}{\fnsymbol{footnote}}
\geometry{a4paper, top=15mm, left=20mm, right=40mm, bottom=20mm, headsep=10mm, footskip=12mm}
\linespread{1.5}
\setlength\parindent{0pt}
\begin{document}
\begin{center}
\bfseries % Fettdruck einschalten
\sffamily % Serifenlose Schrift
\vspace{-40pt}
Analysis II, Wintersemester 2013/2013, 8. Übungsblatt

Markus Fenske, Luis Herrmann, Tutor: Sebastian Bierke
\vspace{-10pt}
\end{center}
\allowdisplaybreaks % Seitenumbrüche in Formeln erlauben
\section*{Aufgabe 8.1}
Da $f(x,y) = 2y$ linear nur von $y$ abhängig ist (und daher bei größer
werdendem $y$ monoton steigt), muss das Minimum, falls existent, an dem, durch
die Nebenbedingung gegebenen, kleinstmöglichen $y$ liegen.

Die Nebenbedingung kann parametrisiert werden durch $y(x) = \sqrt[5]{3x^2}$.
Wegen der Monotonie der Wurzel suchen wir das Minimum von $3x^2$, das liegt
bekanntermaßen bei $x=0$. Daraus folgt $y(x) = 0$.

Daher gibt es ein Minimum und dieses liegt bei $(x,y) = (0,0)$.
\end{document}
