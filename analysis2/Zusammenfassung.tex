\documentclass[a4paper,german,10pt,smallheadings]{scrartcl}
\usepackage[T1]{fontenc}
\usepackage[utf8]{inputenc}
\usepackage{babel}
\usepackage{tikz}
\usepackage{geometry}
\usepackage{amsmath}
\usepackage{amssymb}
\usepackage{float}
\usepackage{enumerate}
%\usepackage{wrapfig}
\usepackage[thinspace,thinqspace,squaren,textstyle]{SIunits}
\restylefloat{table}
\geometry{a4paper, top=15mm, left=20mm, right=40mm, bottom=20mm, headsep=10mm, footskip=12mm}
\linespread{1.5}
\setlength\parindent{0pt}
\begin{document}
\begin{center}
\bfseries % Fettdruck einschalten
\sffamily % Serifenlose Schrift
\vspace{-40pt}
Analysis 2: Rezepte

Markus Fenske
\vspace{-10pt}
\end{center}

\section{Altes Zeug}

Kompletter Stoff von Analysis 1

\section{Grenzwert von mehrdimensionalen Funktionen zeigen/widerlegen}

\textbf{Zum Widerlegen:} Zwei Grenzwerte auf verschiedenen Kurven finden, die
ungleich sind.

Beispiel:
\begin{equation*}
  \lim_{(x,y) \to (0,0)} \frac{x-y+x^2+y^2}{x+y}
\end{equation*}

Mit $k(t) = (t,0), t \to 0$ ist Grenzwert $1$. Mit $k(t) = (0,t), t \to 0$ ist
Grenzwert $-1$. Also existiert kein Grenzwert.

\textbf{Zum Beweisen:} Umformen und/oder abschätzen mit Sandwich-Kriterium.
Umformen in Polarkoordinaten, wenn nur $x^2+y^2 = r^2$ Terme auftauchen. Regel
von L'Hospital (Wenn im Bruch oben und unten bei Grenzwert 0, dann oben und
unten durch Ableitung ersetzen) ist nützlich.

Beispiel:
\begin{equation*}
  \lim \frac{x^2+y^2}{\sqrt{x^2+y^2+1}-1} = \lim \frac{r^2}{\sqrt{r^2+1}-1} \overset{\text{L'Hospital}}{=} \lim 2\sqrt{r^2+1} = 2
\end{equation*}





\end{document}
