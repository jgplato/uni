\documentclass[a4paper,german,12pt,smallheadings]{scrartcl}
\usepackage[T1]{fontenc}
\usepackage[utf8]{inputenc}
\usepackage{babel}
\usepackage{tikz}
\usetikzlibrary{calc}
\usepackage{tkz-euclide}
\usetkzobj{all}
\usepackage{pgfplots}
\usepackage{geometry}
\usepackage[fleqn]{amsmath}
\usepackage{amssymb}
\usepackage{float}
\usepackage{enumerate}
\usepackage{commath} % http://tex.stackexchange.com/questions/14821/whats-the-proper-way-to-typeset-a-differential-operator
\usepackage{cancel}
\usepackage{gnuplottex}

% Number only referenced equations
\usepackage[fleqn]{mathtools}
\mathtoolsset{showonlyrefs}

%\usepackage{wrapfig}
\usepackage{siunitx}
\sisetup{locale = DE}

% New command for color underlining
\usepackage{xcolor}

\newsavebox\MBox
\newcommand\colul[2][red]{{\sbox\MBox{$#2$}%
  \rlap{\usebox\MBox}\color{#1}\rule[-1.2\dp\MBox]{\wd\MBox}{0.5pt}}}

% Stupid units for astronomy
\DeclareSIUnit\day{d}
\DeclareSIUnit\year{yr}
\DeclareSIUnit\au{AU}
% Astronomy-Style angles (hours, minutes, seconds)
\newcommand*{\ra}[2][]{{
  \def\SIUnitSymbolDegree{\textsuperscript{h}}%
  \def\SIUnitSymbolArcminute{\textsuperscript{m}}%
  \def\SIUnitSymbolArcsecond{\textsuperscript{s}}%
  \ang[#1]{#2}}%
}

\restylefloat{table}
\geometry{a4paper, top=15mm, left=10mm, right=20mm, bottom=20mm, headsep=10mm, footskip=12mm}
\linespread{1.5}
\setlength\parindent{0pt}
\DeclareMathOperator{\Tr}{Tr}
\DeclareMathOperator{\Var}{Var}

\begin{document}
\allowdisplaybreaks % Seitenumbrüche in Formeln erlauben
\begin{center}
\bfseries % Fettdruck einschalten
\sffamily % Serifenlose Schrift
\vspace{-40pt}
Einführung in die Astronomie und Astrophysik, Sommersemester 2014, Aufgabenblatt 1

Markus Fenske, Julia Schuch, Tutor: Daniel Härdt
\vspace{-10pt}
\end{center}
\section*{Aufgabe 1: Präzession}
\begin{enumerate}[a)]
  \item
  Aus dem Übungsskript (S. 17, 1.1.5) entnehmen wir, dass die Präzessionsperiode
  etwa 25800 Jahre beträgt. Da die Tierkreiszeichen die Ekliptik in 12 Abschnitte
  einteilt, die jeweils einem Monat zugeordnet werden, dauert die Verschiebung um
  einen Monat

  \begin{equation}
    \SI{25800}{\year} \cdot \frac{1}{12} = \SI{2150}{\year}
  \end{equation}

  \item
    Mit den im Übungsskript (S. 22, 1.3, Nr. 2) gegebenen Formeln ergibt sich
    \begin{align}
      &\Delta \alpha = \ang{0;17;23} \\
      &\Delta \delta = \ang{0;0;15.87}
    \end{align}

    Das bedeutet, dass der Polarstern, wie alle \textit{Fix}sterne, seine
    (scheinbare) Position ändert und natürlich nicht am Himmelsnordpol
    festgetackert ist. Es wird also in einigen Jahrhunderten bis Jahrtausenden
    einen hellernen Stern geben, den man dann eher als Polarstern bezeichnen
    kann.
  \item
    Weil die Objekte des Sonnensystems keine Fixsterne sind. Sie laufen auf
    elliptischen Bahnen, die separat berechnet werden müssen.
  \item
    Wir suchen
    \begin{align}
      \del{46 + 20 \tan \delta \sin \alpha} &= 0 \\
      20 \cos \alpha &= 0
    \end{align}

    Damit die zweite Gleichung erfüllt werden kann, muss $\cos \alpha = 0$,
    somit muss $\alpha = \ang{90} \cup \alpha = \ang{270}$ sein.

    Damit reduziert sich die erste Gleichung auf
    \begin{align}
      &46 \pm 20 \tan \delta = 0 \\
      \Leftrightarrow \quad & \tan \delta = \pm \frac{23}{20}
    \end{align}

    Die Lösungen sind $\delta = \pm \ang{66.5} \cup \delta = \pm \ang{113.5}$,
    wobei die zweite Lösung wegfällt, da sie außerhalb des Wertebereichs liegt.
    Es ergeben sich also die Positionen $(\ra{6},\ang{-66;30;0})$ und $\ra{18},
    \ang{66;30;0}$.
    % FIXME: Interessant. Das sind genau die Sommersonnenwenden auf den Polarkreisen. Warum?
\end{enumerate}

\section*{Aufgabe 2: Julianisches Datum}
Das Julianische Datum zählt die Tage seit dem 1. Januar -4712, 12:00 Uhr UT.
Der Tagesanfang wurde auf Mittags gelegt, damit sich keine Änderungen der
Tageszahl bei nächtlichen astronomischen Beobachtungen ergeben.  2456777{,}5
gibt also das Datum um 0 Uhr UT an, während 2456778 das Datum um 12 Uhr UT ist.

\section*{Aufgabe 3: Schiffbruch}

Die Höhe des Sterns ist $h = \ang{90}$, der Azimut damit egal. Deswegen setze
ich ihn auf $A = \ang{90}$ und erhalte somit aus der zweiten Azimut-Gleichung
(Übungsskript, S. 23):

\begin{equation}
  \cos h = \cos \delta \sin \tau
\end{equation}

Da $\cos h = 0$ ist, muss $\sin \tau = 0$ sein. Somit ist der Stundenwinkel
$\tau = \ra{0} \cup \tau = \ra{12}$.

Aus der Höhengleichung (Übungsskript, S. 22) erhalte ich

\begin{equation}
  1 = \sin \phi \sin \delta \pm \cos \phi \cos \delta
\end{equation}

Für den positiven Zweig ($\tau = 0$) ist eine mögliche Lösung $\phi = \delta$ (wegen $\sin^2
+ \cos^2 = 1$).

Mit der Formel für Ortssternzeit und Stundenwinkel erhalten wir
\begin{equation}
  \tau = \theta(\lambda_G, 0) - \frac{\lambda}{\SI{15}{\degree\per\hour}} + t \cdot 1{,}00274 - \alpha
\end{equation}

Und somit
\begin{equation}
  \lambda = \SI{15}{\degree\per\hour} \del{\theta(\lambda_G, 0) - \tau + t \cdot 1{,}00274 - \alpha}
\end{equation}

Mit den gegebenen Werten erhalten wir also insgesamt
\begin{equation}
  \lambda = \ang{44;44;22.5} \qquad \phi = \ang{-60;50;2.3}
\end{equation}

Der negative Zweig kann ausgeschlossen werden, denn dieser würde auf einen
positiven Breitengrad führen, also auf eine Insel auf der Nordhalbkugel. Alpha
Centauri ist jedoch auf der Nordhalbkugel nicht zu sehen.

\section*{Aufgabe 4: Neptun}
\begin{enumerate}[a)]
  \item
    Wir erhalten aus dem 3. Keplerschen Gesetz
    \begin{equation}
      \frac{T^2}{a^3} = \frac{4 \pi^2}{G M}
    \end{equation}
    durch Umstellen und Einsetzen der gegebenen Werte
    \begin{equation}
      M \approx \SI{3.304e25}{\kilogram} \approx 5{,}5 \; M_E
    \end{equation}
  \item
    % FIXME: Plot einfügen (wie?)
  \item
    % FIXME: Nochmal was nettes malen?
    Wir nehmen alle Objekte als perfekte Kugeln und deren Bahnen als Kreise an.

    Damit die Objekte eine totale Sonnenfinsternis hervorrufen können, muss der
    scheinbare Durchmesser des Objekts mindestens so groß sein, wie der der
    Sonne.

    Der scheinbare Durchmesser der Sonne ergibt sich durch
    \begin{equation}
      \tan \alpha_0 = 2 \frac{R_0}{r_0}
    \end{equation}
    wobei $R_0$ der Sonnenradius ($\approx \SI{6.95e5}{\kilo\meter}$) ist und $r_0$
    der Abstand zwischen Sonne und Neptun ($\approx \SI{29.7}{\au}$).

    Der scheinbare Durchmesser der jeweiligen Saturnmonde ergibt sich analog
    durch
    \begin{equation}
      \tan \alpha = \frac{D}{r}
    \end{equation}
    wobei $D$ der Durchmesser des Objekts ist und $r$ der Bahnradius.

    Somit muss also
    \begin{equation}
      \frac{D}{r} \ge \frac{2R_0}{r_0} \approx 4{,}68 \cdot 10^{4}
    \end{equation}
    sein, damit eine Sonnenfinsternis theoretisch möglich ist.

    Wir erhalten

    \begin{tabular}{l|r|r|r}
      Name       & $r$                 & $D$                      & $D/r$ \\
      \hline
      Naiad      & $\SI{3.22e-4}{\au}$ & $\SI{67}{\kilo\meter}$   & $2.08 \cdot 10^5$ \\
      Thalassa   & $\SI{3.35e-4}{\au}$ & $\SI{81}{\kilo\meter}$   & $2.42 \cdot 10^5$ \\
      Despina    & $\SI{3.51e-4}{\au}$ & $\SI{150}{\kilo\meter}$  & $4.27 \cdot 10^5$ \\
      Galatea    & $\SI{4.14e-4}{\au}$ & $\SI{175}{\kilo\meter}$  & $4.23 \cdot 10^5$ \\
      Larissa    & $\SI{4.92e-4}{\au}$ & $\SI{195}{\kilo\meter}$  & $3.40 \cdot 10^5$ \\
      S/2004 N 1 & $\SI{7.04e-4}{\au}$ & $\SI{18}{\kilo\meter}$   & $2.56 \cdot 10^4$ \\
      Proteus    & $\SI{7.86e-4}{\au}$ & $\SI{420}{\kilo\meter}$  & $5.34 \cdot 10^5$ \\
      Triton     & $\SI{2.37e-3}{\au}$ & $\SI{2707}{\kilo\meter}$ & $1.14 \cdot 10^6$ \\
      Nereid     & $\SI{3.69e-2}{\au}$ & $\SI{340}{\kilo\meter}$  & $9.21 \cdot 10^3$ \\
      Halimede   & $\SI{1.05e-1}{\au}$ & $\SI{48}{\kilo\meter}$   & $4.57 \cdot 10^2$ \\
      Sao        & $\SI{1.50e-1}{\au}$ & $\SI{44}{\kilo\meter}$   & $2.93 \cdot 10^2$ \\
      Laomedeia  & $\SI{1.58e-1}{\au}$ & $\SI{42}{\kilo\meter}$   & $2.66 \cdot 10^2$ \\
      Psamathe   & $\SI{3.12e-1}{\au}$ & $\SI{38}{\kilo\meter}$   & $1.22 \cdot 10^2$ \\
      Neso       & $\SI{3.23e-1}{\au}$ & $\SI{60}{\kilo\meter}$   & $1.86 \cdot 10^2$ \\
    \end{tabular}

    Es können also Naiad, Thalassa, Despina, Galatea, Lariassa, Proteus und
    Trition theoretisch eine totale Sonnenfinsternis auf dem Neptun
    hervorrufen.



\end{enumerate}



\end{document}
