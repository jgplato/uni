\documentclass[a4paper,german,12pt,smallheadings]{scrartcl}
\usepackage[T1]{fontenc}
\usepackage[utf8]{inputenc}
\usepackage{babel}
\usepackage{tikz}
\usetikzlibrary{calc}
\usepackage{tkz-euclide}
\usetkzobj{all}
\usepackage{pgfplots}
\usepackage{geometry}
\usepackage[fleqn]{amsmath}
\usepackage{amssymb}
\usepackage{float}
\usepackage{lscape}
\usepackage{enumerate}
\usepackage{commath} % http://tex.stackexchange.com/questions/14821/whats-the-proper-way-to-typeset-a-differential-operator
\usepackage{cancel}
\usepackage{gnuplottex}

% Number only referenced equations
\usepackage[fleqn]{mathtools}
\mathtoolsset{showonlyrefs}

%\usepackage{wrapfig}
\usepackage{siunitx}
\sisetup{locale = DE}

% New command for color underlining
\usepackage{xcolor}

\newsavebox\MBox
\newcommand\colul[2][red]{{\sbox\MBox{$#2$}%
  \rlap{\usebox\MBox}\color{#1}\rule[-1.2\dp\MBox]{\wd\MBox}{0.5pt}}}

% Stupid units for astronomy
\DeclareSIUnit\day{d}
\DeclareSIUnit\year{yr}
\DeclareSIUnit\au{AU}
% Astronomy-Style angles (hours, minutes, seconds)
\newcommand*{\ra}[2][]{{
  \def\SIUnitSymbolDegree{\textsuperscript{h}}%
  \def\SIUnitSymbolArcminute{\textsuperscript{m}}%
  \def\SIUnitSymbolArcsecond{\textsuperscript{s}}%
  \ang[#1]{#2}}%
}

\restylefloat{table}
\geometry{a4paper, top=15mm, left=10mm, right=20mm, bottom=20mm, headsep=10mm, footskip=12mm}
\linespread{1.2}
\setlength\parindent{0pt}
\DeclareMathOperator{\Tr}{Tr}
\DeclareMathOperator{\Var}{Var}

\begin{document}
\allowdisplaybreaks % Seitenumbrüche in Formeln erlauben
\begin{center}
\bfseries % Fettdruck einschalten
\sffamily % Serifenlose Schrift
\vspace{-40pt}
Einführung in die Astronomie und Astrophysik, Sommersemester 2014, Aufgabenblatt 5

Markus Fenske, Julia Schuch, Tutor: Daniel Härdt
\vspace{-10pt}
\end{center}
\section*{Aufgabe 1: Energiehaushalt eines Hot Jupiter}
\begin{enumerate}[a)]
  \item
    Aus der Grafik kann man 9 Perioden im Zeitraum von 80 Tagen ablesen. Daraus
    ergibt sich die Periode $T = \dfrac{80}{9}\,\mathrm{d} = 8{,}\overline{8}\,\mathrm{d}$.

    Aus dem 3. Keplerschen Gesetz erhalten wir
    % WA: ((gravitational constant) * (1.05*(mass of sun) + 2.96*(mass of
    % jupiter))*(80/9 days)^2/(4 pi^2))^(1/3) in astronomical units
    \begin{equation}
      a = \sqrt[3]{
        \frac{G(M+m)T^2}{4 \pi^2}
      } \approx \SI{0.0854}{AU}
    \end{equation}
  \item
    Die Strahlungsgleichgewichtstemperatur erhalten wir aus
    \begin{equation}
      T_\text{Pl} = \sqrt[4]{
        (1-A) \frac{F_\text{abs}}{F_\text{em}} \frac{R_\star^2}{a^2} T_\star^4
      }
    \end{equation}

    Dabei ist $F_\text{abs} = \pi R_\text{Pl}^2$ die absorbierende Fläche (also
    die Größe des Schattens des Planeten).

    %Aus der gegebenen Masse und der
    %Dichte erhalten wir den Radius, wenn wir den Planeten als kugelförmig
    %annehmen
    %\begin{equation}
    % V = \frac{M_\text{Pl}}{\rho_\text{Pl}} = \frac{4}{3} \pi R_\text{Pl}^3
    % \Leftrightarrow R_\text{Pl} = \sqrt[3]{\frac{3M}{4\pi \rho}}
    %\end{equation}

    \begin{enumerate}[(i)]
      \item
        Im Fall eines schnell rotierenden Planetens ist die emmitierende
        Fläche die gesamte Oberfläche $F_\text{em} = 4 \pi R_\text{Pl}^2$. Die
        Temperatur ist dann
        % WA: ((1-0.5)*1/4*(1.025 radius of sun)^2/(0.08543 AU)^2 (6050
        % kelvin)^4)^(1/4)
        \begin{equation}
          T_\text{Pl} = \sqrt[4]{
            \del{1-A} \frac{1}{4} \frac{R_\star^2}{a^2} T_\star^4
          } = \SI{849.6}{\kelvin}
        \end{equation}

      \item
        Im Fall eines langsam rotierenden Planeten halbiert sich die
        emmitierende Fläche.
        % WA: ((1-0.5)*1/2*(1.025 radius of sun)^2/(0.08543 AU)^2 (6050
        % kelvin)^4)^(1/4)
        \begin{equation}
          T_\text{Pl} = \sqrt[4]{
            \del{1-A} \frac{1}{2} \frac{R_\star^2}{a^2} T_\star^4
          } = \SI{1010}{\kelvin}
        \end{equation}

      %TODO: Zeug mit innerer Energiequelle!
    \end{enumerate}
  \item
    Da die direkte Beobachtung nicht möglich ist, werden zur Suche von
    Exoplaneten indirekte Entdeckungsmethoden angewandt. Zwei erfolgreiche
    Methoden nutzen dabei aus, dass sich ein System aus Stern und Planet um den
    gemeinsamen Schwerpunkt bewegt. Bei der Radialgeschwindigkeitsmethode misst
    man die Rot- und Blauverschiebung des Sternenspektrums durch die Bewegung
    vom Beobachter weg und auf den Beobachter zu (Dopplereffekt), um die
    Geschwindigkeitskomponente in Radialrichtung zu bestimmen. Bei der
    astrometrischen Methode misst man durch direkte Beobachtung über einen
    Zeitraum die Rotation des Sterns um den Schwerpunkt.

    Je massereicher der Stern ist, desto ausgeprägter sind die Effekte. Je
    kürzer seine Umlaufzeit ist, umso kleiner ist die Periode der Effekte.
    Beides erhöht die Erkennungswahrscheinlichkeit, so dass man bevorzugt
    Planeten mit größer Masse und geringen Umlaufzeiten entdeckt.
\end{enumerate}


\end{document}
