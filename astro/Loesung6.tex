\documentclass[a4paper,german,12pt,smallheadings]{scrartcl}
\usepackage[T1]{fontenc}
\usepackage[utf8]{inputenc}
\usepackage{babel}
\usepackage{tikz}
\usetikzlibrary{calc}
\usepackage{tkz-euclide}
\usetkzobj{all}
\usepackage{pgfplots}
\usepackage{geometry}
\usepackage[fleqn]{amsmath}
\usepackage{amssymb}
\usepackage{float}
\usepackage{lscape}
\usepackage{enumerate}
\usepackage{commath} % http://tex.stackexchange.com/questions/14821/whats-the-proper-way-to-typeset-a-differential-operator
\usepackage{cancel}
\usepackage{gnuplottex}

% Number only referenced equations
\usepackage[fleqn]{mathtools}
\mathtoolsset{showonlyrefs}

%\usepackage{wrapfig}
\usepackage{siunitx}
\sisetup{locale = DE}

% New command for color underlining
\usepackage{xcolor}

\newsavebox\MBox
\newcommand\colul[2][red]{{\sbox\MBox{$#2$}%
  \rlap{\usebox\MBox}\color{#1}\rule[-1.2\dp\MBox]{\wd\MBox}{0.5pt}}}

% Stupid units for astronomy
\DeclareSIUnit\day{d}
\DeclareSIUnit\year{yr}
\DeclareSIUnit\au{AU}
% Astronomy-Style angles (hours, minutes, seconds)
\newcommand*{\ra}[2][]{{
  \def\SIUnitSymbolDegree{\textsuperscript{h}}%
  \def\SIUnitSymbolArcminute{\textsuperscript{m}}%
  \def\SIUnitSymbolArcsecond{\textsuperscript{s}}%
  \ang[#1]{#2}}%
}

\restylefloat{table}
\geometry{a4paper, top=15mm, left=10mm, right=20mm, bottom=20mm, headsep=10mm, footskip=12mm}
\linespread{1.2}
\setlength\parindent{0pt}
\DeclareMathOperator{\Tr}{Tr}
\DeclareMathOperator{\Var}{Var}

\begin{document}
\allowdisplaybreaks % Seitenumbrüche in Formeln erlauben
\begin{center}
\bfseries % Fettdruck einschalten
\sffamily % Serifenlose Schrift
\vspace{-40pt}
Einführung in die Astronomie und Astrophysik, Sommersemester 2014, Aufgabenblatt 6

Markus Fenske, Julia Schuch, Tutor: Daniel Härdt
\vspace{-10pt}
\end{center}
\section*{Aufgabe 1: Klassifikation von Sternspektren aus Objektivprismenaufnahmen}
Im Tutorium bearbeitet.

\section*{Aufgabe 3: Roche-Flächen in Doppelsternsystem}
\begin{enumerate}[a)]
  \item
    Als Koordinatensystem wählen wir ein mitrotierendes
    Schwerpunktkoordinatensystem. Die Sterne befinden sich dann an den Orten
    % solve([d=M_A/(M_A+M_B)*x_A + M_B/(M_A+M_B)*x_B, x_A+x_B=d], [x_A, x_B]);
    \begin{equation}
      x_A = \frac{aq}{q-1}, x_B = -\frac{a}{q-1}
    \end{equation}

    Sie haben die Massen
    % solve([q=M_A/M_B, M=M_A+M_B], [M_A, M_B])
    \begin{equation}
      M_A = \frac{qM}{q+1}, M_B = \frac{M}{q+1}
    \end{equation}

    Aus den Gravitationskräften erhalten wir die Potentiale
    \begin{equation}
      \phi_{A,B} = -G \frac{M_{A,B}}{\envert{x - x_{A,B}}}
    \end{equation}

    Das System hat einen einen konstanten Abstand, rotiert also nach
    Keplerschem Gesetz mit konstanter Winkelgeschwindigkeit $\omega$. Daraus
    resultiert das Zentrifugalpotential
    \begin{equation}
      \phi_Z = \frac{\omega^2 x^2}{2}
    \end{equation}

    Die Kraft, die $M_A$ an der Stelle $x_B$ erzeugt, muss dabei genauso groß
    wie die Zentrifugalkraft sein. Also
    \begin{equation}
      \partial_x \del{\phi_A + \phi_Z} \sVert[2]_{\mathrlap{x=x_B}} \overset{!}{=} 0 \Rightarrow \omega = \sqrt{\frac{q(q-1)^3 GM}{a^3(q+1)^3}}
    \end{equation}

    Insgesamt ist das Potential also...

  \item
    %TODO: Skizze plotten
    Ein Materietausch kann stattfinden, sobald Materie des einen Sterns das
    Potentialmaximum überschreitet, denn dort kehrt die Kraft ihr Vorzeichen
    um. Einer der Sterne muss also größer sein, als der entsprechende Radius.

    Die Lagrangepunkte des Systems sind die Punkte an denen sich ein Körper
    im instabilen Gleichgewicht befindet. Sie befinden sich zwischen den
    Sternen und an diversen anderen Positionen \textbf{FIXME}.
  \item
    Nicht bearbeitet.
\end{enumerate}
\end{document}
