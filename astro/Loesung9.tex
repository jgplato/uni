\documentclass[a4paper,german,12pt,smallheadings]{scrartcl}
\usepackage[T1]{fontenc}
\usepackage[utf8]{inputenc}
\usepackage{babel}
\usepackage{tikz}
\usetikzlibrary{calc}
\usepackage{tkz-euclide}
\usetkzobj{all}
\usepackage{pgfplots}
\usepackage{geometry}
\usepackage{longtable}
\usepackage[fleqn]{amsmath}
\usepackage{amssymb}
\usepackage{float}
\usepackage{lscape}
\usepackage{enumerate}
\usepackage{commath} % http://tex.stackexchange.com/questions/14821/whats-the-proper-way-to-typeset-a-differential-operator
\usepackage{cancel}
\usepackage{gnuplottex}

% Number only referenced equations
\usepackage[fleqn]{mathtools}
\mathtoolsset{showonlyrefs}

%\usepackage{wrapfig}
\usepackage{siunitx}
\sisetup{locale = DE}

% New command for color underlining
\usepackage{xcolor}

\newsavebox\MBox
\newcommand\colul[2][red]{{\sbox\MBox{$#2$}%
  \rlap{\usebox\MBox}\color{#1}\rule[-1.2\dp\MBox]{\wd\MBox}{0.5pt}}}

% Stupid units for astronomy
\DeclareSIUnit\day{d}
\DeclareSIUnit\year{yr}
\DeclareSIUnit\au{AU}
% Astronomy-Style angles (hours, minutes, seconds)
\newcommand*{\ra}[2][]{{
  \def\SIUnitSymbolDegree{\textsuperscript{h}}%
  \def\SIUnitSymbolArcminute{\textsuperscript{m}}%
  \def\SIUnitSymbolArcsecond{\textsuperscript{s}}%
  \ang[#1]{#2}}%
}

\restylefloat{table}
\geometry{a4paper, top=15mm, left=10mm, right=20mm, bottom=20mm, headsep=10mm, footskip=12mm}
\linespread{1.2}
\setlength\parindent{0pt}
\DeclareMathOperator{\Tr}{Tr}
\DeclareMathOperator{\Var}{Var}

\begin{document}
\allowdisplaybreaks % Seitenumbrüche in Formeln erlauben
\begin{center}
\bfseries % Fettdruck einschalten
\sffamily % Serifenlose Schrift
\vspace{-40pt}
Einführung in die Astronomie und Astrophysik, Sommersemester 2014, Aufgabenblatt 9

Markus Fenske, Julia Schuch, Tutor: Daniel Härdt
\vspace{-10pt}
\end{center}
\section*{Aufgabe 1: Fleißaufgabe ohne Lerneffekt}
\begin{enumerate}[a)]
  \item
    Die kinetische Energie der Kugelsternhaufen ist (unter Vernachlässigung interner kinetischer Energie)
    \begin{equation}
      E_\text{kin} = \frac{1}{2} m v^2.
    \end{equation}

    Die potentielle Energie ist (unter Vernachlässigung jeglicher Interaktion
    untereinander und der Annahe einer sphärischen Milchstraße)
    \begin{equation}
      E_\text{pot} = \frac{GM(r)m}{r}
    \end{equation}

    Unter Nutzung des Virialsatzes
    \begin{equation}
      \left<E_\text{pot}\right> + 2 \left<E_\text{kin}\right> = 0
    \end{equation}

    und der Annahme, dass die momentanen Geschwindigkeiten ihren Mittelwerten
    entsprechen, erhalten wir
    \begin{equation}
      1 = -\frac{2\sum E_\text{kin}}{\sum E_\text{pot}} = \frac{\sum m_i v_i^2}{\sum \dfrac{GM m_i}{r_i}}
    \end{equation}

    Umstellen und Vernachlässigung der kinetischen Energie der Milchstraße
    ergibt
    \begin{equation}
      M = \frac{\sum m_i v_i^2}{G \sum \dfrac{m_i}{R_i}}
    \end{equation}

    Wenn wir annehmen, dass die Massen aller Kugelsternhaufen gleich sind und
    dass die Kugelsternhaufen sphärisch verteilt sind, so dass die
    Geschwindigkeitsquadrate durch die Radialgeschwindigkeiten ausgedrückt
    werden können: $v_i^2 = 3 u_i^2$, erhalten wir
    \begin{equation}
      M = \frac{3 \sum u_i^2}{G \sum \frac{1}{R_i}}
    \end{equation}
  \item Mit der Gleichung aus a) ergeben sich folgende Werte:

    \begin{longtable}{|l|l|l|l|l|l|}
    \hline
    $k$  & NGC Nr. &R [kpc] & $u$ [km/s] & $\sum_{i=1}^k R_i^{-1}$
    [$\mathrm{kpc}^-1$] & $M(k)$ [$M_\odot$] \\
     1 & 2419 & $\num{77.1}$ & -4   & $\num{0.0013}$ & $\num{1.8e9}$ \\
     2 & 5694 & $\num{45.3}$ & -281 & $\num{0.0036}$ & $\num{572e9}$ \\
     3 & 7006 & $\num{43.5}$ & -109 & $\num{0.0060}$ & $\num{092e9}$ \\
     4 & 2298 & $\num{26.5}$ & -187 & $\num{0.0100}$ & $\num{916e9}$ \\
     5 & 6229 & $\num{24.8}$ & 63   & $\num{0.0142}$ & $\num{665e9}$ \\
     6 & 4147 & $\num{24,1}$ & 1    & $\num{0.0186}$ & $\num{587e9}$ \\
     7 & 1904 & $\num{22.9}$ & 36   & $\num{0.0232}$ & $\num{475e9}$ \\
     8 & 5824 & $\num{20.9}$ & -156 & $\num{0.0282}$ & $\num{454e9}$ \\
     9 & 5024 & $\num{19.6}$ & -120 & $\num{0.0336}$ & $\num{413e9}$ \\
    10 & 1851 & $\num{18.3}$ & 76   & $\num{0.0393}$ & $\num{363e9}$ \\
    11 & 5634 & $\num{17.4}$ & -96  & $\num{0.0454}$ & $\num{330e9}$ \\
    12 & 6864 & $\num{15.4}$ & -122 & $\num{0.0522}$ & $\num{308e9}$ \\
    13 & 6981 & $\num{14.9}$ & -113 & $\num{0.0592}$ & $\num{287e9}$ \\
    14 & 6934 & $\num{13.9}$ & -146 & $\num{0.0668}$ & $\num{278e9}$ \\
    15 & 7089 & $\num{13.3}$ & 181  & $\num{0.0747}$ & $\num{280e9}$ \\
    16 & 6779 & $\num{12.9}$ & 98   & $\num{0.0828}$ & $\num{261e9}$ \\
    17 & 7078 & $\num{12.4}$ & 111  & $\num{0.0913}$ & $\num{247e9}$ \\
    18 & 5272 & $\num{12.0}$ & -100 & $\num{0.1000}$ & $\num{233e9}$ \\
    19 & 6681 & $\num{11.5}$ & 229  & $\num{0.1092}$ & $\num{248e9}$ \\
    20 & 6341 & $\num{11.1}$ & 103  & $\num{0.1186}$ & $\num{235e9}$ \\
    21 & 4590 & $\num{10.8}$ & -292 & $\num{0.1284}$ & $\num{266e9}$ \\
    22 & 7099 & $\num{10.0}$ & -72  & $\num{0.1389}$ & $\num{248e9}$ \\
    23 & 6205 & $\num{ 9.0}$ & -36  & $\num{0.1505}$ & $\num{230e9}$ \\
    24 & 6284 & $\num{ 8.6}$ & 37   & $\num{0.1628}$ & $\num{213e9}$ \\
    25 & 6652 & $\num{ 7.4}$ & -98  & $\num{0.1769}$ & $\num{200e9}$ \\
    26 & 5986 & $\num{ 7.2}$ & -79  & $\num{0.1915}$ & $\num{187e9}$ \\
    27 & 5904 & $\num{ 7.1}$ & 80   & $\num{0.2063}$ & $\num{176e9}$ \\
    28 & 6171 & $\num{ 6.7}$ & -109 & $\num{0.2220}$ & $\num{167e9}$ \\
    29 & 6440 & $\num{ 6.5}$ & -76  & $\num{0.2382}$ & $\num{158e9}$ \\
    30 & 6638 & $\num{ 6.2}$ & 46   & $\num{0.2551}$ & $\num{148e9}$ \\
    31 & 6656 & $\num{ 6.2}$ & -83  & $\num{0.2720}$ & $\num{140e9}$ \\
    32 & 6626 & $\num{ 5.2}$ & 57   & $\num{0.2922}$ & $\num{132e9}$ \\
    33 & 6218 & $\num{ 5.0}$ & 124  & $\num{0.3132}$ & $\num{126e9}$ \\
    34 & 6293 & $\num{ 5.0}$ & -63  & $\num{0.3342}$ & $\num{119e9}$ \\
    35 & 6254 & $\num{ 4.9}$ & 163  & $\num{0.3557}$ & $\num{117e9}$ \\
    36 & 6402 & $\num{ 4.4}$ & -12  & $\num{0.3795}$ & $\num{110e9}$ \\
    37 & 6712 & $\num{ 4.3}$ & 5    & $\num{0.4040}$ & $\num{103e9}$ \\
    38 & 6304 & $\num{ 4.2}$ & -93  & $\num{0.4290}$ & $\num{ 99e9}$ \\
    39 & 6723 & $\num{ 3.6}$ & 17   & $\num{0.4581}$ & $\num{ 93e9}$ \\
    40 & 6624 & $\num{ 3.5}$ & 101  & $\num{0.4882}$ & $\num{ 88e9}$ \\
    41 & 6093 & $\num{ 3.4}$ & 11   & $\num{0.5190}$ & $\num{ 83e9}$ \\
    42 & 6273 & $\num{ 3.4}$ & 112  & $\num{0.5499}$ & $\num{ 80e9}$ \\
    43 & 6637 & $\num{ 3.4}$ & 122  & $\num{0.5808}$ & $\num{ 78e9}$ \\
    44 & 6715 & $\num{ 3.4}$ & 151  & $\num{0.6117}$ & $\num{ 76e9}$ \\
    45 & 6333 & $\num{ 3.1}$ & 272  & $\num{0.6456}$ & $\num{ 81e9}$ \\
    46 & 6266 & $\num{ 2.9}$ & -90  & $\num{0.6818}$ & $\num{ 77e9}$ \\
    47 & 6356 & $\num{ 2.3}$ & 84   & $\num{0.7275}$ & $\num{ 73e9}$ \\
    48 & 6441 & $\num{ 2.3}$ & -80  & $\num{0.7731}$ & $\num{ 69e9}$ \\
    \hline
    \end{longtable}

  \item
    Siehe letzte Seite

\end{enumerate}
\section*{Aufgabe 3: Potentielle Energie einer Molekülwolke}
Die potentielle Energie ist die Energie, die ich erhalte, wenn ich die Wolke
zusammenbaue.

Angenommen es gäbe bereits eine kugelförmige Masse $M$ und ich hole aus
unendlicher Entfernung eine kleine Masse $dM$ die ich sphärisch um die Kugel
verteile, dann wird die potentielle Energie
\begin{equation}
  dV = - \frac{G M dM}{r}
\end{equation}
frei.

Wenn ich die Wolke aus solchen sphärischen Schichten der Dichte $\rho$ aufbaue,
erhalte ich jeweils eine zusätzliche Kugelschale der Dichte $\rho$ mit dem
Radius $dr$. Diese hat jeweils die Masse
\begin{equation}
  dM = 4 \pi r^2 \rho dr
\end{equation}

Angenommen, ich habe solche Schichten bis zum Radius $r$ gestapelt und die
Dichte sei konstant, dann ist die Masse
\begin{equation}
  M(r) = \rho V = \rho \frac{4}{3} \pi r^3
\end{equation}

Wenn wir über die einzelnen Schichten integrieren, erhalten wir
\begin{equation}
  V = -G \int_0^R \frac{M(r) dM(r)}{r} = -G \frac{16 \pi^2}{4} \rho^2 \int_0^R \dif \; r^4 = -G\frac{16 \pi^2 \rho^2}{4} \frac{1}{5} R^5
\end{equation}

Mit
\begin{equation}
  \rho = \frac{M}{V} = \frac{3 M}{4 \pi R^3}
\end{equation}

also
\begin{equation}
  V = -G \frac{16 \pi^2}{4} \frac{1}{5} R^5 \frac{9 M^2}{16\pi^2 R^6} = - \frac{3}{5} \frac{G M^2}{R}
\end{equation}

\newpage
\begin{landscape}
  % gnuplot ./data-9.gnuplot
  % GNUPLOT: LaTeX picture with Postscript
\begingroup
  \makeatletter
  \providecommand\color[2][]{%
    \GenericError{(gnuplot) \space\space\space\@spaces}{%
      Package color not loaded in conjunction with
      terminal option `colourtext'%
    }{See the gnuplot documentation for explanation.%
    }{Either use 'blacktext' in gnuplot or load the package
      color.sty in LaTeX.}%
    \renewcommand\color[2][]{}%
  }%
  \providecommand\includegraphics[2][]{%
    \GenericError{(gnuplot) \space\space\space\@spaces}{%
      Package graphicx or graphics not loaded%
    }{See the gnuplot documentation for explanation.%
    }{The gnuplot epslatex terminal needs graphicx.sty or graphics.sty.}%
    \renewcommand\includegraphics[2][]{}%
  }%
  \providecommand\rotatebox[2]{#2}%
  \@ifundefined{ifGPcolor}{%
    \newif\ifGPcolor
    \GPcolorfalse
  }{}%
  \@ifundefined{ifGPblacktext}{%
    \newif\ifGPblacktext
    \GPblacktexttrue
  }{}%
  % define a \g@addto@macro without @ in the name:
  \let\gplgaddtomacro\g@addto@macro
  % define empty templates for all commands taking text:
  \gdef\gplbacktext{}%
  \gdef\gplfronttext{}%
  \makeatother
  \ifGPblacktext
    % no textcolor at all
    \def\colorrgb#1{}%
    \def\colorgray#1{}%
  \else
    % gray or color?
    \ifGPcolor
      \def\colorrgb#1{\color[rgb]{#1}}%
      \def\colorgray#1{\color[gray]{#1}}%
      \expandafter\def\csname LTw\endcsname{\color{white}}%
      \expandafter\def\csname LTb\endcsname{\color{black}}%
      \expandafter\def\csname LTa\endcsname{\color{black}}%
      \expandafter\def\csname LT0\endcsname{\color[rgb]{1,0,0}}%
      \expandafter\def\csname LT1\endcsname{\color[rgb]{0,1,0}}%
      \expandafter\def\csname LT2\endcsname{\color[rgb]{0,0,1}}%
      \expandafter\def\csname LT3\endcsname{\color[rgb]{1,0,1}}%
      \expandafter\def\csname LT4\endcsname{\color[rgb]{0,1,1}}%
      \expandafter\def\csname LT5\endcsname{\color[rgb]{1,1,0}}%
      \expandafter\def\csname LT6\endcsname{\color[rgb]{0,0,0}}%
      \expandafter\def\csname LT7\endcsname{\color[rgb]{1,0.3,0}}%
      \expandafter\def\csname LT8\endcsname{\color[rgb]{0.5,0.5,0.5}}%
    \else
      % gray
      \def\colorrgb#1{\color{black}}%
      \def\colorgray#1{\color[gray]{#1}}%
      \expandafter\def\csname LTw\endcsname{\color{white}}%
      \expandafter\def\csname LTb\endcsname{\color{black}}%
      \expandafter\def\csname LTa\endcsname{\color{black}}%
      \expandafter\def\csname LT0\endcsname{\color{black}}%
      \expandafter\def\csname LT1\endcsname{\color{black}}%
      \expandafter\def\csname LT2\endcsname{\color{black}}%
      \expandafter\def\csname LT3\endcsname{\color{black}}%
      \expandafter\def\csname LT4\endcsname{\color{black}}%
      \expandafter\def\csname LT5\endcsname{\color{black}}%
      \expandafter\def\csname LT6\endcsname{\color{black}}%
      \expandafter\def\csname LT7\endcsname{\color{black}}%
      \expandafter\def\csname LT8\endcsname{\color{black}}%
    \fi
  \fi
  \setlength{\unitlength}{0.0500bp}%
  \begin{picture}(15306.00,10204.00)%
    \gplgaddtomacro\gplbacktext{%
      \csname LTb\endcsname%
      \put(1474,704){\makebox(0,0)[r]{\strut{} 0}}%
      \put(1474,1730){\makebox(0,0)[r]{\strut{} 5e+10}}%
      \put(1474,2756){\makebox(0,0)[r]{\strut{} 1e+11}}%
      \put(1474,3782){\makebox(0,0)[r]{\strut{} 1.5e+11}}%
      \put(1474,4808){\makebox(0,0)[r]{\strut{} 2e+11}}%
      \put(1474,5835){\makebox(0,0)[r]{\strut{} 2.5e+11}}%
      \put(1474,6861){\makebox(0,0)[r]{\strut{} 3e+11}}%
      \put(1474,7887){\makebox(0,0)[r]{\strut{} 3.5e+11}}%
      \put(1474,8913){\makebox(0,0)[r]{\strut{} 4e+11}}%
      \put(1474,9939){\makebox(0,0)[r]{\strut{} 4.5e+11}}%
      \put(1606,484){\makebox(0,0){\strut{} 0.05}}%
      \put(3269,484){\makebox(0,0){\strut{} 0.1}}%
      \put(4932,484){\makebox(0,0){\strut{} 0.15}}%
      \put(6595,484){\makebox(0,0){\strut{} 0.2}}%
      \put(8258,484){\makebox(0,0){\strut{} 0.25}}%
      \put(9920,484){\makebox(0,0){\strut{} 0.3}}%
      \put(11583,484){\makebox(0,0){\strut{} 0.35}}%
      \put(13246,484){\makebox(0,0){\strut{} 0.4}}%
      \put(14909,484){\makebox(0,0){\strut{} 0.45}}%
      \put(176,5321){\rotatebox{-270}{\makebox(0,0){\strut{}$M$ [$M_\odot$]}}}%
      \put(8257,154){\makebox(0,0){\strut{}$1/R$ [1/kpc]}}%
      \put(9588,6245){\makebox(0,0)[l]{\strut{}\textbf{Masse der Galaxis}}}%
      \put(9588,6025){\makebox(0,0)[l]{\strut{}}}%
      \put(9588,5805){\makebox(0,0)[l]{\strut{}Fitgleichung: $M(R) = \frac{a}{R}$}}%
      \put(9588,5585){\makebox(0,0)[l]{\strut{}}}%
      \put(9588,5365){\makebox(0,0)[l]{\strut{}$a = 2{,}146\pm0{,}037 \cdot 10^{10}$}}%
    }%
    \gplgaddtomacro\gplfronttext{%
    }%
    \gplbacktext
    \put(0,0){\includegraphics{plot-data-9}}%
    \gplfronttext
  \end{picture}%
\endgroup

\end{landscape}

\end{document}
