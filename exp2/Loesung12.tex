\documentclass[a4paper,german,12pt,smallheadings]{scrartcl}
\usepackage[T1]{fontenc}
\usepackage[utf8]{inputenc}
\usepackage{babel}
\usepackage{tikz}
\usepackage{geometry}
\usepackage{amsmath}
\usepackage{amssymb}
\usepackage{float}
\usepackage{cancel}
%\usepackage{wrapfig}
\usepackage[thinspace,thinqspace,squaren,textstyle]{SIunits}
\restylefloat{table}
\geometry{a4paper, top=15mm, left=20mm, right=40mm, bottom=20mm, headsep=10mm, footskip=12mm}
\linespread{1.5}
\setlength\parindent{0pt}
\begin{document}
\begin{center}
\bfseries % Fettdruck einschalten
\sffamily % Serifenlose Schrift
\vspace{-40pt}
Elektrodynamik und Optik, Sommersemester 2013, 12. Blatt \\
Markus Fenske, Tutor: Dr. Marko Wietstruk
\vspace{-10pt}
\end{center}
\section*{Aufgabe 5: Totalreflexion}

Skizze: Siehe Vorlesungsfolie vom 26.06, dritte Seite oder in so ziemlich jedem
Physikbuch.

Es gilt das Snelliussche Brechungsgesetz:

\begin{align*}
  \frac{\sin \alpha}{\sin \beta} = \frac{n_2}{n_1}
\end{align*}

Ab einem gewissen Einfallswinkel $\alpha$ wird der Ausfallswinkel $\beta$
größer als $\frac{\pi}{2}$. Ab diesem Punkt passiert eine Totalreflexion, da
der Strahl nicht mehr in das optisch dünnere Medium eintaucht. Wir setzen $\beta = \frac{\pi}{2}$ und lösen nach Einfallswinkel $\alpha$ auf.


\begin{align*}
  &\frac{\sin \alpha}{\sin \frac{\pi}{2}} = \frac{n_2}{n_1} \\
  \Leftrightarrow\quad&\frac{\sin \alpha}{1} = \frac{n_2}{n_1} \\
  \Leftrightarrow\quad&\alpha = \arcsin \frac{n_2}{n_1}
\end{align*}

Sobald also $\alpha > \arcsin \frac{n_2}{n_1}$ passiert Totalreflexion.

\end{document}
