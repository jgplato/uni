\documentclass[a4paper,german,12pt,smallheadings]{scrartcl}
\usepackage[T1]{fontenc}
\usepackage[utf8]{inputenc}
\usepackage{babel}
\usepackage{geometry}
\usepackage{pdfpages}
\usepackage{tikz}
\usepackage{wrapfig}
\usepackage[fleqn]{amsmath}
\usepackage{amssymb}
\usepackage{float}
\usepackage{enumerate}
\usepackage{listings} % Source code
\usepackage{lscape} % landscape
\usepackage{commath} % http://tex.stackexchange.com/questions/14821/whats-the-proper-way-to-typeset-a-differential-operator
\usepackage{cancel}
\usepackage[fleqn]{mathtools}
% Number only referenced equations
%\mathtoolsset{showonlyrefs}

%\usepackage{wrapfig}
\usepackage{siunitx}
\sisetup{separate-uncertainty=true,locale=DE}

% http://tex.stackexchange.com/questions/38818/best-way-to-denote-an-angle-in-tikz
\newcommand\markangle[6][red]{% [color] {X} {origin} {Y} {mark} {radius}
  % filled circle: red by default
  \begin{scope}
    \path[clip] (#2) -- (#3) -- (#4);
    \fill[color=#1,fill opacity=0.5,draw=#1,name path=circle]
    (#3) circle (#6mm);
  \end{scope}
  % middle calculation
  \path[name path=line one] (#3) -- (#2);
  \path[name path=line two] (#3) -- (#4);
  \path[%
  name intersections={of=line one and circle, by={inter one}},
  name intersections={of=line two and circle, by={inter two}}
  ] (inter one) -- (inter two) coordinate[pos=.5] (middle);
  % bissectrice definition
  \path[%
  name path=bissectrice
  ] (#3) -- (barycentric cs:#3=-1,middle=1.2);
  % put mark
  \path[
  name intersections={of=bissectrice and circle, by={middleArc}}
  ] (#3) -- (middleArc) node[pos=1.3] {#5};
  }

% New command for color underlining
\usepackage{xcolor}
\newcommand\invisiblesection[1]{%
    \refstepcounter{section}%
      \addcontentsline{toc}{section}{\protect\numberline{\thesection}#1}%
        \sectionmark{#1}}
\newsavebox\MBox
\newcommand\colul[2][red]{{\sbox\MBox{$#2$}%
  \rlap{\usebox\MBox}\color{#1}\rule[-1.2\dp\MBox]{\wd\MBox}{0.5pt}}}

\restylefloat{table}
\geometry{a4paper, top=15mm, left=20mm, right=10mm, bottom=20mm, headsep=10mm, footskip=12mm}
\linespread{1.5}
\setlength\parindent{0pt}
\DeclareMathOperator{\Tr}{Tr}
\DeclareMathOperator{\Var}{Var}
\begin{document}

\begin{titlepage}
\newcommand{\HRule}{\rule{\linewidth}{0.5mm}}

\begin{center}
  \textsc{\Large Physikalisches Grundpraktkum II}
  \HRule\\[0.4 cm]
  {\huge \bfseries Optische Spektroskopie}
  \HRule\\[0.4 cm]

  \begin{minipage}{0.60\textwidth}
  \begin{flushleft}
    Markus Fenske \texttt{<iblue@zedat.fu-berlin.de>} \\
    Alexandra Krause \texttt{<alexandra.krause2@gmail.com>}
  \end{flushleft}
  \end{minipage}
  \hfill
  \begin{minipage}{0.35\textwidth}
  \begin{flushright}
    Tutor: Madsen \\
    Versuchstag: 19. November 2014
  \end{flushright}
  \end{minipage}

  \vspace{1cm}

  \tableofcontents


  %{\large \today}
  \vfill
\end{center}
\newpage

\end{titlepage}

\allowdisplaybreaks % Seitenumbrüche in Formeln erlauben
\begin{center}
\bfseries % Fettdruck einschalten
\sffamily % Serifenlose Schrift
\vspace{-40pt}
Physikalisches Grundpraktikum 2, Wintersemester 2014/2015

Markus Fenske \texttt{<iblue@zedat.fu-berlin.de>}

Alexandra Krause \texttt{<alexandra.krause2@gmail.com>}

Fabry-Perot-Etalon
\vspace{-10pt}
\end{center}

\section{Physikalische Grundlagen}

Ein Fabry-Perot-Etalon besteht schematisch gesehen aus einem optisch
durchlässigen Medium, das von zwei parallelen teilverspiegelten Grenzflächen
eingeschlossen wird. Wenn ein Strahl unter dem Winkel $\alpha$ auf das Etalon
trifft, wird er zwischen den Spiegeln hin und her reflektiert, wobei ein Anteil
das Etalon verlässt.

\begin{figure}[h]
  \centering
  \begin{tikzpicture}
    \pgfmathsetmacro{\angle}{35}
    \pgfmathsetmacro{\distance}{4}
    \pgfmathsetmacro{\width}{15}
    \pgfmathsetmacro{\rayAt}{1}
    \pgfmathsetmacro{\reflectionWidth}{\distance*tan(\angle)}

    % Koordinaten
    % FIXME: Labels
    \coordinate                       (RayStart)        at
      (-\width/2+\rayAt-    \reflectionWidth/2, -\distance   );
    \coordinate                       (RayIn)           at
      (-\width/2+\rayAt,                        -\distance/2 );
    \coordinate[label=above left:$A$] (RayBounceTop1)   at
      (-\width/2+\rayAt+     \reflectionWidth,   \distance/2 );
    \coordinate                       (RayBounceBottom) at
      (-\width/2+\rayAt+ 2  *\reflectionWidth,  -\distance/2 );
    \coordinate                       (RayBounceTop2)   at
      (-\width/2+\rayAt+ 3  *\reflectionWidth,   \distance/2 );
    \coordinate                       (RayEnd)          at
      (-\width/2+\rayAt+ 3.5*\reflectionWidth,   0           );
    \coordinate                       (RayTrans1)          at
      (-\width/2+\rayAt +1.5*\reflectionWidth,   \distance   );
    \coordinate                       (RayTrans2)          at
      (-\width/2+\rayAt +3.5*\reflectionWidth,   \distance   );

    % Oberer und unterer Spiegel
    \draw (-\width/2,\distance/2)  -- (\width/2,\distance/2);
    \draw (-\width/2,-\distance/2) -- (\width/2,-\distance/2);

    % Strahl
    \draw (RayStart) -- (RayIn) -- (RayBounceTop1) -- (RayBounceBottom) 
    -- (RayBounceTop2) -- (RayEnd); % FIXME: Fadein, Fadeout

    % Outputstrahlen
    \draw (RayBounceTop1) -- (RayTrans1);
    \draw (RayBounceTop2) -- (RayTrans2);

    % Gestrichelte Linie Einfallswinkel und Einfallswinkel
    \draw[dashed] (-\width/2+\rayAt,-\distance/2) -- (-\width/2+\rayAt,-\distance);
    %\markangle[green]{(-\width/2+\rayAt-\reflectionWidth/2,-\distance)}{(-\width/2+\rayAt,-\distance/2)}{(-\width/2+\rayAt,-\distance)}{$\gamma$}{12}

    % Winkel



  \end{tikzpicture}
  \caption{Strahlengang im Etalon}
  \label{fig:rays}
\end{figure}






\end{document}
