\documentclass[a4paper,german,12pt,smallheadings]{scrartcl}
\usepackage[T1]{fontenc}
\usepackage[utf8]{inputenc}
\usepackage{babel}
\usepackage{geometry}
\usepackage{pdfpages}
\usepackage{tikz}
\usepackage{wrapfig}
\usepackage[fleqn]{amsmath}
\usepackage{amssymb}
\usepackage{float}
\usepackage{enumerate}
\usepackage{listings} % Source code
\usepackage{lscape} % landscape
\usepackage{commath} % http://tex.stackexchange.com/questions/14821/whats-the-proper-way-to-typeset-a-differential-operator
\usepackage{cancel}
\usepackage[fleqn]{mathtools}
% Number only referenced equations
%\mathtoolsset{showonlyrefs}

%\usepackage{wrapfig}
\usepackage{siunitx}
\sisetup{separate-uncertainty=true,locale=DE}

% New command for color underlining
\usepackage{xcolor}
\newcommand\invisiblesection[1]{%
    \refstepcounter{section}%
      \addcontentsline{toc}{section}{\protect\numberline{\thesection}#1}%
        \sectionmark{#1}}
\newsavebox\MBox
\newcommand\colul[2][red]{{\sbox\MBox{$#2$}%
  \rlap{\usebox\MBox}\color{#1}\rule[-1.2\dp\MBox]{\wd\MBox}{0.5pt}}}

\restylefloat{table}
\geometry{a4paper, top=15mm, left=20mm, right=10mm, bottom=20mm, headsep=10mm, footskip=12mm}
\linespread{1.5}
\setlength\parindent{0pt}
\DeclareMathOperator{\Tr}{Tr}
\DeclareMathOperator{\Var}{Var}
\begin{document}

\begin{titlepage}
\newcommand{\HRule}{\rule{\linewidth}{0.5mm}}

\begin{center}
  \textsc{\Large Physikalisches Grundpraktkum II}
  \HRule\\[0.4 cm]
  {\huge \bfseries Optische Spektroskopie}
  \HRule\\[0.4 cm]

  \begin{minipage}{0.60\textwidth}
  \begin{flushleft}
    Markus Fenske \texttt{<iblue@zedat.fu-berlin.de>} \\
    Alexandra Krause \texttt{<alexandra.krause2@gmail.com>}
  \end{flushleft}
  \end{minipage}
  \hfill
  \begin{minipage}{0.35\textwidth}
  \begin{flushright}
    Tutor: Madsen \\
    Versuchstag: 19. November 2014
  \end{flushright}
  \end{minipage}

  \vspace{1cm}

  \tableofcontents


  %{\large \today}
  \vfill
\end{center}
\newpage

\end{titlepage}

\allowdisplaybreaks % Seitenumbrüche in Formeln erlauben
\begin{center}
\bfseries % Fettdruck einschalten
\sffamily % Serifenlose Schrift
\vspace{-40pt}
Physikalisches Grundpraktikum 2, Wintersemester 2014/2015

Markus Fenske \texttt{<iblue@zedat.fu-berlin.de>}

Alexandra Krause \texttt{<alexandra.krause2@gmail.com>}

Optische Spektroskopie, Tutor: Kevin Madsen
\vspace{-10pt}
\end{center}

\section{Physikalische Grundlagen}

Untersucht werden soll die spektrale Zerlegung von Licht an einem Prisma und
einem Beugungsgitter.

\subsection{Brechung am Prisma}

Wechselt ein Lichtstrahl das Medium, gilt

\begin{equation}
  n_1 \sin(\delta_1) = n_2 \sin(\delta_2)
\end{equation}

dabei sind $n_1$, $n_2$ die Brechungsindizes der jeweiligen Medien, $\delta_1$
der Einfallswinkel und $\delta_2$ der Winkel des gebrochenen Strahls. Somit
tritt eine Ablenkung am Wechsel des Mediums auf.

Der Brechungsindex eines optischen Mediums ist dabei nicht nur vom Medium,
sondern auch von der Wellenlänge des Lichts abhängig, somit

\begin{equation}
  n_1(\lambda) \sin(\delta_1) = n_2(\lambda) \sin(\delta_2)
\end{equation}

\subsection{Prisma}

Ein Prisma ist ein durchsichtiges optisches Bauteil in Form eines Prismas, das
zur Spektralzerlegung und Ablenkung von Licht benutzt werden kann.

\begin{figure}[h!]
    \centering
    \includegraphics[width=0.8\textwidth]{prisma.png}
    \caption{Prisma}
    \label{fig:prisma}
\end{figure}

Für den Sonderfall, dass ein Lichtstrahl das Prisma parallel zu dessen Basis
durchquert, gilt, dass, wenn $\epsilon$ der Winkel an der Spitze des Prismas ist und
$\alpha$ der Einfallswinkel zum Lot und $\beta$ der Austrittswinkel zum Lot,
sich der gesamte Brechwinkel $\gamma$ durch

\begin{equation}
  \gamma = 2 \del{\alpha - \beta}
\end{equation}

ergibt.

Die beiden Austrittswinkel $\beta$ spannen zusammen mit dem dritten Winkel
$\delta$ ein Dreieck auf. Wegen Innenwinkelsatz gilt $\delta = 180^\circ -
\epsilon$. Aufgrund der Winkelsumme im Dreieck ist somit

\begin{align*}
  2 \beta + \delta &= 180^\circ \\
  2 \beta + 180^\circ - \epsilon = 180^\circ \\
  \beta = \frac{\epsilon}{2}
\end{align*}

Durch Einsetzen der beiden Gleichungen ineinander erhält man

\begin{equation}
  \alpha = \frac{\gamma + \epsilon}{2}
\end{equation}

\subsection{Auflösungsvermögen des Prismas}

Um getrennte Spektrallinien zu erhalten, wird das Licht durch einen Einzelspalt
geschickt. Dabei treten Beugungserscheinungen auf. Linien können erst dann als
getrennt betrachtet werden, wenn der Abstand derart ist, dass das
Beugungsmaximum der einen Linie mit dem ersten Beugungsminimum der anderen
Linie zusammenfällt (Rayleigh-Kriterium).

Um dies quantitativ zu bestimmen, entwickeln wir den Brechnungsindex in einer
Taylorreihe linearer Ordnung:

\begin{equation}
  n(\lambda + \Delta \lambda) \approx n + \frac{\dif n}{\dif \lambda} \Delta \lambda
\end{equation}

Läuft nun gemischtes Licht der Wellenlängen $\lambda$ und $\lambda + \Delta
\lambda$ parallel zur Basis auf einer Länge $t$ durch das Prisma, entsteht ein
Gangunterschied

\begin{align}
  n(\lambda_1) t - n(\lambda_1 + \Delta \lambda) t &= \lambda \\
  \del{ n + \frac{\dif n}{\dif \lambda} \Delta \lambda}t - nt &= \lambda
\end{align}

\begin{equation}
  \lambda = t \Delta \lambda \frac{\dif n}{\dif \lambda}
\end{equation}

Das Auflösungsvermögen eines Prismas bestimmt sich also durch die
materialabhängige Dispersion, die Wellenlänge des Lichts und insbesondere durch
die Basislänge $t$.

\subsection{Beugungsgitter}

Ein optisches Gitter lässt sich modellhaft betrachten als eine Blende mit einer
periodischen Folge von lichtdurchlässigen Spalten. Aufgrund des
Wellencharakters des Lichts entstehen beim Durchgang kohärenten Lichtes durch
das Gitter dabei Interferenzeffekte. Zu beobachten sind scharfe Beugungsmaxima
in Abständen die von der jeweiligen Wellenlänge des Lichts abhängig sind. Somit
lässt sich auch ein Beugungsgitter zur Spektralzerlegung benutzen.

Die Hauptmaxima lassen sich dabei aus der Bedingung herleiten, dass für
konstruktive Interferrenz der Gangunterschied benachbarter Spalte ein
ganzzahliges Vielfaches der Wellenlänge betragen muss. Für eine Gitterkonstante
$d$ und die Wellenlänge $\lambda$ gilt für den Winkel $\alpha$:

\begin{equation}
  d \sin \alpha = z \lambda \quad \text{mit} \quad z \in \mathbf{N}
  \label{eq:1}
\end{equation}

$z$ wird dabei als Ordnung der Beugungsmaxima bezeichnet.

\subsection{Auflösungsvermögen des Beugungsgitters}

Neben den Hauptmaxima existieren noch Nebenmaxima, deren Intensität jedoch
schnell abnimmt, so dass wir nur das erste Nebenmaximum betrachten. Die Lage
des ersten Nebenminimums ist gegeben durch

\begin{equation}
  d \sin \alpha_\text{min} = \del{1 + \frac{1}{N}} \lambda
\end{equation}

wobei $N$ die Gesamtzahl der beitragenden Gitterspalte ist.

Gemäß Rayleigh-Kriterium (siehe Prisma) ergibt sich dann die auflösbare
Wellenlängendifferenz

\begin{equation}
  \Delta \lambda = \frac{\lambda}{z N}
\end{equation}

Das Auflösungsvermögen steigt also mit wachsender Spaltzahl und steigender Ordnung.

\subsection{Wasserstoffspektrum}

Aus dem Bohrschen Atommodell lässt sich das Wasserstoffspektrum ableiten als

\begin{equation}
  \frac{1}{\lambda} = R \del{ \frac{1}{m^2} - \frac{1}{n^2}}
  \quad \text{mit} \quad m,n \in \mathbf{N}
\end{equation}

Dabei ist
\begin{equation}
  R = \frac{2 \pi^2 m_e e^4}{h^3 c}
\end{equation}
die Rydberg-Konstante.

Für $m = 2$ erhält man die Balmer-Serie.

\newpage
\section{Aufgaben}
Da wir die Messungen am Beugungsgitter durchführen, geben wir nur diese
Aufgaben wieder.

\begin{enumerate}
  \item Aufbau und Justage des Spektrometers
  \item Aufnahme des Spektrums einer Quecksilberdampflampe in der ersten und
    zweiten Ordnung und Bestimmung der Gitterkonstanten
  \item Spektroskopie einer unbekannten Lampe und Analyse der Lampenfüllung
  \item Bestimmung des Auflösungsvermögens des Gitters in erster und zweiter
    Ordnung und Vergleich mit den theoretischen Erwartungen
  \item Qualitative Beobachtung und Diskussion des Dispersionsspektrums eines
    Prismas.
\end{enumerate}

\section{Experimenteller Aufbau}

Die Experimente wurden am Beugungsgitter durchgeführt. Der Versuchsaufbau
besteht aus einem festen Teil mit:

\begin{itemize}
  \item Spektrallampe (Hg-Dampflampe bzw. unbekannte Spektrallampe Nr. 8)
  \item Kollimatorlinse
  \item Größenverstellbarer Eintrittsspalt
  \item Beugungsgitter mit unbekannter Gitterkonstante
\end{itemize}

Sowie drehbar gelagert und mit einem Goniometer versehen:

\begin{itemize}
  \item Objektivlinse
  \item Okular mit Fadenkreuz
\end{itemize}

Der Aufbau lässt sich der Abbilung \ref{fig:aufbau} entnehmen.

\begin{figure}[h!]
    \centering
    \includegraphics[width=0.8\textwidth]{aufbau.png}
    \caption{Versuchsaufbau}
    \label{fig:aufbau}
\end{figure}

\section{Durchführung}
\subsection{Aufgabe 1}

Nach dem Einschalten der Quecksilberdampflampe haben wir den Aufbau
entsprechend des Skriptes durch Autokollimation justiert. Um größtmögliche
Messgenauigkeit zu erreichen, haben wir den Einlassspalt auf den
kleinstmöglichen Wert eingestellt.

\subsection{Aufgabe 2}

Entsprechend der Aufgabenstellung wurden die Winkel sowie Farbe und subjektive
Intensität aller sichtbaren Spektrallinien 1. Ordnung aufgenommen. Die Winkel
der Spektrallinien 2. Ordnung wurden nur für die hellsten Linien aufgenommen.
Zur Kontrolle und zur Messung Bestimmung des Winkelnullpunkts wurden diese
Messungen jeweils auf beiden Seiten durchgeführt. Außerdem wurde zur Kontrolle
des Nullpunkts die Linie 0. Ordnung aufgenommen.

\subsection{Aufgabe 3}

Die Lampe wurde gegen eine unbekannte Spektrallampe ausgetauscht. Die Messungen
aus Aufgabe 2 wurden an dieser Spektrallampe wiederholt.

\subsection{Aufgabe 4}

Ein zusätzlicher einstellbarer Spalt wurde hinter dem Kollimator und direkt vor
dem Beugungsgitter eingesetzt. Der Spalt wurde so justiert, dass das
gelb-orange $579{,}1$ / $577{,}0 \operatorname{nm}$ Linienpaar gerade noch im Sinne
des Raygleigh-Kriteriums unterscheidbar war. Die Millimeterschraube des Spaltes
wurde am Nullpunkt und am eingestellten Punkt abgelesen, um aus der Differenz
die Spaltbreite zu ermitteln.

\subsection{Aufgabe 5}

Das Beugungsgitter wurde durch ein Prisma ersetzt und die qualitativen
Beobachtungen notiert.
\newpage
\invisiblesection{Messwerte}
\includepdf{mess-0001.pdf}
\includepdf{mess-0002.pdf}
\includepdf{mess-0003.pdf}
\invisiblesection{Tabellen zur Auswertung}
\includepdf[landscape=true]{hg-lampe.pdf}
\includepdf[landscape=true]{zn-lampe.pdf}
\section{Auswertungen}
\subsection{Aufgabe 1}

Die Justage wurde durch Autokollimation und Scharfstellen des Okkulars wie im
Skript beschrieben durchgeführt.

\subsection{Aufgabe 2}
Zur Bestimmung des Nullpunktes haben wir zuerst die 4 prägnantesten Linien
(Violette Hauptlinie, grüne Hauptlinie und gelbe Doppellinie) jeweils ihren
Entsprechungen in den unterschiedlichen Ordnungen anhand von Intensität und
Farbe zugeordnet. Bedeutet konkret: Die violette Hauptlinie der 1. Ordnung
wurde der gegenüberliegenden violetten Hauptlinie der 1. Ordnung zugeordnet,
usw.

Da genug Werte gemessen wurden, benutzen wir in dieser Aufgabe die
Fehlerschätzung aus der Streuung. Dabei ist der Mittelwert der Werte $x_1,
\dots, x_n$ definiert als

\begin{equation}
  \overline{x} = \frac{1}{n} \sum_i x_i
\end{equation}

Die Standardabweichung schätzen wir durch

\begin{equation}
  \sigma = \sqrt{\frac{1}{n-1} \sum_i (x_i - \overline{x})^2}
\end{equation}

Durch Berechnung der Mittelwerte der jeweils 2 Werte erhalten wir 8 Nullpunkte.
Durch die Berechnung des Mittelwerts und der Standardabweichung der 8 Werte
erhalten wir den Nullpunkt:

\begin{equation*}
  \alpha_0 = (177{,}98\pm0{,}91)^\circ
\end{equation*}

Dieser ist verträglich mit dem Nullpunkt der durch den Winkel der 0. Ordnung
bestimmt wurde.

Durch Differenzbildung zu diesem Nullpunkt haben wir dann jeweils den
Ablenkwinkel $\alpha$ ermittelt. Da der ermittelte Fehler des Nullpunktes den
Ablesefehler um mehr als eine Größenordnung übersteigt, kann der Ablesefehler
unberücksichtigt bleiben. Der Fehler der Winkel ist somit identisch zum aus der
Streuung ermittelten Fehler.

\begin{equation*}
  \Delta \alpha = 0{,}91^\circ
\end{equation*}

Die ermittelten Werte finden sich in der anliegenden Tabelle.

Die anderen Linien konnten auch durch einen Vergleich mit den Differenzen nicht
eindeutig identifiziert werden, so dass sie für die Ermittlung der
Gitterkonstanten unberücksichtigt bleiben müssen.

Anhand des im Skript enthaltenen Spektrums haben wir den identifizierbaren
Spektrallinien Wellenlängen zugeordnet. Anhand von Gleichung \ref{eq:1} haben
wir dann für jeden Wert eine Gitterkonstante bestimmt.

Anstatt die Gaußßsche Fehlerfortpflanzung zu benutzen, erschien es uns
sinnvoll, da der Fehler des Ablesewinkels ja sowieso durch Streuung bestimmt
wurde, auch den Fehler der Gitterkonstanten aus der Streung zu ermitteln.

Es ergibt sich somit als Zwischenergebnis eine Gitterkonstante von

\begin{equation*}
  d = (1{,}658\pm0{,}055) \operatorname{\mu m}
\end{equation*}

Eine kurze Abschätzung in der Tabellenkalkulation zeigte, dass die Genauigkeit
der Gitterkonstanten nicht ausreichend ist, um auf die Wellenlängen
zurückzurechnen. Die anderen Linien bleiben somit für uns leider nicht
eindeutig identifizierbar.

\subsection{Aufgabe 3}

Nachdem wir die Spektrallinien untereinander zugeordnet haben, haben wir unter
Benutzung der in Aufgabe 2 ermittelten Gitterkonstante die Wellenlänge
bestimmt. Für jede Spektrallinie ergeben sich 4 Wellenlängen. Die tatsächliche
Wellenlänge und den Fehler erhalten wir durch Mittelwertbildung und Berechnung
der Standardabweichung. Gaußsche Fehlerfortplanzung wäre in diesem Fall
möglich, aber da das Ergebnis eindeutig ist, nicht nötig.

Die ersten 3 hellen Spektrallinien lassen sich gut mit den $467{,}8
\operatorname{nm}$, $480{,}0 \operatorname{nm}$ und $508{,}6 \operatorname{nm}$
Linien identifizieren.

Die vierte Linie konnten wir leider in der einen zweiten Ordnung nicht finden,
so dass sich wohl ein systematischer Fehler ergibt, der durch die
Mittelwertbildung nicht aufgehoben wird. Wir können daher nur vermuten, dass es
sich dabei um die hellste Linie handelt, nämlich die $643{,}8
\operatorname{nm}$-Linie.

Da Helium nicht in Frage kommt und die Linien von Zink viel dichter liegen und
daher mit den farblichen Beobachtung nicht übereinstimmen, folgern wir, dass es
sich um eine \textbf{Cadmium-Spektrallampe} handeln muss.

\subsection{Aufgabe 4}

Wird die beleuchtete Breite eingeschränkt, sinkt die Anzahl der beleuchteten
Gitterspalte $N$. Da die Gitterkonstante den Abstand zwischen den Spalten
angibt, ist das Inverse $\frac{1}{d}$ die Anzahl der Spalte pro Breite $b$.
Somit ist die Anzahl der beleuchteten Spalte

\begin{equation}
  N = \frac{b}{d}
\end{equation}

Dabei ist b die Differenz der beiden gemessenen Werte. Der Fehler ergibt sich
durch Gaußsche Fehlerfortpflanzung als $\Delta b = \sqrt{2} \Delta b_1$.

\begin{equation}
  b = (3{,}08 \pm 0{,}01) \operatorname{mm} - (2{,}78\pm0{,}01) \operatorname{mm}
    = (0{,}300 \pm 0{,}015) \operatorname{mm}
\end{equation}

Der Fehler von $N$ ist

\begin{align*}
  \Delta N &= \sqrt{
    \del{\frac{\partial N}{\partial b}}^2 \Delta b^2 +
    \del{\frac{\partial N}{\partial d}}^2 \Delta d^2
  } \\ &= \sqrt{
    \frac{1}{d^2} \Delta b^2 +
    \frac{b^2}{d^4} \Delta d^2
  } \\ & = \sqrt{
    \del{\frac{0{,}015 \operatorname{mm}}{1{,}658 \operatorname{\mu m}}}^2 +
    \frac{(0{,}300 \operatorname{mm})^2 (0{,}055 \operatorname{\mu m})^2}{
      (1{,}658 \operatorname{\mu m})^4
    }
  } \approx 10{,}9
\end{align*}


Somit ist
\begin{equation}
  N = \frac{(0{,}300 \pm 0{,}015) \operatorname{mm}}{(1{,}658\pm0{,}055) \operatorname{\mu m}}
  = 181\pm11
\end{equation}

Experimentell ergibt sich also ein Auflösungsvermögen in der ersten Ordnung v

\begin{equation}
  \frac{\lambda}{\Delta \lambda} = z N = 180\pm20
\end{equation}

Theoretisch ergibt sich hingegen

\begin{equation}
  \frac{\lambda}{\Delta \lambda} = \frac{579{,}1 \operatorname{nm}}{579{,}1 \operatorname{nm} - 577{,}0 \operatorname{nm}} = 275{,}76
\end{equation}

Die Ergebnisse sind signifikant verschieden. Die Ursache wird im Teil
\textit{Auswertung} diskutiert.

\subsection{Aufgabe 5}
Beim Ersetzen des Beugungsgitters durch das Prisma ergeben sich drei
Beobachtungen:

\begin{itemize}
  \item Es gibt nur einen Satz Linien (also keine Linien unterschiedlicher
    Ordnung) auf nur einer Seite
  \item Die Spektrallinien befinden sich nur auf einer Seite.
  \item Die Farben sind anders herum geordnet
\end{itemize}

Die ersten beiden Punkte lassen sich dadurch erklären, dass die Funktionsweise
des Beugungsgitters auf Interferrenzeffekten beruht. Das Prisma hingegen
zerlegt einen Lichtstrahl durch Brechung und Dispersion in seine Bestandteile.
Es gibt also nur ein Spektrum und nur auf einer Seite.

Dass die Linien anders herrum geordnet sind liegt am positiven
Dispersionskoeffizienten des verwendeten Prismas. Wie bei den meisten Stoffen
wird Licht kurzer Wellenlänge dabei stärker gebrochen als Licht längerer
Wellenlänge. Ist der Dispersionskoeffizient negativ, spricht man daher von
\textit{anomaler Dispersion}.

\section{Auswertung und Diskussion}

Wir konnten die Gitterkonstante erfolgreich als
\begin{equation*}
  d = (1{,}66\pm0{,}06) \operatorname{\mu m}
\end{equation*}
bestimmen.

Auch konnten wir die Art der Lampe sicher bestimmen.

Die Annahme der hohe Messgenauigkeit von $\Delta \alpha = 0{,}01^\circ$ konnten
wir anhand der Fehlerbestimmung aus der Streuung als falsch erkennen.
Dementsprechend stellte es sich auch herraus, dass die Einstellung des Spaltes
auf kleinstmögliche Breite kontraproduktiv war. Es wäre besser gewesen, den
Spalt breiter einzustellen, um auch dunklere Spektrallinien zu finden.

Das experimentell bestimmte Auflösungsvermögen des Gitters
\begin{equation}
  z N = 180\pm20
\end{equation}

weicht signifikant vom theoretischen Wert
\begin{equation}
  \frac{\lambda}{\Delta \lambda} = 275{,}76
\end{equation}

ab. Es gehen die Gitterkonstante und die Spaltbreite in die Messgleichung ein.
Da die Gitterkonstante benutzt wurde, um die Wellenlängen zu berechnen und sich
dabei keine signifikanten Abweichungen zu den Erwartungen ergaben, ist davon
auszugehen, dass die Spaltbreite falsch bestimmt wurde. Das Problem liegt dabei
sicherlich in der Bestimmung des Nullpunkts. Der Nullpunkt wurde dort
festgestellt, wo sich ein Gegendruck der Rückstellfeder bemerkbar macht.  Diese
subjektive Art der Messung führt zu einem Fehler, der viel größer ist, als
angenommen.

Die qualitativen Untersuchungen am Prisma konnten die Unterschiede zum
Beugungsgitter aufzeigen.

Insgesamt konnten die Messdaten trotz Fehleinstellungen und Fehlannahmen
produktiv verwertet werden, so dass wir das Experiment als erfolgreich
betrachten.

\section{Weiterführende Überlegungen}
Das im Skript abgebildete Spektrum der Cadmium-Lampe scheint nicht mit dem
Experiment übereinzustimmen. Obwohl die Linien bei $609{,}9 \operatorname{nm}$
und $611{,}2 \operatorname{nm}$ in einer Niederdruckdampflampe überhaupt nicht
sichtbar sind\footnote{Siehe
http://de.wikipedia.org/wiki/Datei:Cd\_Niederdruck\_Spektrum.png}, werden diese
im Skript als starke Linien dargestellt. Dadurch ist es leicht mit dem Spektrum
der Zink-Lampe verwechselbar.

Zur Verbesserung der Messergebnisse wäre es sicherlich sinnvoll, die
Spaltbreite mit anderen Mitteln zu messen. Eine Mikroschieblehre könnte die
Genauigkeit signifikant verbessern.

Um die dunkleren Spektrallinien besser zu messen wäre es außerdem sinnvoll
gewesen, den Spalt auf Kosten der Genauigkeit breiter einzustellen.
\end{document}
