\documentclass[a4paper,german,12pt,smallheadings]{scrartcl}
\usepackage[T1]{fontenc}
\usepackage[utf8]{inputenc}
\usepackage{babel}
\usepackage{geometry}
\usepackage[fleqn]{amsmath}
\usepackage{amssymb}
\usepackage{float}
\usepackage{enumerate}
\usepackage{commath} % http://tex.stackexchange.com/questions/14821/whats-the-proper-way-to-typeset-a-differential-operator
\usepackage{cancel}

\usepackage[fleqn]{mathtools}
% Number only referenced equations
%\mathtoolsset{showonlyrefs}

%\usepackage{wrapfig}
\usepackage[thinspace,thinqspace,squaren,textstyle]{SIunits}
\usepackage{tikz}
\usepackage[europeanresistors]{circuitikz}

% New command for color underlining
\usepackage{xcolor}

\newsavebox\MBox
\newcommand\colul[2][red]{{\sbox\MBox{$#2$}%
  \rlap{\usebox\MBox}\color{#1}\rule[-1.2\dp\MBox]{\wd\MBox}{0.5pt}}}

\restylefloat{table}
\geometry{a4paper, top=15mm, left=20mm, right=10mm, bottom=20mm, headsep=10mm, footskip=12mm}
\linespread{1.5}
\setlength\parindent{0pt}
\DeclareMathOperator{\Tr}{Tr}
\DeclareMathOperator{\Var}{Var}
\begin{document}
\allowdisplaybreaks % Seitenumbrüche in Formeln erlauben
\begin{center}
\bfseries % Fettdruck einschalten
\sffamily % Serifenlose Schrift
\vspace{-40pt}
Physikalisches Grundpraktikum 2, Wintersemester 2014/2015

Markus Fenske \texttt{<iblue@zedat.fu-berlin.de>}

Ohmscher Widerstand, Tutor: Andreas Maier
\vspace{-10pt}
\end{center}
\section{Einführung}
Ziel des Versuches ist die Untersuchung von (ohmschen und nicht-ohmschen)
Widerständen und daraus aufgebauten Schaltungen. Dabei behandeln wir die
Widerstandskennlinie, Strom- und Spannungsteiler (belastet und unbelastet),
Innenwiderstände von Spannungsquellen und Messgeräten und darauf aufbauend die
strom- und spannungsrichtige Messung.

\section{Theoretische Grundlagen}
% TODO: Kirchhoffsche Regeln
% TODO: Herleitung Gesamtwiderstand von Reihen- und Parallelschaltung
% TODO: Kennlinie eines Widerstands (Ohm: linear, ansonsten nicht-linear)
% TODO: Spannungsteiler, Stromteiler
% TODO: Innenwiederstände von Strom- und Spannungsquellen sowie Messgeräten
% TODO: Ersatzschaltbilder
% TODO: Analog von Messgeräten (mit Ersatzschaltbildern)
% TODO: Strom- und spannungsrichtige Messung
% TODO: Berechnung Spannungsfehler (siehe Kursmaterial)

\subsection{Kirchhoffsche Regeln}
In der elektrischen Schaltungstechnik verwendet man die Kirchhoffschen
Regeln, um den Zusammenhang zwischen mehreren elektrischen Strömen bzw.
mehreren elektrischen Spannungen zu beschreiben. Sie bestehen aus zwei
grundlegenden Sätzen, die im Folgenden beschrieben werden sollen.

\subsubsection{Knotenregel}

\begin{figure}[H]
  \begin{center}
    \begin{circuitikz}
      \draw (0,0) to (4,0);
      \draw (0,0) to (-4,0);
      \draw (0,0) to (0,2);
      \draw (0,0) to (0,-2);
      \draw node[circ] at (0,0) {};
      \draw[<-] (2.5,0.4) -- ++(1.4,0)    node [midway,fill=white] {$I_1$};
      \draw[<-] (-2.5,0.4) -- ++(-1.5,0)  node [midway,fill=white] {$I_3$};
      \draw[<-] (0.35,0.6) -- ++(0,1.4)   node [midway,fill=white] {$I_2$};
      \draw[<-] (0.35,-0.6) -- ++(0,-1.4) node [midway,fill=white] {$I_4$};
    \end{circuitikz}
    \caption{Knotenregel}
  \end{center}
\end{figure}


Betrachten wir einen beliebigen Knoten innerhalb einer elektrischen
Schaltung, also einen Punkt in dem mehrere Leitungen elektrisch verbunden
sind, so ist klar, dass in diesem Punkt keine Ladung gespeichert werden
kann. Aufgrund der Ladungserhaltung muss also zufließende Ladung wieder
abfließen. Den Fluss von Ladungen definiert man als elektrischen Strom

\begin{equation}
  I := \frac{\dif Q}{\dif t}
\end{equation}

Soll der Zufluss von Ladungen also dem Abfluss von Ladungen entsprechend,
muss die Summe aller Ströme $I_1, I_2, \dots, I_n$ in den Knoten hinein und
aus dem Knoten herraus verschwinden.

\begin{equation}
  \sum_{n=1}^k I_n = 0
\end{equation}

Diese Erkenntnis bezeichnet man als Knotensatz oder auch 1. Kirchhoffsches
Gesetz.

Es gilt natürlich nur für Knoten, die elektrisch neutral bleiben. Wird zum
Beispiel ein Kondensator benutzt, so werden die Ladungen auf der
Kondensatorplatte gespeichert. Betrachtet man also nur eine Kondensatorplatte,
so muss zusätzlich der Verschiebungsstrom berücksichtigt werden. Da wir in
diesem Versuch nicht mit Kondensatoren oder Spulen arbeiten, sondern nur
statische Fälle betrachten, kann dies unberücksichtigt bleiben.

\subsubsection{Maschenregel}

\end{document}
