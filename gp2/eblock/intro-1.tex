\documentclass[a4paper,german,12pt,smallheadings]{scrartcl}
\usepackage[T1]{fontenc}
\usepackage[utf8]{inputenc}
\usepackage{babel}
\usepackage{geometry}
\usepackage[fleqn]{amsmath}
\usepackage{amssymb}
\usepackage{float}
\usepackage{enumerate}
\usepackage{commath} % http://tex.stackexchange.com/questions/14821/whats-the-proper-way-to-typeset-a-differential-operator
\usepackage{cancel}

\usepackage[fleqn]{mathtools}
% Number only referenced equations
%\mathtoolsset{showonlyrefs}

%\usepackage{wrapfig}
\usepackage[thinspace,thinqspace,squaren,textstyle]{SIunits}
\usepackage{tikz}
\usepackage[europeanresistors]{circuitikz}

% New command for color underlining
\usepackage{xcolor}

\newsavebox\MBox
\newcommand\colul[2][red]{{\sbox\MBox{$#2$}%
  \rlap{\usebox\MBox}\color{#1}\rule[-1.2\dp\MBox]{\wd\MBox}{0.5pt}}}

\restylefloat{table}
\geometry{a4paper, top=15mm, left=20mm, right=10mm, bottom=20mm, headsep=10mm, footskip=12mm}
\linespread{1.5}
\setlength\parindent{0pt}
\DeclareMathOperator{\Tr}{Tr}
\DeclareMathOperator{\Var}{Var}
\begin{document}
\allowdisplaybreaks % Seitenumbrüche in Formeln erlauben
\begin{center}
\bfseries % Fettdruck einschalten
\sffamily % Serifenlose Schrift
\vspace{-40pt}
Physikalisches Grundpraktikum 2, Wintersemester 2014/2015

Markus Fenske \texttt{<iblue@zedat.fu-berlin.de>}

Ohmscher Widerstand, Tutor: Andreas Maier
\vspace{-10pt}
\end{center}
\section{Einführung}
Ziel des Versuches ist die Untersuchung von (ohmschen und nicht-ohmschen)
Widerständen und daraus aufgebauten Schaltungen. Dabei behandeln wir die
Widerstandskennlinie, Strom- und Spannungsteiler (belastet und unbelastet),
Innenwiderstände von Spannungsquellen und Messgeräten und darauf aufbauend die
strom- und spannungsrichtige Messung.

\section{Theoretische Grundlagen}
% TODO: Kirchhoffsche Regeln
% TODO: Herleitung Gesamtwiderstand von Reihen- und Parallelschaltung
% TODO: Kennlinie eines Widerstands (Ohm: linear, ansonsten nicht-linear)
% TODO: Spannungsteiler, Stromteiler
% TODO: Innenwiederstände von Strom- und Spannungsquellen sowie Messgeräten
% TODO: Ersatzschaltbilder
% TODO: Analog von Messgeräten (mit Ersatzschaltbildern)
% TODO: Strom- und spannungsrichtige Messung
% TODO: Berechnung Spannungsfehler (siehe Kursmaterial)

\subsection{Kirchhoffsche Regeln}
In der elektrischen Schaltungstechnik verwendet man die Kirchhoffschen
Regeln, um den Zusammenhang zwischen mehreren elektrischen Strömen bzw.
mehreren elektrischen Spannungen zu beschreiben. Sie bestehen aus zwei
grundlegenden Sätzen, die im Folgenden beschrieben werden sollen.

\subsubsection{Knotenregel}

\begin{figure}[H]
  \begin{center}
    \begin{circuitikz}
      \draw (0,0) to (4,0);
      \draw (0,0) to (-4,0);
      \draw (0,0) to (0,2);
      \draw (0,0) to (0,-2);
      \draw node[circ] at (0,0) {};
      \draw[<-] (2.5,0.4) -- ++(1.4,0)    node [midway,fill=white] {$I_1$};
      \draw[<-] (-2.5,0.4) -- ++(-1.5,0)  node [midway,fill=white] {$I_3$};
      \draw[<-] (0.35,0.6) -- ++(0,1.4)   node [midway,fill=white] {$I_2$};
      \draw[<-] (0.35,-0.6) -- ++(0,-1.4) node [midway,fill=white] {$I_4$};
    \end{circuitikz}
    \caption{Knotenregel}
  \end{center}
\end{figure}


Betrachten wir einen beliebigen Knoten innerhalb einer elektrischen
Schaltung, also einen Punkt in dem mehrere Leitungen elektrisch verbunden
sind, so ist klar, dass in diesem Punkt keine Ladung gespeichert werden
kann. Aufgrund der Ladungserhaltung muss also zufließende Ladung wieder
abfließen. Den Fluss von Ladungen definiert man als elektrischen Strom

\begin{equation}
  I := \frac{\dif Q}{\dif t}
\end{equation}

Soll der Zufluss von Ladungen also dem Abfluss von Ladungen entsprechend,
muss die Summe aller Ströme $I_1, I_2, \dots, I_n$ in den Knoten hinein und
aus dem Knoten herraus verschwinden.

\begin{equation}
  \sum_{n=1}^k I_n = 0
\end{equation}

Diese Erkenntnis bezeichnet man als Knotensatz oder auch 1. Kirchhoffsches
Gesetz.

Es gilt natürlich nur für Knoten, die elektrisch neutral bleiben. Wird zum
Beispiel ein Kondensator benutzt, so werden die Ladungen auf der
Kondensatorplatte gespeichert. Betrachtet man also nur eine Kondensatorplatte,
so muss zusätzlich der Verschiebungsstrom berücksichtigt werden. Da wir in
diesem Versuch nicht mit Kondensatoren oder Spulen arbeiten, sondern nur
statische Fälle betrachten, kann dies unberücksichtigt bleiben.

\subsubsection{Maschenregel}
\begin{figure}[H]
  \begin{center}
    \begin{circuitikz}
      \draw (0,0) to[R=$R_1$, v=$U_1$] (6,0)
                  to[R=$R_2$, v=$U_2$] (6,2)
                  to[R=$R_3$, v=$U_3$] (0,2)
                  to[R=$R_4$, v=$U_4$] (0,0);
    \end{circuitikz}
    \caption{Maschenregel}
  \end{center}
\end{figure}

Die 3. Maxwellsche Gleichung stellt den Zusammengang zwischen der elektrischen
Zirkulation über eine Randfläche $\partial A$ und der Änderung des magnetischen
Flusses durch die Fläche $A$. Sie lautet in Integralschreibweise

\begin{equation}
  \oint\limits_{\partial A} \vec{E} \cdot \dif \vec{s} = - \iint\limits_{A} \frac{\partial \vec{B}}{\partial t} \cdot \vec{A}
\end{equation}


Bei Betrachtungen elektrischer Schaltkreise mit zeitlich konstanten Strömen
existieren keine zeitlich variablen Magnetfelder, also folgt:

\begin{equation}
  \oint\limits_{\partial A} \vec{E} \cdot \dif \vec{s} = 0
  \label{eq:maxwell3static}
\end{equation}


Das Ringintegral in (\ref{eq:maxwell3static}) lässt sich aufteilen in $n$
Teilstücke $[A_1, A_2], [A_2, A_3], \dots, [A_{n-1}, A_n], [A_n, A_1]$. Damit
wird das Integral zu

\begin{equation}
  \int\limits_{A_1}^{A_2} \vec{E} \cdot \dif \vec{s} +
  \int\limits_{A_2}^{A_3} \vec{E} \cdot \dif \vec{s} +
  \dots +
  \int\limits_{A_{n-1}}^{A_n} \vec{E} \cdot \dif \vec{s} +
  \int\limits_{A_{n}}^{A_1} \vec{E} \cdot \dif \vec{s}
  = 0
\end{equation}

Die elektrische Spannung zwischen zwei Punkten $A$ und $B$ ist definiert durch
\begin{equation}
  U_{AB} = \int\limits_A^B \vec{E} \cdot \dif \vec{s}
\end{equation}

Somit

\begin{equation}
  \underbrace{U_{A_1A_2}}_{=: U_1} +
  \underbrace{U_{A_2A_3}}_{=: U_2} +
  \dots +
  \underbrace{U_{A_{n-1}A_n}}_{=: U_{n-1}} +
  \underbrace{U_{A_nA_1}}_{=: U_n} = 0
\end{equation}

Das bedeutet die Summe der Spannungen innerhalb einer Masche verschwindet.

\begin{equation}
  \sum_{n=1}^k U_n = 0
\end{equation}

Diese Regel ist bekannt als 2. Kirchhoffsches Gesetz oder als Maschenregel.

Sie gilt nur für zeitlich konstante Magnetfelder. Werden Spulen oder
Kondensatoren eingesetzt, kann nur der statische Fall betrachtet werden. Es
gibt jedoch Korrekturen für Wechselströme (komplexe Wechselstromrechnung).
Diese sind jedoch für diesen Versuch nicht relevant und werden daher nicht
betrachtet.

\subsection{Ohmsche Widerstände}

Der elektrische Widerstand $R$ ist in der Elektrotechnik ein Maß dafür, welche
Spannung $U$ notwendig ist, um einen bestimmten Strom $I$ durch einen Leiter
fließen zu lassen. Er ist definiert durch das \textit{Ohmsche Gesetz}

\begin{equation}
  R = \frac{U}{I}
\end{equation}

Wenn $R$ unabhängig von Strom und Spannung ist (also $R = \text{const.}$),
spricht man von einem \textit{ohmschen Widerstand}.

Im Folgenden wollen wir einfache Schaltungen aus mehreren ohmschen Widerständen
betrachten, um deren effektiven Gesamtwiderstand zu berechnen.

\subsubsection{Reihenschaltung}

% Schaltbild Reihenschaltung R_1 ... R_n mit Spannung U_0
\begin{figure}[H]
  \begin{center}
    \begin{circuitikz}
      \draw (0,0)
      to[V,v=$U_0$] (0,2)
      to[R=$R_1$, v=$U_1$] (2,2)
      to[R=$R_2$, v=$U_2$] (4,2);
      \draw [dashed] (4,2) -- (6,2);
      \draw (6,2)
      to[R=$R_n$, v=$U_n$] (8,2)
      to[short] (8,0)
      to[short] (0,0);
    \end{circuitikz}
    \caption{Reihenschaltung von Widerständen}
  \end{center}
\end{figure}

Wenn $n$ Widerstände $R_1, \dots, R_n$ in Reihe geschaltet sind (siehe
Abbildung), fällt über jedem eine bestimmte Spannung $U_i$ ab.

Gemäß der Maschenregel folgt sofort, dass die Summe der abfallenden Spannungen
gleich der Summe der von der Spannungsquelle erzeugten Spannung sein muss.

\begin{equation}
  \sum_{i=1}^n U_i = U_0
\end{equation}

Wenn wir das Ohmsche Gesetz für jeden Widerstand einsetzen, erhalten wir sofort

\begin{equation}
  \sum_{i=1}^n I_i R_i = U_0
\end{equation}

Aus der Knotenregel wird klar, dass alle Ströme durch die Widerstände gleich
sein müssen, also $I_i =: I_0$. Wir können diesen Term vor die Summe ziehen.

\begin{equation}
  I_0 \sum_{i=1}^n R_i = U_0
\end{equation}

Durch Umstellen ergibt sich

\begin{equation}
  \frac{U_0}{I_0} = \sum_{i=1}^n R_i
\end{equation}

Die linke Seite hat die Dimension eines Widerstandes (ohmsches Gesetz), also
schreiben wir dafür $R_\text{ges} = \frac{U_0}{I_0}$ und erhalten


\begin{equation}
  R_\text{ges} = \sum_{i=1}^n R_i
\end{equation}

In einer Reihenschaltung ergibt sich also der effektive Gesamtwiderstand durch
Addition der einzelnen Widerstände.


\end{document}
