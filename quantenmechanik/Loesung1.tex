\documentclass[a4paper,german,12pt,smallheadings]{scrartcl}
\usepackage[T1]{fontenc}
\usepackage[utf8]{inputenc}
\usepackage{babel}
\usepackage{tikz}
\usepackage{geometry}
\usepackage{amsmath}
\usepackage{amssymb}
\usepackage{float}
\usepackage{enumerate}
%\usepackage{wrapfig}
\usepackage[thinspace,thinqspace,squaren,textstyle]{SIunits}
\restylefloat{table}
\geometry{a4paper, top=15mm, left=20mm, right=40mm, bottom=20mm, headsep=10mm, footskip=12mm}
\linespread{1.5}
\setlength\parindent{0pt}
\begin{document}
\begin{center}
\bfseries % Fettdruck einschalten
\sffamily % Serifenlose Schrift
\vspace{-40pt}
Quantum Mechanics, winter term 2013/2014, exercise sheet 1

Markus Fenske
\vspace{-10pt}
\end{center}

\section*{Problem 1: Raditation and photons}

\begin{enumerate}[a)]

\item
  Using $E = hf = \frac{hc}{\lambda}$ I get the following values ($1$ eV $\approx 1.6\cdot10^{-19}\;\joule$).
  \vspace{7mm}

  \begin{tabular}{r | r | r | c}
    $\lambda$ & $E\;\text{[eV]}$ & $E\;\text{[J]}$ & spectrum \\
    \hline
    140.2 nm  & 8.8 eV & $1.4 \cdot 10^{-18}$ J & UV \\
    313.2 nm  & 4.0 eV & $6.4 \cdot 10^{-19}$ J & UV \\
    546.0 nm  & 2.3 eV & $3.7 \cdot 10^{-19}$ J & V \\
    1207.2 nm  & 1.0 eV & $1.6 \cdot 10^{-19}$ J & IR \\
  \end{tabular}

\item
  Using the given surface power density $S = 10^{-10}\;\watt\per\meter^2$
  and the surface $A = 5\cdot10^{-5}\;\meter^2$ I calulate the required power
  as $P = AS = 5 \cdot 10^{-15}\;\joule\per\second$. With the given photon
  energy $E=3.7 \cdot 10^{-19} J$ and the time interval $\Delta t =
  1\;\second$, this leads to the required number of photons.

  \begin{equation*}
    N = \frac{P}{E} \cdot \Delta t \approx 13500
  \end{equation*}

\end{enumerate}

\section*{Problem 3: Solar radiation}
\begin{enumerate}[a)]
  \item
    The radiation power of the earth can be calculated by Stefan–Boltzmann law as

    \begin{equation*}
      P_{\text{earth}} = \sigma A_{\text{Earth}} T^4 = \sigma 4 \pi r_{\text{Earth}}^2
    \end{equation*}

    Because this radiation power originates from the sun, the total solar
    radiation power can be calculated as

    \begin{equation*}
      P_{\text{solar}} = \frac{A_{\text{sphere}}}{A_{\text{earth shadow}}} \cdot P_{\text{earth}}
    \end{equation*}

    where $A_{\text{sphere}}$ is an imaginary sphere behind the earths orbit
    with its center in the center of the sun and $A_{\text{earth shadow}}$ is
    the shadow the earth creates on this sphere. Because the earth is small
    against the sun and at large distances the sun rays are nearly parallel,
    this can be approximated as

    \begin{equation*}
      \frac{A_{\text{sphere}}}{A_{\text{earth shadow}}} \approx \frac{4 \pi d^2}{\pi r_{\text{earth}}^2}
    \end{equation*}

    where $d$ is the distance between sun and earth.

    Putting together these equations leads to

    \begin{equation*}
      P_\text{solar} = \frac{4 \pi d^2}{\pi r_\text{Earth}^2} \sigma 4 \pi r_\text{Earth}^2 T^4 = 16\pi d^2 \sigma T^4
    \end{equation*}

    The total power generated by the sun can be estimated as

    \begin{equation*}
      P_\text{solar} = 16 \pi \left(1.5 \cdot 10^{11} \; \meter\right)^2 \cdot 5.67 \cdot 10^{-8}\;\watt\meter^{-2}\kelvin^{-4} \cdot (290 \; \kelvin)^4 \approx 4.5 \cdot 10^{26}\;\watt
    \end{equation*}
\end{enumerate}
\end{document}
