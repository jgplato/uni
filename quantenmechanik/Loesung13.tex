\documentclass[a4paper,german,12pt,smallheadings]{scrartcl}
\usepackage[T1]{fontenc}
\usepackage[utf8]{inputenc}
\usepackage{babel}
\usepackage{geometry}
\usepackage{amsmath}
\usepackage{amssymb}
\usepackage{float}
\usepackage{enumerate}
\usepackage{braket} % Teh quantum stuff
\usepackage{commath} % http://tex.stackexchange.com/questions/14821/whats-the-proper-way-to-typeset-a-differential-operator
\usepackage{cancel}
%\usepackage{wrapfig}
\usepackage[thinspace,thinqspace,squaren,textstyle]{SIunits}

% New command for color underlining
\usepackage{xcolor}

\newsavebox\MBox
\newcommand\colul[2][red]{{\sbox\MBox{$#2$}%
  \rlap{\usebox\MBox}\color{#1}\rule[-1.2\dp\MBox]{\wd\MBox}{0.5pt}}}

\restylefloat{table}
\geometry{a4paper, top=15mm, left=10mm, right=20mm, bottom=20mm, headsep=10mm, footskip=12mm}
\linespread{1.5}
\setlength\parindent{0pt}
\DeclareMathOperator{\Tr}{Tr}
\DeclareMathOperator{\Var}{Var}
\begin{document}
\allowdisplaybreaks % Seitenumbrüche in Formeln erlauben
\begin{center}
\bfseries % Fettdruck einschalten
\sffamily % Serifenlose Schrift
\vspace{-40pt}
Quantum Mechanics, winter term 2013/2014, exercise sheet 13

Markus Fenske, Tutor: Adam Nagy
\vspace{-10pt}
\end{center}

\section*{Exercise 1: Knowledge Questions}
\begin{enumerate}[a)]
  \item
    \begin{align*}
      a^\dagger \ket{n} &= \sqrt{n+1} \ket{n+1} \\
      a         \ket{n} &= \sqrt{n}   \ket{n-1}
    \end{align*}

    Eselsbrücken dafür habe ich noch keine gefunden. Es ist ganz anders als man
    es erwartet. Der Erzeugungsoperator hat ein $\dagger$ dran, obwohl man den
    \textit{dagger} (engl. Dolch) eher beim Vernichtungsoperator erwarten
    würde.

    Die Wurzeln merkt man sich am besten, indem man $N = a^\dagger a$ auf
    $\ket{0}$ und $\ket{1}$ anwendet und sieht, dass mit den falschen nicht die
    Eigenwerte $0$ und $1$ rauskommen.

  \item $\mathcal{H}_A \otimes \mathcal{H}_B$

  \item
    Dichteoperatoren werden für gemischte Zustände benötigt, also bei
    klassischen Zufallsexperimenten. Wenn ich aus einer Kiste rote und blaue
    Murmeln ziehe, aber nicht weiß, was ich habe. Nicht zu verwechseln mit der
    Superposition! Das wäre, wenn ich eine Kiste voller Murmeln habe, aber alle
    in einer Superposition von rot und blau sind.

    Beispiel wäre ein Atomofen des Stern-Gernach-Experiments. Die
    Spin-Orientierungen der Silberatome sind völlig unbekannt, können also nur
    durch eine Dichtematrix beschrieben werden.

  \item
    Der Zustand ist gemischt. Reine Zustände hätten nämlich ausschließlich
    Diagonaleinträge.

  \item
    \textbf{TODO: Tabelle der 4 Quantenzahlen}

  \item
    \textbf{TODO: Ich weiß nicht, wie ich das mache}

  \item
    Sei $A$ eine Obervable, sei $\psi(t)$ ein Zustand. $U$ ist der
    Zeitentwicklungsoperator. Dann ist der Erwartungswert von $A$

    \begin{align*}
      \underbrace{\braket{\psi(t)|A|\psi(t)}}_{\text{Schrödingerbild}}
      = \braket{\psi(0)|U^\dagger(t)AU(t)|\psi(0)}
      = \underbrace{\braket{\psi|A(t)|\psi}}_{\text{Heisenbergbild}}
    \end{align*}

    Im Schrödingerbild sind die Zustandsvektoren zeitabhängig und die
    Observablen fest, während im Heisenbergbild die Zustandsvektoren fest sind,
    aber die Observablen zeitabhängig. Der Unterschied ist in diesem Fall rein
    psychologisch. Mathematisch ist das äquivalent (so, als ob der Zug vom
    Bahnsteig oder der Bahnsteig vom Zug abfährt).

  \item
    Die Strahlungsleistung ist proportional zur vierten Potenz der Temperatur
    (und natürlich zur Fläche des Strahlers).

    \begin{align*}
      P = \sigma A T^4
    \end{align*}

  \item
    Bei der \textbf{Gasentladung in Wasserstoff} (Wasserstoff wird in einer
    Gasentladungslampe zum Leuchten angeregt), erhält man ein diskretes
    Linienspektrum. Dies ist im Rahmen der frühen halbklassischen Quantenphysik
    durch eine Drehimpulsquantelung (Bohrsches Atommodell, 1913) erklärbar.
    Erst 13 Jahre später konnten dann Pauli und Schrödinger das Wasserstoffatom
    vollständig erklären.

    \textit{Andere Antwortmöglichkeiten:}

    Beim \textbf{photoelektrischen Effekt} wird ein negativ geladener
    Metallkörper mit Licht verschiedener Wellenlänge bestrahlt. Der Körper
    entlädt sich erst, wenn er mit Licht einer bestimmten Wellenlänge bestrahlt
    wird, darunter gibt es keinen Effekt. Klassisch würde man vermuten, dass
    ein Elektron auch von zwei Photonen niedrigerer Energie ausgelöst werden
    würde, was jedoch nicht der Fall ist.

    Beim \textbf{Doppelspaltversuch} werden Elektronen durch einen Doppelspalt
    geschickt. Es ergibt sich ein Interferenzmuster, dass man klassisch nicht
    erwarten würde. Dieser wurde allerdings erst 1961 mit Elektronen
    durchgeführt.

    Außerdem \textbf{Stern-Gerlach} (würde ich nicht nehmen, zu viel Text, zu
    komplex).

  \item
    $\lambda_C = \frac{h}{mc}$
\end{enumerate}
\end{document}
