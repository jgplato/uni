\documentclass[a4paper,german,12pt,smallheadings]{scrartcl}
\usepackage[T1]{fontenc}
\usepackage[utf8]{inputenc}
\usepackage{babel}
\usepackage{tikz}
\usepackage{geometry}
\usepackage{amsmath}
\usepackage{amssymb}
\usepackage{float}
\usepackage{enumerate}
\usepackage{braket} % Teh quantum stuff
%\usepackage{wrapfig}
\usepackage[thinspace,thinqspace,squaren,textstyle]{SIunits}

% New command for color underlining
\usepackage{xcolor}

\newsavebox\MBox
\newcommand\colul[2][red]{{\sbox\MBox{$#2$}%
  \rlap{\usebox\MBox}\color{#1}\rule[-1.2\dp\MBox]{\wd\MBox}{0.5pt}}}

\restylefloat{table}
\geometry{a4paper, top=15mm, left=20mm, right=40mm, bottom=20mm, headsep=10mm, footskip=12mm}
\linespread{1.5}
\setlength\parindent{0pt}
\DeclareMathOperator{\Tr}{Tr}
\begin{document}
\begin{center}
\bfseries % Fettdruck einschalten
\sffamily % Serifenlose Schrift
\vspace{-40pt}
Quantum Mechanics, winter term 2013/2014, exercise sheet 4

Markus Fenske, Tutor: Adam Nagy
\vspace{-10pt}
\end{center}

\section*{Exercise 3: Commutators and eigenvalues of hermitian $2 \times 2$ matricies}

\begin{enumerate}[a)]
  \item
    \begin{align*}
      &[A, BC] =
      ABC - BCA =
      ABC - BCA + (BAC-BAC) =
      BAC - BCA + ABC - BAC = \\
      &B(AC-CA) + (AB-BA)C =
      B[A,C] + [A,B]C
    \end{align*}
    \begin{align*}
      &[A, [B,C]] + [B, [C,A]] + [C, [A,B]] = \\
      &[A, (BC-CB)] + [B, (CA-AC)] + [C, (AB-BA)] = \\
      &A(BC-CB)-(BC-CB)A + B(CA-AC)-(CA-AC)B + C(AB-BA)-(AB-BA)C = \\
      &\colul[red]{ABC}\colul[green]{-ACB}\colul[blue]{-BCA}\colul[yellow]{+CBA} \;\;
      \colul[blue]{+BCA}\colul[magenta]{-BAC}\colul[cyan]{-CAB}\colul[green]{+ACB} \;\;
      \colul[cyan]{+ CAB}\colul[yellow]{-CBA}\colul[red]{-ABC}\colul[magenta]{+BAC} = 0
    \end{align*}
\end{enumerate}
\end{document}
