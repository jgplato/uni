\documentclass[a4paper,german,12pt,smallheadings]{scrartcl}
\usepackage[T1]{fontenc}
\usepackage[utf8]{inputenc}
\usepackage{babel}
\usepackage{tikz}
\usepackage{geometry}
\usepackage{amsmath}
\usepackage{amssymb}
\usepackage{float}
\usepackage{enumerate}
\usepackage{braket} % Teh quantum stuff
%\usepackage{wrapfig}
\usepackage[thinspace,thinqspace,squaren,textstyle]{SIunits}

% New command for color underlining
\usepackage{xcolor}

\newsavebox\MBox
\newcommand\colul[2][red]{{\sbox\MBox{$#2$}%
  \rlap{\usebox\MBox}\color{#1}\rule[-1.2\dp\MBox]{\wd\MBox}{0.5pt}}}

\restylefloat{table}
\geometry{a4paper, top=15mm, left=20mm, right=40mm, bottom=20mm, headsep=10mm, footskip=12mm}
\linespread{1.5}
\setlength\parindent{0pt}
\DeclareMathOperator{\Tr}{Tr}
\begin{document}
\begin{center}
\bfseries % Fettdruck einschalten
\sffamily % Serifenlose Schrift
\vspace{-40pt}
Quantum Mechanics, winter term 2013/2014, exercise sheet 5

Markus Fenske, Tutor: Adam Nagy
\vspace{-10pt}
\end{center}

\section*{Exercise 1: Measurements}
\begin{enumerate}[a)]
  \item
    \begin{equation*}
      \braket{\psi|A|\psi} =
      \frac{1}{\sqrt{18}}
      \begin{pmatrix}
        2 & 1 & 2 & 3
      \end{pmatrix}
      \begin{pmatrix}
         9 & -6 & -1 &  0 \\
        -6 &  9 &  0 & -1 \\
        -1 &  0 &  9 & -6 \\
         0 & -1 & -6 &  9
      \end{pmatrix}
      \frac{1}{\sqrt{18}}
      \begin{pmatrix}
       2 \\
       1 \\
       2 \\
       3
      \end{pmatrix}
      = \frac{52}{\sqrt{18}^2} = \frac{26}{9}
    \end{equation*}
  \item
    In order to diagonalize the matrix, we first need to calculate the
    eigenvalues by solving the characteristic polynomial. 
    \begin{align*}
      &\begin{vmatrix}
        9-\lambda & -6        & -1        &  0         \\
        -6        & 9-\lambda & 0         & -1         \\
        -1        & 0         & 9-\lambda & -6         \\
        0         & -1        & -6        &  9-\lambda
      \end{vmatrix} 
      \quad\text{(Expanding the determinant along the first row)}
      \\
      = &(9-\lambda)
      \begin{vmatrix}
        9-\lambda & 0         & -1       \\
        0         & 9-\lambda & -6       \\
        -1        & -6        & 9-\lambda
      \end{vmatrix}
      +6
      \begin{vmatrix}
       -6 & 0         & -1         \\
       -1 & 9-\lambda & -6         \\
       0  & -6        & 9 - \lambda
      \end{vmatrix}
      -1
      \begin{vmatrix}
        -6 & 9 - \lambda & -1        \\
        -1 & 0           & -6        \\
        0  & -1          & 9-\lambda
      \end{vmatrix} \\
      &\text{(Using rule of Saurus)} \\
      = &(9-\lambda)((9-\lambda)^3+0+0-(9-\lambda)-36(9-\lambda)-0) \\
        &+6(-6(9-\lambda)^2+0-6-0+6^3-0) \\
        &-1(0+9-1-0+36+(9-\lambda)^2) \\
      = &(9-\lambda)^4-37(9-\lambda)^2
        -6^2(9-\lambda)^2-6^2+6^4
        -35-(9-\lambda)^2 \\
      = &(9-\lambda)^4-74(9-\lambda)^2+1225 
        \qquad\text{(Expanding the polynomial)} \\
      = &9^4-4\cdot9^3 \lambda + 6\cdot9^2\lambda^2-4\cdot9\lambda^3+\lambda^4-74(81-18\lambda+\lambda)^2+1225 \\
      = &6561-2916\lambda+486\lambda^2-36\lambda^3+\lambda^4-5994+1332\lambda-74\lambda^2+1225 \\
      = &\lambda^4-36\lambda^3+412\lambda-1584\lambda+1792
    \end{align*}
\end{enumerate}

\end{document}
