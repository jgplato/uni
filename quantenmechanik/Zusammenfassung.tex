\documentclass[a4paper,german,10pt,smallheadings]{scrartcl}
\usepackage[T1]{fontenc}
\usepackage[utf8]{inputenc}
\usepackage{babel}
\usepackage{tikz}
\usepackage{geometry}
\usepackage{amsmath}
\usepackage{amssymb}
\usepackage{float}
\usepackage{enumerate}
\usepackage{braket}
%\usepackage{wrapfig}
\usepackage[thinspace,thinqspace,squaren,textstyle]{SIunits}
\restylefloat{table}
\geometry{a4paper, top=15mm, left=20mm, right=40mm, bottom=20mm, headsep=10mm, footskip=12mm}
\linespread{1.5}
\setlength\parindent{0pt}
\begin{document}
\begin{center}
\bfseries % Fettdruck einschalten
\sffamily % Serifenlose Schrift
\vspace{-40pt}
Quantenmechanics: Semesterzusammenfassung

Markus Fenske
\vspace{-10pt}
\end{center}

\section{Doppelspaltexperiment}

Ist klar.

\section{Planck'sches Strahlungsgesetz}

Ein Problem der klassischen Physik ist die Berechnung der Strahlung in einem
hohlen Körper mit kleiner Öffnung (sogenannte Hohlraumstrahlung). Wird dieses
klassisch gelöst, konvergiert die Strahlungsleistung für kleine Wellenlängen
gegen unendlich (sog. Ultraviolettkatastrophe).

Max Planck stellte die Hypothese auf, dass die Energie gequantelt ist (d.h. sie
tritt in kleinsten Einheiten auf). Daraus folgt das Planck'sche
Strahlungsgesetz, dass die Hohlraumstrahlung korrekt wiedergibt. Es erklärt das
Strahlungsverhalten eines sog. Schwarzen Strahlers.

Ein Schwarzer Strahler ist ein Körper, der jegliche ihn treffende Strahlung
absorbiert und gemäß seiner Temperatur strahlt.

\subsection{Wichtige Begriffe} 

Schwarzer Strahler, Planck'sches Wirkungsquantum, Energie des Photons, Licht:
Sichtbare Wellenlänge etwa (violett) 380 nm - 780 nm (rot), Elektronenvolt (eV)

\subsection{Wichtige Formeln}

Planck'sches Strahlungsgesetz: Beschreibt die Modenverteilung (Wellenlängenverteilung) eines Schwarzen Strahlers.
\begin{equation}
  \omega(\nu) \; d \nu  = \frac{8 \pi h \nu^3}{c^3} \frac{1}{e^\frac{h \nu}{kT} - 1} \; d \nu
\end{equation}

Wien'sches Verschiebungsgesetz: Setzt das Wellenlängenmaximum eines Schwarzen Strahlers in Bezug zur Temperatur.
\begin{equation}
  T \cdot \lambda_\mathrm{max} = \mathrm{const} = 2897{,}8 \, \mathrm{\mu m \mathrm{K}}
\end{equation}

Stefan-Boltzmann-Gesetz: Gibt die Gesamtstrahlungsleistung eines schwarzen Strahlers in Abhängigkeit von der Temperatur an.
\begin{equation}
  P = \sigma A T^4 \qquad \text{mit} \qquad \sigma = 5{,}67 \cdot 10^{-8} \;\watt\meter^{-2}\kelvin^{-4}
\end{equation}

\newpage

\section{Welle-Teilchen-Dualismus}

Alle Teilchen (Photonen, Elektronen, \dots) haben sowohl Teilchen- als auch
Welleneigenschaften.

\section{Photoelektrischer Effekt (Photoeffekt)}
\begin{equation*}
  E_\text{kin,max} = h \nu - W_A
  \quad\quad\quad\qquad
  \quad\quad\quad\qquad
  U_\text{Brems} = \frac{E_\text{kin,max}}{\text{e}}
\end{equation*}

Elektronen werden von Photonen mit ausreichender Frequenz $\nu$ aus Metallen
ausgelöst. Zeigt die Quantennatur des Lichtes (Erhöhung der Intensität bringt
nichts, wenn Frequenz zu niedrig). Tritt nicht bei freien Elektronen auf,
sondern nur bei Elektronen in Leitern (wegen Impulserhaltung).

\section{Compton-Streuung}

Photonen werden unter Verringerung der Frequenz an Elektronen gestreut.
Herleitung über Energie- und Impulserhaltung. (Es folgt: Dem Photon muss ein
Impuls zugeschrieben werden).

\begin{equation*}
  \Delta \lambda = \lambda_C \left( 1 - \cos \phi \right)
  \quad\quad\quad\qquad
  \quad\quad\quad\qquad
  \quad\quad\quad
  \lambda_C = \frac{h}{mc}
\end{equation*}

\section{Beugung am Doppelspalt mit Wellenfunktion}

\begin{equation*}
  P = |\psi_1 + \psi_2|^2 = |\psi_1|^2 + |\psi_2|^2 + \underbrace{2 |\psi_1| |\psi_2| \cos \delta}_\text{Interferenzterm}
\end{equation*}

\section{De Broglie Wellenlänge}
Ordnet jedem Teilchen eine Wellenlänge zu. Wird benutzt um Interferenzen und
Streuverhalten zu erklären, die nach klassischer Physik nicht erklärbar wären.

\begin{equation*}
  \lambda = \frac{h}{p} = \frac{h}{mv} = \frac{h}{\sqrt{2 m E_\text{kin}}}
\end{equation*}

\textbf{Thermische De-Broglie-Wellenlänge}: Setze mittlere (quantenmechanische!) kinetische Energie
ein.

\vspace{-20pt}
\begin{equation*}
  E_\text{kin} = \pi k_B T
  \quad\quad\Rightarrow\quad\quad
  \lambda = \frac{h}{\sqrt{2 \pi k_B T m}} 
\end{equation*}

\section{Bragg-Gleichung}

Streuung von Elektronen am Kristall. Dadurch wurde die Wellennatur der
Elektronen ersichtlich. \textbf{Gleichungen fehlen - nicht verstanden!}


\section{Matrizenmechanik, Dirac-Notation (Bra-Ket-Notation)}

$|+>, |->, |0>, |1>, |\times>, \dots$: Zustandsvektoren (im komplexen Vektorraum)

$\sigma_x, \sigma_y, \sigma_z$: Messungen (hermiteische Operatoren)

$\sigma_x|0> = |0>, \sigma_x |1> = -|1>$: Eigenwerte $1, -1$ sind beobachtbar.

Bloch-Kugel für Spins.

\newpage
\section{Fouriertransformation}
Transformiert eine Funktion in die Spektralzerlegung. In der Quantenmechanik
benutzt um Wellenfunktionen vom Orts- in den Impulsraum zu transformieren.
Kleine Impulse im Ortsraum werden im Impulsraum breit und umgekehrt. $\to$
``Herleitung'' der Heisenbergschen Unschärferelation. Ortsraum: $f(x)$,
Impulsraum $g(k)$. Literatur: Forster, Analysis III.

\begin{equation*}
  g(k) = \int_{-\infty}^{\infty} f(x) \; e^{-ikx} \;\text{d}x 
  \qquad
  \qquad
  \qquad
  f(x) = \frac{1}{2 \pi} \int_{-\infty}^{\infty} g(k) \; e^{ikx} \;\text{d}k
\end{equation*}

\section{Heisenbergsche Unschärferelation}
\begin{equation*}
  \Delta x \Delta p \ge \frac{\hbar}{2}
\end{equation*}

\section{Bohrsches Atommodell}
Beobachtung: Diskrete Linienspektren. Erklärung von Bohr: Drehimpulsquantelung.
Aus den Differenzen der kinetischen
Energien der Elektronen auf diskreten Bahnen (Coulomb-Potential mit Kernladung $Z$) Herleitung der Rydberg-Formel für
das Wasserstoffatom. (Hier: Grafik: Elektron wechselt Bahn, Abstrahlung von Photon)

\begin{equation*}
  L = n \hbar
  \qquad \Rightarrow \qquad
  \frac{1}{\lambda} = R_y \left( \frac{1}{m^2} - \frac{1}{n^2} \right) \quad \text{mit} \quad n,m = 1, 2, 3, \dots
\end{equation*}

Nachteile des Bohrschen Atommodells: Gilt nur für Atome mit einem Elektron,
erklärt nicht die Intensitäten der Spektrallinien, Drehimpulsquantelung ist ein
Postulat, Erklärt nicht die Feinstruktur der Spektrallinien.

\section{Charakteristische Röntgenstrahlung}

Empirische von Moseley: Bohrsches Atommodell mit Abschirmkonstante (Abschirmung
durch besetzte Schalen):

\begin{equation*}
  E = R (Z-S)^2 \left( \frac{1}{m^2} - \frac{1}{n^2} \right)
\end{equation*}

\section{Matritzenmechanik 1: Vektorraum}
Quantenmechanik (genauer Matritzenmechanik) operiert auf einem
$d$-dimensionalen komplexen normierten Hilbertraum ($\mathcal{H} = \mathbb{C}^d$) der Basis $\ket{0}, \ket{1}, \dots$ mit
Dualraum $\bra{0}, \bra{1}, \dots$ und orthonormalen Basisvektoren.

\begin{equation*}
  \text{Ket: } \ket{\psi} = \sum_{j=0}^{d-1} \alpha_j \ket{j} = \begin{pmatrix} \alpha_0 \\ \vdots \\ \alpha_{d-1} \end{pmatrix}
  \qquad \Rightarrow \qquad
  \text{Bra: } \bra{\psi} = \sum_{j=0}^{d-1} \overline{\alpha_j} \bra{j} = (\overline{\alpha_0}, \dots, \overline{\alpha_{d-1}})
\end{equation*}

Skalarprodukt und Normierung:

\begin{equation*}
  \braket{\phi|\psi} = \sum_{j=1}^{d-1} \overline{\beta_j} \alpha_j = \braket{\psi|\phi}^*
\qquad \text{und} \qquad
  || \ket{\psi} ||^2 = \braket{\psi|\psi} = \sum_{j=0}^{d-1} |\alpha_j|^2 = 1
\end{equation*}

\newpage

\section{Matritzenmechanik 2: Overservablen}

Observablen sind hermitische (invariant wenn nacheinander transponiert und
komplex konjugiert) Operatoren $A = A^\dagger$, die in Matrixform geschrieben
werden können: $A_{jk} = \braket{j|A|k}$. Sie haben reelle Eigenwerte.

Pauli-Matritzen:

\begin{equation*}
  \sigma_z = \begin{pmatrix} 1 & 0 \\ 0 & -1\end{pmatrix}, \quad
  \sigma_y = \begin{pmatrix} 0 & -i \\ i & 0\end{pmatrix}, \quad
  \sigma_x = \begin{pmatrix} 0 & 1 \\ 1 & 0\end{pmatrix}, \quad
  I = \begin{pmatrix} 1 & 0 \\ 0 & 1\end{pmatrix}, \quad
\end{equation*}

Jede $2\times2$-Matrix lässt sisch schreiben als
\begin{equation*}
  A = \alpha \sigma_x + \beta \sigma_y + \gamma \sigma_z + \delta I
\end{equation*}

Kommutator ($[A,B] = 0$ heißt die Operatoren kommutieren, heißt die Observablen sind kompatibel):

\begin{equation*}
  [\sigma_x, \sigma_z] = \sigma_x\sigma_z - \sigma_z\sigma_x
\end{equation*}

Jede Observable kann über ihre Eigenwerte ($\lambda_n$) und Eigenvektoren
($\ket{\phi_n}$ bzw. $\bra{\phi_n}$) geschrieben werden

\begin{equation*}
  A = \sum_{j=0}^{d-1} \lambda_j \ket{\phi_j} \bra{\phi_j}
\end{equation*}

Jede Observable kann diagnonalisiert werden:

\begin{equation*}
  A = UDU^\dagger \; \text{mit} \; D = \operatorname{diag}(\lambda_0, \lambda_1, \dots, \lambda_{d-1}) \; \text{und U unitär}
\end{equation*}





\end{document}
