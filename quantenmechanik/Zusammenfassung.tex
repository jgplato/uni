\documentclass[a4paper,german,10pt,smallheadings]{scrartcl}
\usepackage[T1]{fontenc}
\usepackage[utf8]{inputenc}
\usepackage{babel}
\usepackage{tikz}
\usepackage{geometry}
\usepackage{amsmath}
\usepackage{amssymb}
\usepackage{float}
\usepackage{enumerate}
%\usepackage{wrapfig}
\usepackage[thinspace,thinqspace,squaren,textstyle]{SIunits}
\restylefloat{table}
\geometry{a4paper, top=15mm, left=20mm, right=40mm, bottom=20mm, headsep=10mm, footskip=12mm}
\linespread{1.5}
\setlength\parindent{0pt}
\begin{document}
\begin{center}
\bfseries % Fettdruck einschalten
\sffamily % Serifenlose Schrift
\vspace{-40pt}
Quantenmechanics: Semesterzusammenfassung

Markus Fenske
\vspace{-10pt}
\end{center}

\section{Doppelspaltexperiment}

Ist klar.

\section{Planck'sches Strahlungsgesetz}

Ein Problem der klassischen Physik ist die Berechnung der Strahlung in einem
hohlen Körper mit kleiner Öffnung (sogenannte Hohlraumstrahlung). Wird dieses
klassisch gelöst, konvergiert die Strahlungsleistung für kleine Wellenlängen
gegen unendlich (sog. Ultraviolettkatastrophe).

Max Planck stellte die Hypothese auf, dass die Energie gequantelt ist (d.h. sie
tritt in kleinsten Einheiten auf). Daraus folgt das Planck'sche
Strahlungsgesetz, dass die Hohlraumstrahlung korrekt wiedergibt. Es erklärt das
Strahlungsverhalten eines sog. Schwarzen Strahlers.

Ein Schwarzer Strahler ist ein Körper, der jegliche ihn treffende Strahlung
absorbiert und gemäß seiner Temperatur strahlt.

\subsection{Wichtige Begriffe} 

Schwarzer Strahler, Planck'sches Wirkungsquantum, Energie des Photons, Licht:
Sichtbare Wellenlänge etwa (violett) 380 nm - 780 nm (rot), Elektronenvolt (eV)

\subsection{Wichtige Formeln}

Planck'sches Strahlungsgesetz: Beschreibt die Modenverteilung (Wellenlängenverteilung) eines Schwarzen Strahlers.
\begin{equation}
  \omega(\nu) \; d \nu  = \frac{8 \pi h \nu^3}{c^3} \frac{1}{e^\frac{h \nu}{kT} - 1} \; d \nu
\end{equation}

Wien'sches Verschiebungsgesetz: Setzt das Wellenlängenmaximum eines Schwarzen Strahlers in Bezug zur Temperatur.
\begin{equation}
  T \cdot \lambda_\mathrm{max} = \mathrm{const} = 2897{,}8 \, \mathrm{\mu m \mathrm{K}}
\end{equation}

Stefan-Boltzmann-Gesetz: Gibt die Gesamtstrahlungsleistung eines schwarzen Strahlers in Abhängigkeit von der Temperatur an.
\begin{equation}
  P = \sigma A T^4 \qquad \text{mit} \qquad \sigma = 5{,}67 \cdot 10^{-8} \;\watt\meter^{-2}\kelvin^{-4}
\end{equation}

\newpage

\section{Welle-Teilchen-Dualismus}

Alle Teilchen (Photonen, Elektronen, \dots) haben sowohl Teilchen- als auch
Welleneigenschaften.

\section{Photoelektrischer Effekt (Photoeffekt)}
\begin{equation*}
  E_\text{kin,max} = h \nu - W_A
  \quad\quad\quad\qquad
  \quad\quad\quad\qquad
  U_\text{Brems} = \frac{E_\text{kin,max}}{\text{e}}
\end{equation*}

Elektronen werden von Photonen mit ausreichender Frequenz $\nu$ aus Metallen
ausgelöst. Zeigt die Quantennatur des Lichtes (Erhöhung der Intensität bringt
nichts, wenn Frequenz zu niedrig). Tritt nicht bei freien Elektronen auf,
sondern nur bei Elektronen in Leitern (wegen Impulserhaltung).

\section{Compton-Streuung}

Photonen werden unter Verringerung der Frequenz an Elektronen gestreut.
Herleitung über Energie- und Impulserhaltung. (Es folgt: Dem Photon muss ein
Impuls zugeschrieben werden).

\begin{equation*}
  \Delta \lambda = \lambda_C \left( 1 - \cos \phi \right)
  \quad\quad\quad\qquad
  \quad\quad\quad\qquad
  \quad\quad\quad
  \lambda_C = \frac{h}{mc}
\end{equation*}

\section{Beugung am Doppelspalt mit Wellenfunktion}

\begin{equation*}
  P = |\psi_1 + \psi_2|^2 = |\psi_1|^2 + |\psi_2|^2 + \underbrace{2 |\psi_1| |\psi_2| \cos \delta}_\text{Interferenzterm}
\end{equation*}

\section{De Broglie Wellenlänge}
Ordnet jedem Teilchen eine Wellenlänge zu. Wird benutzt um Interferenzen und
Streuverhalten zu erklären, die nach klassischer Physik nicht erklärbar wären.

\begin{equation*}
  \lambda = \frac{h}{p} = \frac{h}{mv} = \frac{h}{\sqrt{2 m E_\text{kin}}}
\end{equation*}

\textbf{Thermische De-Broglie-Wellenlänge}: Setze mittlere (quantenmechanische!) kinetische Energie
ein.

\vspace{-20pt}
\begin{equation*}
  E_\text{kin} = \pi k_B T
  \quad\quad\Rightarrow\quad\quad
  \lambda = \frac{h}{\sqrt{2 \pi k_B T m}} 
\end{equation*}

\section{Bragg-Gleichung}

Streuung von Elektronen am Kristall. Dadurch wurde die Wellennatur der
Elektronen ersichtlich. \textbf{Gleichungen fehlen - nicht verstanden!}


\section{Matrizenmechanik, Dirac-Notation (Bra-Ket-Notation)}

$|+>, |->, |0>, |1>, |\times>, \dots$: Zustandsvektoren (im komplexen Vektorraum)

$\sigma_x, \sigma_y, \sigma_z$: Messungen (hermiteische Operatoren)

$\sigma_x|0> = |0>, \sigma_x |1> = -|1>$: Eigenwerte $1, -1$ sind beobachtbar.

Bloch-Kugel für Spins.


\end{document}
