\documentclass[a4paper,german,12pt,smallheadings]{scrartcl}
\usepackage[T1]{fontenc}
\usepackage[utf8]{inputenc}
\usepackage{babel}
\usepackage{tikz}
\usepackage{geometry}
\usepackage{amsmath}
\usepackage{amssymb}
\usepackage{float}
\usepackage[thinspace,thinqspace,squaren,textstyle]{SIunits}
\restylefloat{table}
\geometry{a4paper, top=15mm, left=20mm, right=40mm, bottom=20mm, headsep=10mm, footskip=12mm}
\linespread{1.5}
\setlength\parindent{0pt}
\begin{document}
\begin{center}
\bfseries % Fettdruck einschalten
\sffamily % Serifenlose Schrift
\vspace{-40pt}
Analytische Mechanik, Sommersemester 2013, 8. Blatt \\
Luis Herrmann und Markus Fenske, Tutor: Clemens Meyer zu Rheda
\vspace{-10pt}
\end{center}
\section*{Aufgabe 2: Zentralkraft}
\subsection*{Teil a}
Wenn $F(r) = a \cdot \frac{1}{r^5}$ sein soll, folgt aus $F(r) =
-\frac{\partial}{\partial r} V(r)$ und unter Benutzung des Hinweises:

\begin{align*}
  V(r) = - \frac{2R^2L^2}{m} r^{-4}
\end{align*}

Zusammen mit der kinetischen Energie in Polarkoordinaten ergibt dies die
Lagrangefunktion:

\begin{align*}
  L &= T - V \\
    &= \frac{m}{2}(\dot{r}^2 + r^2\dot{\phi}^2) + \frac{2R^2L^2}{m} r^{-4}
\end{align*}

Durch Anwendung der Lagrange-Gleichung:

\begin{align*}
  m\ddot{r} &= -\frac{8R^2L^2}{m}r^{-5} \\
  mr^2\ddot{\phi} &= 0
\end{align*}


\section*{Aufgabe 3: Yukawa-Potential}

Da im Rest der Aufgabenstellung nur von einem Drehimpuls die Rede ist und die
$p_z$-Impulserhaltung (die in 3D-Zylinderkoordinaten auftreten würde) für das
Problem egal ist, stellen wir die Lagrange-Gleichung in 2D-Polarkoordinaten
auf.

\begin{align*}
  T &= \frac{m}{2} \left(\dot{r}^2 + r^2\dot{\phi}^2\right) \\
  V &= -\frac{k}{r} e^{-\frac{r}{a}}
\end{align*}

Dies resultiert in der Lagrangefunktion

\begin{align*}
  L &= T - V \\
    &= \frac{m}{2} \left(\dot{r}^2 + r^2\dot{\phi}^2\right) + \frac{k}{r} e^{-\frac{r}{a}}
\end{align*}

Es ergeben sich die Bewegungsgleichungen:

\begin{align*}
  m\ddot{r} &= \frac{1}{r^2} e^{-\frac{r}{a}} +  \frac{k}{ra} e^{-\frac{r}{a}} \\
            &= \left(\frac{1}{r^2} + \frac{k}{ra}\right) e^{-\frac{r}{a}} \\
  mr^2\ddot{\phi} &= 0 \\
\end{align*}

Es gibt also eine Drehimpulserhaltung:

\begin{align*}
  L_{\phi} &= mr^2\dot{\phi}
\end{align*}

Damit können wir die Gesamtenergie umschreiben und ein effektives Potential
$V'(r)$ aufstellen.

\begin{align*}
  E &= V(r) + \frac{L_{\phi}^2}{2mr^2} + \frac{m\dot{r}^2}{2} \\
  E &= V'(r) + \frac{m\dot{r}^2}{2} \\
  V'(r) &= V(r) + \frac{L_{\phi}^2}{2mr^2} \\
        &= \frac{L_{\phi}^2}{2mr^2} - \frac{k}{r} e^{-\frac{r}{a}}
\end{align*}


\end{document}
