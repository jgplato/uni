\documentclass[a4paper,german,12pt,smallheadings]{scrartcl}
\usepackage[T1]{fontenc}
\usepackage[utf8]{inputenc}
\usepackage{babel}
\usepackage{tikz}
\usepackage{geometry}
\usepackage{amsmath}
\usepackage{amssymb}
\usepackage{float}
\usepackage[thinspace,thinqspace,squaren,textstyle]{SIunits}
\restylefloat{table}
\geometry{a4paper, top=15mm, left=20mm, right=40mm, bottom=20mm, headsep=10mm, footskip=12mm}
\linespread{1.5}
\setlength\parindent{0pt}
\begin{document}
\begin{center}
\bfseries % Fettdruck einschalten
\sffamily % Serifenlose Schrift
\vspace{-40pt}
Analytische Mechanik, Sommersemester 2013, 9. Blatt \\
Luis Herrmann und Markus Fenske, Tutor: Clemens Meyer zu Rheda
\vspace{-10pt}
\end{center}
\section*{Aufgabe 1}
\subsection*{Teil a}

Wir nutzen die gegebenen Gleichungen und schreiben $c_{ij} = a_{ik} b_{kj}$

\begin{align*}
  x_i'' &= a_{ik} x'_k \\
         = a_{ik} b_{kj} x_j \\
         = c_{ij} x_j
\end{align*}

Dann ist klar, dass $C = AB$ die Transformation $\vec{r} \to \vec{r}''$ ist.

Wenn $A$ eine orthogonale Transformation ist, dann gilt $|Ar| = |r|$ (Länge
bleibt erhalten). Analog gilt für $B$: $|Br| = |r|$. Also:

\begin{align*}
  |r''| = |Ar'| = |r'| = |Br| = |r|
\end{align*}

Da $r'' = Cr$ und $|r''| = |r|$ muss $C$ eine orthogonale Transformation sein.

\subsection*{Teil b}

Es ist offensichtlich wahr, dass allgemein gilt:

\begin{align*}
  a_{ik} b_{kj} \neq b_{ik} a_{kj}
\end{align*}

Also $[A,B] \neq 0$, wie gefordert.


\section*{Aufgabe 2}

Sei

\begin{align*}
  r_i  &= a^{-1}_{ij} r_j'
  r_j' &= a_{jk} r_k \\
\end{align*}

Dann folgt

\begin{align*}
   r_i  &= a^{-1}_{ij} a_{jk} r_k
\end{align*}

Da $r_i = r_k\;\forall r$, muss $a^{-1}_{ij} a_{jk} = \delta_{ik}$.

Für die Rücktransformation gilt dies analog, woraus folgt, dass $a_{ij}
a^{-1}_{jk} = \delta_{ik}$.

Also ist $[A, A^{-1}] = E_n - E_n = 0$.

Sei $A$ orthogonal, dann bilden die Zeilen- und die Spaltenvektoren eine Orthonormalbasis. Es ist also:

\begin{align*}
  AA^T = A^TA = E_n
\end{align*}

Daraus folgt

\begin{align*}
  AA^T = E_n \Leftrightarrow A^T = A^{-1}
\end{align*}

\end{document}
