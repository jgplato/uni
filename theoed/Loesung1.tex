\documentclass[a4paper,german,12pt,smallheadings]{scrartcl}
\usepackage[T1]{fontenc}
\usepackage[utf8]{inputenc}
\usepackage{babel}
\usepackage{geometry}
\usepackage{amsmath}
\usepackage{amssymb}
\usepackage{float}
\usepackage{enumerate}
\usepackage{commath} % http://tex.stackexchange.com/questions/14821/whats-the-proper-way-to-typeset-a-differential-operator
\usepackage{cancel}
%\usepackage{wrapfig}
\usepackage[thinspace,thinqspace,squaren,textstyle]{SIunits}

% New command for color underlining
\usepackage{xcolor}

\newsavebox\MBox
\newcommand\colul[2][red]{{\sbox\MBox{$#2$}%
  \rlap{\usebox\MBox}\color{#1}\rule[-1.2\dp\MBox]{\wd\MBox}{0.5pt}}}

\restylefloat{table}
\geometry{a4paper, top=15mm, left=10mm, right=20mm, bottom=20mm, headsep=10mm, footskip=12mm}
\linespread{1.5}
\setlength\parindent{0pt}
\DeclareMathOperator{\Tr}{Tr}
\DeclareMathOperator{\Var}{Var}
\begin{document}
\allowdisplaybreaks % Seitenumbrüche in Formeln erlauben
\begin{center}
\bfseries % Fettdruck einschalten
\sffamily % Serifenlose Schrift
\vspace{-40pt}
Theoretische Elektrodynamik, Sommersemester 2014, Aufgabenblatt 1

Markus Fenske, Tutor: Clemens Meyer zu Rheda
\vspace{-10pt}
\end{center}

\section*{Aufgabe 1: Linienintegral}

Man sieht, dass das Feld $\vec{\omega}_1$ nicht konservativ ist:

\begin{align*}
  \nabla \times \vec{\omega}_1
= \begin{pmatrix} \partial_y \omega_z - \partial_z \omega_y \\ \partial_z \omega_x - \partial_x \omega_z \\ \partial_x \omega_y - \partial_y \omega_x \end{pmatrix}
= \begin{pmatrix} 0 \\ 0 \\ -2A \end{pmatrix}
\end{align*}

Daher existiert kein Skalarpotential und die Integration ist wegabhängig.

Wir parametrisieren:

\begin{align*}
  \mathcal{C}_1: \vec{r}(t) = \begin{pmatrix} a \sin(t) \\ a \cos(t) \\ 0\end{pmatrix} \text{ mit } t \in \left[-\frac{\pi}{2}, \frac{\pi}{2}\right], \;\;\;
  \mathcal{C}_2: \vec{r}(t) = \begin{pmatrix} t \\ 0 \\ 0\end{pmatrix} \text{ mit } t \in [-a,a]
\end{align*}

Damit ergeben sich die Integrale

\begin{align*}
  \int_{\mathcal{C}_1} \vec{\omega}_1 \cdot \dif\vec{r}
  &= \int\limits_{-\frac{\pi}{2}}^{\frac{\pi}{2}} \vec{\omega}(\vec{r}(t)) \cdot \frac{\dif\vec{r}}{\dif t} \dif t
  = \int\limits_{-\frac{\pi}{2}}^{\frac{\pi}{2}} \begin{pmatrix} Aa(\sin(t) + \cos(t)) \\ -Aa(\sin(t) + \cos(t)) \\ 0 \end{pmatrix} \cdot \begin{pmatrix} a\cos(t) \\ -a\sin(t) \\ 0 \end{pmatrix} \dif t \\
  &= Aa^2 \int\limits_{-\frac{\pi}{2}}^{\frac{\pi}{2}} \sin(t)\cos(t) + \cos^2(t) + \sin^2(t) + \sin(t)\cos(t) \dif t \\
  &= Aa^2 \int\limits_{-\frac{\pi}{2}}^{\frac{\pi}{2}} 1 + 2\cos(t)\sin(t) \dif t
  = Aa^2 \int\limits_{-\frac{\pi}{2}}^{\frac{\pi}{2}} 1 + \sin(2t) \dif t
  = Aa^2 \pi \text{ (weil $\sin$ ungerade)}
\end{align*}

und

\begin{align*}
  \int_{\mathcal{C}_2} \vec{\omega}_1 \cdot \dif\vec{r}
  &= \int\limits_{-a}^{a} \vec{\omega}(\vec{r}(t)) \cdot \frac{\dif\vec{r}}{\dif t} \dif t
  = \int\limits_{-a}^{a} \begin{pmatrix} At \\ -At \\ 0  \end{pmatrix} \cdot \begin{pmatrix} 1 \\ 0 \\ 0\end{pmatrix} \dif t
  = \int\limits_{-a}^{a} At \dif t = 0
\end{align*}

Das Feld $\vec{\omega}_2$ ist hingegen konservativ (weil Gradient eines Skalarpotentials)

\begin{align*}
  \vec{\omega}_2 = \vec{\nabla} \underbrace{\left( \frac{A}{2} (x+y)^2 + \frac{B}{2} z^2 \right)}_{=: \Phi}
\end{align*}

Damit ist die Integration wegunabhängig, die beiden Integrale sind also gleich.
Das Ergebnis ergibt sich als Potentialdifferenz $\Phi(-a, 0, 0) - \Phi(a, 0, 0)
= 0$.

\section*{Aufgabe 2: Flächenintegral}

Über die Anwendung des Gaußschen Satzes erhalten wir

\begin{equation*}
  \oint\limits_\text{$\partial$ Würfel} \vec{r} \dif \vec{a} = \int\limits_\text{Würfel} \nabla \vec{r} \dif V = 3 \int\limits_\text{Würfel} \dif V = 3
\end{equation*}

\section*{Aufgabe 3: Gradient, Divergenz und Rotation}

Wir benutzen hier auch den Gradienten in Kugelkoordinaten ($\nabla = \hat{r}
\partial_r + \frac{1}{r} \hat{\theta} \partial_\theta + \frac{1}{r \sin \theta}
\hat{\phi} \partial_\phi$), wodurch die Lösung einfacher wird.

\begin{enumerate}[a)]
  \item $\vec{\nabla} U = 2 \vec{r}$ mit $\vec{r} = (x,y,z)$
  \item $\vec{\nabla} U = -2ab \exp(-br^2) \vec{r}$
  \item $\vec{\nabla} U = mr^{m-1} \hat{r} = mr^{m-2} \vec{r}$
  \item
    Für die $x$-Komponente
    \begin{align*}
      \frac{\partial}{\partial x} \frac{p_x x}{(x^2+y^2+z^2)^\frac{3}{2}}
      &= \frac{p_x(x^2+y^2+z^2)^\frac{3}{2} - 3p_xx^2\sqrt{x^2+y^2+z^2}}{(x^2+y^2+z^2)^3} \\
      &= \frac{p_x r^3}{r^6} - \frac{3p_x x^2 r}{r^6} \\
      &= \frac{p_x}{r^3} - 3 \frac{p_x x}{r^5} x
    \end{align*}

    Für die anderen Komponenten analog. Somit
    \begin{align*}
      \nabla U = \frac{\vec{p}}{r^3} - 3 \frac{\vec{p} \cdot \vec{r}}{r^5} \vec{r}
    \end{align*}
\end{enumerate}


Für den nächsten Teil nutzen wir auch die Divergenz in Kugelkoordinaten:
$\nabla \vec{v} = \frac{1}{r^2} \frac{\partial}{\partial r} r^2 v_r + \dots$

\begin{enumerate}[a)]
  \setcounter{enumi}{4}
  \item $\vec{\nabla} \vec{v} = 3$
  \item $\vec{\nabla} \vec{v} = \frac{1}{r^2} \frac{\partial}{\partial r} r^2 \frac{r}{r^3} = 0$
  \item
    \begin{align*}
      \nabla \vec{c}(\vec{r} \cdot \vec{c})
      &= \nabla \vec{c}(r_1c_1 + r_2c_2 + r_3c_3) \\
      & = \partial_1 c_1 (r_1c_1 + r_2c_2 + r_3c_3) +
      \partial_2 c_2 (r_1c_1 + r_2c_2 + r_3c_3) +
      \partial_3 c_3 (r_1c_1 + r_2c_2 + r_3c_3) \\
      &= c_1^2 + c_2^2 + c_3^2 = c^2
    \end{align*}
  \item
    \begin{align*}
      \nabla \vec{c} \times (\vec{r} \times \vec{c})
      &= \nabla \vec{c} \times \begin{pmatrix}
        r_2c_3 - r_3c_2 \\
        r_3c_1 - r_1c_3 \\
        r_1c_2 - r_2c_1
      \end{pmatrix} \\
      &= \nabla \begin{pmatrix}
        c_2(c_2r_1 - c_1r_2) - c_3(c_1r_3 - c_3r_1) \\
        c_3(c_3r_2 - c_2r_3) - c_1(c_2r_1 - c_1r_2) \\
        c_1(c_1r_3 - c_3r_1) - c_2(c_3r_2 - c_2r_3)
      \end{pmatrix} \\
      &= c_2^2 + c_3^2 + c_3^2 + c_1^2 + c_1^2 + c_2^2 \\
      &= 2c^2
    \end{align*}

  \item $(0,0,2)$
  \item
    \begin{align*}
    \nabla \times (\vec{c} \times \vec{r})
    &= \vec{c} \underbrace{(\nabla \vec{r})}_{=3}
      - \vec{r} \underbrace{(\nabla \vec{c})}_{=0}
      + \underbrace{(\vec{r} \nabla) \vec{c}}_{=0}
      - (\vec{c} \nabla) \vec{r} \\
    &= 3\vec{c} - \underbrace{(c_1\partial_1 + c_2\partial_2 + c_3\partial_3) \vec{r}}_{= \vec{c}} \\
    &= 2\vec{c}
    \end{align*}
\end{enumerate}

\end{document}
