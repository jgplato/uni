\documentclass[a4paper,german,12pt,smallheadings]{scrartcl}
\usepackage[T1]{fontenc}
\usepackage[utf8]{inputenc}
\usepackage{babel}
\usepackage{geometry}
\usepackage[fleqn]{mathtools} % also includes mathclap
\usepackage[fleqn]{amsmath}
\usepackage{amssymb}
\usepackage{float}
\usepackage{enumerate}
\usepackage{commath} % http://tex.stackexchange.com/questions/14821/whats-the-proper-way-to-typeset-a-differential-operator
\usepackage{cancel}

% Number only referenced equations
\mathtoolsset{showonlyrefs}

%\usepackage{wrapfig}
\usepackage[thinspace,thinqspace,squaren,textstyle]{SIunits}

% New command for color underlining
\usepackage{xcolor}

\newsavebox\MBox
\newcommand\colul[2][red]{{\sbox\MBox{$#2$}%
  \rlap{\usebox\MBox}\color{#1}\rule[-1.2\dp\MBox]{\wd\MBox}{0.5pt}}}

\restylefloat{table}
\geometry{a4paper, top=15mm, left=10mm, right=20mm, bottom=20mm, headsep=10mm, footskip=12mm}
\linespread{1.2}
\setlength\parindent{0pt}
\DeclareMathOperator{\Tr}{Tr}
\DeclareMathOperator{\Var}{Var}
\newcommand*\laplace{\mathop{}\!\mathbin\Delta}
\begin{document}
\allowdisplaybreaks % Seitenumbrüche in Formeln erlauben
\begin{center}
\bfseries % Fettdruck einschalten
\sffamily % Serifenlose Schrift
\vspace{-40pt}
Theoretische Elektrodynamik, Sommersemester 2014, Aufgabenblatt 10

Markus Fenske, Mattis Riediger, Tutor: Clemens Meyer zu Rheda
\vspace{-10pt}
\end{center}

\section*{Aufgabe 2: Welle mit Anfangsbedingungen}
Die Lösung der allgemeinen Wellengleichung ist eine Kombination zweier eindimensionaler $C_2$-Funktionen:
\begin{equation}
  u(x,t) = f(x+ct) + g(x-ct)
\end{equation}

Aus der ersten Anfangsbedingung $u(x,0) = 0$ erhalten wir daher
\begin{equation}
  f(x) + g(x) = 0
\end{equation}

Aus der zweiten Anfangsbedingung links:
\begin{align}
 \eval{\partial_t u(x,t)}_{\mathrlap{t=0}} 
 &= \eval{\partial_t \del{f(x+ct) + g(x-ct)}}_{t=0} \\
 &= \eval{\partial_{x+ct} f(x+ct) \partial_{t} (x+ct)}_{\mathrlap{t=0}} +
    \eval{\partial_{x-ct} g(x-ct) \partial_{t} (x-ct)}_{\mathrlap{t=0}} \\
 &= \eval{c \partial_x \del{f(x) - g(x)}}_{t=0} \\
 &= \eval{2c \partial_x f}_{t=0} \\
 &= 2c \partial_x f
\end{align}

Mit $2c \partial_x f = 4c \epsilon \dfrac{x}{(x^2+\epsilon^2)^2}$ folgt:
\begin{align}
  f(x)
  &= 2 \epsilon \int \dif x \; \frac{x}{(x^2+\epsilon^2)^2} \\
  &= -\frac{\epsilon}{x^2 + \epsilon^2} + C
\end{align}

Somit
\begin{equation}
  u(x,t) = f(x+ct) - f(x-ct) = -\frac{\epsilon}{(x+ct)^2 + \epsilon^2} + \frac{\epsilon}{(x-ct)^2 + \epsilon^2}
\end{equation}

Es handelt sich bei dieser Wellengleichung um zwei aufeinander zulaufende
gaußglockenförmige Pulse, die sich bei $x=0$ treffen, dabei kurzzeitig
auslöschen und wieder ausseinander laufen. Für betragsmäßig große Zeiten wären
die Pulse unendlich weit voneinander entfernt. Damit die also noch im
Plotbereich wären, müsste ich herrauszoomen. Für $t \to \pm \infty$ unendlich
weit, so dass die Pulse unendlich klein werden. Es bleibt nichts weiter als
eine flache Linie $u(x) = 0$ übrig.

Da du dir das gerade vorgestellt hast, ist eine Zeichnung hinfällig.

\end{document}
