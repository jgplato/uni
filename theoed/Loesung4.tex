\documentclass[a4paper,german,12pt,smallheadings]{scrartcl}
\usepackage[T1]{fontenc}
\usepackage[utf8]{inputenc}
\usepackage{babel}
\usepackage{geometry}
\usepackage[fleqn]{mathtools} % also includes mathclap
\usepackage[fleqn]{amsmath}
\usepackage{amssymb}
\usepackage{float}
\usepackage{enumerate}
\usepackage{commath} % http://tex.stackexchange.com/questions/14821/whats-the-proper-way-to-typeset-a-differential-operator
\usepackage{cancel}

% Number only referenced equations
\mathtoolsset{showonlyrefs}

%\usepackage{wrapfig}
\usepackage[thinspace,thinqspace,squaren,textstyle]{SIunits}

% New command for color underlining
\usepackage{xcolor}

\newsavebox\MBox
\newcommand\colul[2][red]{{\sbox\MBox{$#2$}%
  \rlap{\usebox\MBox}\color{#1}\rule[-1.2\dp\MBox]{\wd\MBox}{0.5pt}}}

\restylefloat{table}
\geometry{a4paper, top=15mm, left=10mm, right=20mm, bottom=20mm, headsep=10mm, footskip=12mm}
\linespread{1.2}
\setlength\parindent{0pt}
\DeclareMathOperator{\Tr}{Tr}
\DeclareMathOperator{\Var}{Var}
\begin{document}
\allowdisplaybreaks % Seitenumbrüche in Formeln erlauben
\begin{center}
\bfseries % Fettdruck einschalten
\sffamily % Serifenlose Schrift
\vspace{-40pt}
Theoretische Elektrodynamik, Sommersemester 2014, Aufgabenblatt 4

Markus Fenske, Mattis Riediger, Tutor: Clemens Meyer zu Rheda
\vspace{-10pt}
\end{center}

\section*{Aufgabe 1: Geladene Kugeln}
Aus $\nabla E = \dfrac{\rho}{\epsilon_0}$ folgt per Gaußschem Satz
\begin{equation}
  \oint\limits_{\partial V} \dif \vec{a} \cdot \vec{E} = \int\limits_{V} \dif V \; \frac{\rho}{\epsilon_0} = \frac{Q(V)}{\epsilon_0}
\end{equation}
dabei soll $Q(V)$ die im Volumen $V$ eingeschlossene Ladung sein.

Da $\rho(r)$ nur von $r$ abhängt, ist das gesamte Problem kugelsymmetrisch.
Das bedeutet, Kugelschalen müssen Äquipotentialflächen bilden, auf denen
$\vec{E}(r)$ senkrecht steht. Somit ist
\begin{align}
  &E(r) \underbrace{\oint\limits_{\partial V} \dif A}_{\mathclap{= \text{ Kugeloberfläche: } 4 \pi r^2}} = \frac{Q(V)}{\epsilon_0} \\
  \Leftrightarrow\quad
  & \vec{E}(r) = \frac{Q(r)}{4 \pi \epsilon_0 r^2} \hat{e_r}
\end{align}

Dies gilt für alle rotationssymmetrischen elektrostatischen Probleme.

\begin{enumerate}[a)]
  \item
    In diesem Fall ist
    \begin{equation}
      Q(r) = \begin{cases}
        0 & \text{ für } r < R \\
        Q & \text{ für } r \ge R
      \end{cases}
    \end{equation}

    also
    \begin{equation}
      \vec{E}(r) = \begin{cases}
        % FIXME: Evil typographic hack.
        \; \; \; \; 0 & \text{ für } r < R
        \vspace{0.2 cm} \\
        \dfrac{Q}{4 \pi \epsilon_0 r^2} \hat{e}_r & \text{ für } r \ge R
      \end{cases}
    \end{equation}

    % TODO: Skizze. "Skizzieren Sie das elektrische Feld" kann auch sein, den
    % Betrag skizzieren und die Richung angeben? :)
  \item
    Bestimmung der Konstanten $c_n$:

    \begin{align}
      &Q = \iiint\limits_{\mathclap{\text{Kugel mit Radius $R$}}} \dif V \; \rho(r)
      = \int\limits_0^R \dif r \int\limits_0^\pi \dif \theta \int\limits_0^{2 \pi} \dif \phi \; r^2 \sin \theta c_n r^n
      = 4 \pi c_n \int\limits_0^R \dif r \; r^{n+2}
      = 4 \pi c_n \frac{r^{n+3}}{n+3} \\
      \Leftrightarrow\quad
      &c_n = \frac{Q}{4 \pi} \frac{n+3}{R^{n+3}}
    \end{align}

    Wir leiten nun die Funktion $Q(r)$ her, die die in der Kugel mit Radius $r
    \le R$ eingeschlossene Ladung angibt. Für $r > R$ ist $Q(r) = Q$. Für $r \le R$:

    \begin{align}
      &Q(r) = 4 \pi \int\limits_0^r \dif r' \; r'^2 \rho(r') = Q \frac{n+3}{R^{n+3}} \int\limits_0^r \dif r' \; r'^{n+2} = Q \frac{r^{n+3}}{R^{n+3}}
    \end{align}

    Mit der zu Beginn der Aufgabe hergeleiteten Formel ergibt sich das elektrische Feld
    \begin{equation}
      \vec{E}(r) = \dfrac{Q}{4 \pi \epsilon_0 r^2} \hat{e}_r \begin{cases}
        \dfrac{r^{n+3}}{R^{n+3}} & \text{ für } r < R \\
        1 & \text{ für } r \ge R
      \end{cases}
    \end{equation}

    Also explizit ausgeschrieben für den Fall $n = -2$

    \begin{equation}
      \vec{E}(r) = \dfrac{Q}{4 \pi \epsilon_0} \hat{e}_r \begin{cases}
        \dfrac{1}{rR} & \text{ für } r < R \\
        \frac{1}{r^2} & \text{ für } r \ge R
      \end{cases}
    \end{equation}

    Und für $n = 2$

    \begin{equation}
      \vec{E}(r) = \dfrac{Q}{4 \pi \epsilon_0} \hat{e}_r \begin{cases}
        \dfrac{r^3}{R^5} & \text{ für } r < R \\
        \frac{1}{r^2} & \text{ für } r \ge R
      \end{cases}
    \end{equation}

    %FIXME: Skizze!
\end{enumerate}
\end{document}
