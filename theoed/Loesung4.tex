\documentclass[a4paper,german,12pt,smallheadings]{scrartcl}
\usepackage[T1]{fontenc}
\usepackage[utf8]{inputenc}
\usepackage{babel}
\usepackage{geometry}
\usepackage[fleqn]{mathtools} % also includes mathclap
\usepackage[fleqn]{amsmath}
\usepackage{amssymb}
\usepackage{float}
\usepackage{enumerate}
\usepackage{commath} % http://tex.stackexchange.com/questions/14821/whats-the-proper-way-to-typeset-a-differential-operator
\usepackage{cancel}

% Number only referenced equations
\mathtoolsset{showonlyrefs}

%\usepackage{wrapfig}
\usepackage[thinspace,thinqspace,squaren,textstyle]{SIunits}

% New command for color underlining
\usepackage{xcolor}

\newsavebox\MBox
\newcommand\colul[2][red]{{\sbox\MBox{$#2$}%
  \rlap{\usebox\MBox}\color{#1}\rule[-1.2\dp\MBox]{\wd\MBox}{0.5pt}}}

\restylefloat{table}
\geometry{a4paper, top=15mm, left=10mm, right=20mm, bottom=20mm, headsep=10mm, footskip=12mm}
\linespread{1.2}
\setlength\parindent{0pt}
\DeclareMathOperator{\Tr}{Tr}
\DeclareMathOperator{\Var}{Var}
\begin{document}
\allowdisplaybreaks % Seitenumbrüche in Formeln erlauben
\begin{center}
\bfseries % Fettdruck einschalten
\sffamily % Serifenlose Schrift
\vspace{-40pt}
Theoretische Elektrodynamik, Sommersemester 2014, Aufgabenblatt 4

Markus Fenske, Mattis Riediger, Tutor: Clemens Meyer zu Rheda
\vspace{-10pt}
\end{center}

\section*{Aufgabe 1: Geladene Kugeln}
Aus $\nabla E = \frac{\rho}{\epsilon_0}$ folgt per Gaußschem Satz
\begin{equation}
  \oint\limits_{\partial V} \dif \vec{a} \cdot \vec{E} = \int\limits_{V} \dif V \; \frac{\rho}{\epsilon_0} = \frac{Q(V)}{\epsilon_0}
\end{equation}
dabei soll $Q(V)$ die im Volumen $V$ eingeschlossene Ladung sein.

Da $\rho(r)$ nur von $r$ abhängt, ist das gesamte Problem Kugelsymmetrisch.
Das bedeutet, Kugelschalen müssen Äquipotentialflächen bilden, auf denen
$\vec{E}(r)$ senkrecht steht. Somit ist
\begin{align}
  &E(r) \underbrace{\oint\limits_{\partial V} \dif A}_{\mathclap{= \text{ Kugeloberfläche: } 4 \pi r^2}} = \frac{Q(V)}{\epsilon_0} \\
  \Leftrightarrow\quad
  & \vec{E}(r) = \frac{Q(r)}{4 \pi \epsilon_0 r^2} \hat{e_r}
\end{align}

Dies gilt für alle rotationssymmetrischen elektrostatischen Probleme.

\begin{enumerate}[a)]
  \item
    In diesem Fall ist
    \begin{equation}
      Q(r) = \begin{cases}
        0 & \text{ für } r < R \\
        Q & \text{ für } r \ge R
      \end{cases}
    \end{equation}

    also
    \begin{equation}
      \vec{E}(r) = \begin{cases}
        % FIXME: Evil typographic hack.
        \; \; \; \; 0 & \text{ für } r < R
        \vspace{0.2 cm} \\
        \dfrac{Q}{4 \pi \epsilon_0 r^2} \hat{e}_r & \text{ für } r \ge R
      \end{cases}
    \end{equation}

    % TODO: Skizze. "Skizzieren Sie das elektrische Feld" kann auch sein, den
    % Betrag skizzieren und die Richung angeben? :)
  \item
    Für $n = -2$ kann kein Koeffizient berechnet werden, denn $\rho(r) =
    c_{-2}r^{-2}$ ist im Interval $[0, R]$ unbeschränkt (außer $c_{-2} = 0$
    oder $\partial_r c_{-2} \neq 0$ -- ersteres bedeutet keine Ladung, ist also
    Unfug, letzteres ist keine Konstante in $r$ mehr, also geschummelt). Das
    Integral konvergiert nicht.

    Für $n = 2$ gilt
    \begin{align}
      &Q = \int\limits_0^\infty \dif r \; \rho(r) \\
      \Leftrightarrow\quad
      &Q = \int\limits_0^R \dif r \; c_2r^2 \\
      \Leftrightarrow\quad
      &c_2 = \frac{1}{Q} \del{\int\limits_0^R \dif r \; r^2}^{-1} \\
      \Leftrightarrow\quad
      &c_2 = Q \frac{3}{R^3}
    \end{align}

    Wir leiten nun die Funktion $Q(r)$ her, die die in der Kugel mit Radius $r$
    eingeschlossene Ladung angibt.

    \begin{align}
      &Q(r) = \int\limits_0^r \dif r' \; \rho(r') \\
      \Leftrightarrow\quad
      &Q(r) = Q \frac{3}{R^3} \frac{r^3}{3} \\
      \Leftrightarrow\quad
      &Q(r) = Q \frac{r^3}{R^3}
    \end{align}

    Für $r > R$ ist natürlich $Q(r) = Q$.

    Mit der zu Beginn der Aufgabe hergeleiteten Formel ergibt sich das elektrische Feld
    \begin{equation}
      \vec{E}(r) = \begin{cases}
        \dfrac{Q}{4 \pi \epsilon_0 R^3} r \hat{e}_r & \text{ für } r < R
        \vspace{0.2 cm} \\
        \dfrac{Q}{4 \pi \epsilon_0 r^2} \hat{e}_r & \text{ für } r \ge R
      \end{cases}
    \end{equation}

    %FIXME: Skizze!
\end{enumerate}
\end{document}
