\documentclass[a4paper,german,12pt,smallheadings]{scrartcl}
\usepackage[T1]{fontenc}
\usepackage[utf8]{inputenc}
\usepackage{babel}
\usepackage{geometry}
\usepackage[fleqn]{mathtools} % also includes mathclap
\usepackage[fleqn]{amsmath}
\usepackage{amssymb}
\usepackage{float}
\usepackage{enumerate}
\usepackage{commath} % http://tex.stackexchange.com/questions/14821/whats-the-proper-way-to-typeset-a-differential-operator
\usepackage{cancel}

% Number only referenced equations
\mathtoolsset{showonlyrefs}

%\usepackage{wrapfig}
\usepackage[thinspace,thinqspace,squaren,textstyle]{SIunits}

% New command for color underlining
\usepackage{xcolor}

\newsavebox\MBox
\newcommand\colul[2][red]{{\sbox\MBox{$#2$}%
  \rlap{\usebox\MBox}\color{#1}\rule[-1.2\dp\MBox]{\wd\MBox}{0.5pt}}}

\restylefloat{table}
\geometry{a4paper, top=15mm, left=10mm, right=20mm, bottom=20mm, headsep=10mm, footskip=12mm}
\linespread{1.2}
\setlength\parindent{0pt}
\DeclareMathOperator{\Tr}{Tr}
\DeclareMathOperator{\Var}{Var}
\newcommand*\laplace{\mathop{}\!\mathbin\Delta}
\begin{document}
\allowdisplaybreaks % Seitenumbrüche in Formeln erlauben
\begin{center}
\bfseries % Fettdruck einschalten
\sffamily % Serifenlose Schrift
\vspace{-40pt}
Theoretische Elektrodynamik, Sommersemester 2014, Aufgabenblatt 7

Markus Fenske, Mattis Riediger, Tutor: Clemens Meyer zu Rheda
\vspace{-10pt}
\end{center}

\section*{Aufgabe 1: Kugelschale}
Die allgemeine zylindersymmetrische Lösung der Laplace-Gleichung ist
\begin{equation}
  \Phi(r, \theta) = \sum_{l=0}^\infty \del{A_lr^l + \frac{B_l}{r^{l+1}}} P_l(\cos \theta)
\end{equation}

Heuristisch schmeißen wir die Terme für $l>2$ weg (weil wir keine Kosinusterme
höherer Ordnung erwarten) und setzen auch direkt $A_0 = \phi_0, B_0 = 0$. Somit
erhalten wir
\begin{equation}
  \Phi(r, \theta) = \del{Ar + \frac{B}{r^2}} \cos \theta + \phi_0
\end{equation}

Wir fangen mit dem Außenbereich $r > R_2$ an. Das Potential darf nicht gegen
unendlich gehen, also gilt für den Außenbereich offentsichtlich $A=0$.

Aus der Oberflächenladungsdichte bauen wir uns über $\sigma = - \epsilon_0
\partial_r \phi$ das Oberflächenpotential

\begin{equation}
  \phi(r, \theta) \overset{!}{=} - \frac{r k \cos(\theta)}{\epsilon_0} + C \quad\text{ (bei $r=R_2$)}
\end{equation}

Somit
\begin{align*}
  -\frac{R_2 k \cos \theta}{\epsilon_0} + C = \frac{B}{R_2^2} \cos(\theta) + \phi_0
\end{align*}

Mit $C=\phi_0$ und Kürzen von $\cos \theta$ erhalten wir
\begin{align*}
  -\frac{R_2 k}{\epsilon_0} = \frac{B}{R_2^2} \Rightarrow B = -\frac{k}{\epsilon_0} R_2^3
\end{align*}

Somit ist für den Außenbereich

\begin{align*}
  \phi_\text{A}(r,\theta) &= -\frac{k}{\epsilon_0} \frac{R_2^3}{r^2} \cos(\theta) + \phi_0
\end{align*}

Für den Innenbereich muss das Potential stetig an das Potential des
Außenbereichs anschließen und gleichzeitig $\phi(R_1,\theta) = \phi_0$ erfüllt
werden. Daraus folgt das Gleichungssystem
\begin{align*}
  -\frac{R_2 k \cos(\theta)}{\epsilon_0} + \phi_0 &= \del{AR_2 + \frac{B}{R_2^2}} \cos(\theta) + \phi_0 \\
                                           \phi_0 &= \del{AR_1 + \frac{B}{R_1^2}} \cos(\theta) + \phi_0
\end{align*}

Aus der zweiten Gleichung folgt $AR_1 + \dfrac{B}{R_1^2} = 0$, also $B =
-AR_1^3$. Eingesetzt in die erste Gleichung und durch Kürzung folgt
\begin{equation}
  -\frac{R_2 k}{\epsilon_0} = A \del{R_2 - \frac{R_1^3}{R_2^2}} \Rightarrow
  A = -\frac{k}{\epsilon_0} \frac{1}{(1- \del{\frac{R_1}{R_2}}^3}
\end{equation}

Und somit für den Innenbereich
\begin{equation}
  \phi_I(r, \theta) = -\frac{k}{\epsilon_0} \frac{1}{1-\del{\frac{R_1}{R_2}}^3} \del{r - \frac{R_1^3}{r^2}} \cos(\theta) + \phi_0
\end{equation}

\section*{Aufgabe 3: Leiterschleife}

Die Leiterschleife befinde sich in einem ruhenden Inertialsystem $S$. Der
Ursprung liege im Mittelpunkt des Halbkreises. Dort befinde sich eine
Probeladung $q$, die sich mit der Geschwindigkeit $v$ in $x$-Richtung bewege.
Wenn wir die Kraft auf diese Ladung kennen würden, könnten wir über die
Lorentzkraft $\vec{F} = q\vec{v} \times \vec{B}$ das Magnetfeld berechnen. Das
machen wir also.

Der Draht ist elektrisch neutral. Das bedeutet, die Linienladungsdichte
$\lambda$ verschwindet. Sei $\lambda_-$ die Linienladungsdichte, die durch die
Elektronen erzeugt wird und $\lambda_+$ die Linienladungsdichte, die durch die
Protonen erzeugt wird. Dann ist

\begin{equation}
  \lambda = \lambda_- + \lambda_+ = 0
\end{equation}

Damit die Elektronen einen Strom $I$ hervorrufen, müssen sie sich bewegen, denn
Strom ist Ladung pro Zeit. Wir sehen:

\begin{equation}
  I = \frac{dQ}{dt} = \frac{dQ}{dx} \frac{dx}{dt} = \lambda v
\end{equation}

Wir können die Geschwindigkeit $v$ also frei wählen, solange wir entsprechend
$\lambda = \frac{I}{v}$ setzen. Der Einfachheit halber bewegen sich also die
Elektronen mit der selben Geschwindigkeit $v$ wie die Probeladung.

Wir transformieren nun in ein mit $v$ bewegtes Inertialsystem $S'$.

Betrachten wir den oberen Draht in $S'$. Dort bewegen sich die Elektronen nun
nicht mehr. Auch die Ladung $q$ ist unbewegt, also wirkt auf sie kein
Magnetfeld und die Geschwindigkeit der Protonen ist uns deswegen egal. Die neue
Linienladungsdichte ist aufgrund der Längenkontraktion in $x$-Richtung.
\begin{equation}
  \lambda_+' = \frac{\lambda_+}{\sqrt{1 - \frac{v^2}{c^2}}}
\end{equation}

Die Ruhedichte der Elektronen liegt im System $S'$, also muss die
Elektronendichte genau anders herum transformiert werden (von bewegt zu
unbewegt):
\begin{equation}
  \lambda_-' = \lambda_- \sqrt{1 - \frac{v^2}{c^2}}
\end{equation}

Daraus ergibt sich, dass der Draht in diesem Inertialsystem geladen ist. Seine
Linienladungsdichte ist nämlich genau
\begin{align*}
  \lambda' &= \lambda_+' + \lambda_-' \\
       &= \frac{\lambda_+}{\sqrt{1 - \frac{v^2}{c^2}}} + \lambda_- \sqrt{1 - \frac{v^2}{c^2}} \\
       &= \lambda_{+} \frac{v^2}{c^2} \frac{1}{\sqrt{1-\frac{v^2}{c^2}}}
\end{align*}

Daraus folgt, dass auf die Ladung $q$ nun eine elektrische Kraft $\vec{F} = q
\vec{E}$ wirkt. Wir integrieren also über homogen geladenen den Draht entlang
der $x$-Achse von $-\infty$ bis $0$.

Wenn $\vec{r}$ der Vektor zum Draht ist, ist $r_y = R$ und $r_x = R
\tan(\theta)$. Dann muss für die Integration $\theta$ von $0$ bis
$\frac{\pi}{2}$ laufen, also ist das elektrische Feld:

% TODO: Zeichnung hier:
%                           x
% -------------------------------.
%                       \        |
%                        \       |
%                         \      | R
%                          \     |
%                           \    |
%                            \   |
%                             \-^|
%                              \ | <-- \theta
%                               \|
%                                q

\begin{equation}
  4 \pi \epsilon_0 \vec{E} =
  \int \frac{\rho'}{r^3} \vec{r} =
  \int_0^\frac{\pi}{2} \dif \theta \; \frac{\lambda'}{\sqrt{R^2 \del{\tan^2(\theta) +1}}^3}
  \begin{pmatrix}
    R \tan \theta \\
    R
  \end{pmatrix}
  =
  \int_0^\frac{\pi}{2} \dif \theta \; \frac{\lambda'}{\sqrt{\del{\tan^2(\theta) +1}}^3}
  \begin{pmatrix}
    \tan \theta \\
    1
  \end{pmatrix}
\end{equation}

% WA: int from 0 to pi/2 of tan(x)/sqrt(tan^2(x)+1)^3 = 1/3
% WA: int from 0 to pi/2 of 1/sqrt(tan^2(x)+1)^3 = 2/3
Per Maxima berechnen wir die Integrale als $\frac{1}{3}$ bzw. $\frac{2}{3}$ und
erhalten das elektrische Feld des oberen Drahtes im Ursprung.
\begin{equation}
  \vec{E}_1' = \frac{\lambda'}{12\pi \epsilon_0} \begin{pmatrix} 1 \\ 2 \end{pmatrix}
  = \frac{\lambda_+ v^2}{12 \pi \epsilon_0 c^2} \frac{1}{\sqrt{1 - \frac{v^2}{c^2}}} \begin{pmatrix} 1 \\ 2 \end{pmatrix}
\end{equation}

Wegen $\lambda_+ = -\lambda_-$ und $I = \lambda v$

\begin{equation}
  \vec{E}_1'
  = \frac{-v}{12 \pi \epsilon_0 c^2} \frac{1}{\sqrt{1 - \frac{v^2}{c^2}}} \begin{pmatrix} 1 \\ 2 \end{pmatrix}
\end{equation}

Mit $\mu_0 = \frac{1}{\epsilon_0 c^2}$:
\begin{equation}
  \vec{E}_1'
  = \frac{-v \mu_0}{12 \pi} \frac{1}{\sqrt{1 - \frac{v^2}{c^2}}} \begin{pmatrix} 1 \\ 2 \end{pmatrix}
\end{equation}

und mit $\vec{F} = q\vec{E}$:
\begin{equation}
  \vec{F}_1'
  = \frac{-qv \mu_0}{12 \pi} \frac{1}{\sqrt{1 - \frac{v^2}{c^2}}} \begin{pmatrix} 1 \\ 2 \end{pmatrix}
\end{equation}

Wir transformieren die Kraft zurück ins System $S$:
\begin{equation}
  \vec{F}_1
  = \frac{-qv \mu_0}{12 \pi} \begin{pmatrix} 1 \\ 2 \end{pmatrix}
\end{equation}

% TODO: Hieraus (oder später aus der Summe) das Magnetfeld berechnen.



\end{document}
